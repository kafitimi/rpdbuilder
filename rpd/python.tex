\documentclass[a4paper,12pt]{article}
\usepackage[hmargin={30mm, 17mm}, vmargin={20mm,20mm}, nohead]{geometry}
\usepackage{polyglossia, enumitem, array, multirow, tabularx, rotating, indentfirst, caption, float, ulem, titlesec, longtable }%}
\setdefaultlanguage{russian} % выбор основного языка (для переносов)
% шрифты 
  \defaultfontfeatures{Ligatures=TeX} % нужен для того, чтобы работали стандартные сочетания символов ---, -- << >> и т.п.
  \setmainfont{Times New Roman} 
  \setmonofont[Scale=0.9]{Consolas}
  \setsansfont{Trebuchet MS}



% pdf metadata
\usepackage[
  pdfencoding=unicode,
  pdftitle={Рабочая программа дисциплины Программирование на языке Питон},
  pdfauthor={составлена }, 
  bookmarksnumbered=true, 
    colorlinks=true, linktocpage=true, linkcolor=blue, pdfpagemode=UseOutlines]
{hyperref}

% формат заголовков, подписей и списков
  \titleformat*{\section}{\large\bfseries\centering}
  \titleformat*{\subsection}{\bfseries}
  \captionsetup[table]{singlelinecheck=off, font=small, labelsep=period, textfont=it, format=hang, justification=raggedleft}
  \setlist[enumerate,1]{nolistsep,labelindent=0pt,leftmargin=\parindent}
  \setlist[itemize,1]{nolistsep,labelindent=20pt,leftmargin=*,label=--}

\makeatletter

% заполняет ширину текста полем с подчеркиванием.
% #1 - текст слева, #2 - на подчеркивании, #3 - справа, #4 - подпись под подчеркнутым
\newcommand{\ulfield}[4]{
\noindent
\begin{tabularx}{\linewidth}{@{}l@{}X@{}l@{}}
#1\if\relax\detokenize{#1}\relax\else\,~\vrule width 0pt\fi 
& \uline{\vrule width 0pt\hfill#2\hfill\vrule width 0pt} & 
\if\relax\detokenize{#3}\relax\else\vrule width 0pt~\,\fi #3
\\
& {\scriptsize \vrule width 0pt\hfill#4\hfill\vrule width 0pt}
\end{tabularx}
}

%%% Перенос составных слов
  \XeTeXinterchartokenstate=1
  \XeTeXcharclass `\- 24
  \XeTeXinterchartoks 24 0 = {\hskip\z@skip}
  \XeTeXinterchartoks 0 24 = {\nobreak}

\makeatother

\newcommand{\datefield}[1][]{
\if\relax\detokenize{#1}\relax%
«\uline{\hspace{22pt}}»~\uline{\hspace{90pt}}\,~20\uline{\hspace{20pt}}~г.\else
«\uline{\hspace{18pt}}»~\uline{\hspace{60pt}}\,~20\uline{\hspace{18pt}}~г.\fi
}

\newcommand\rotleft{\rotatebox{90}}


\newenvironment{srstable}{
\begin{table}[H]
    \caption{Содержание самостоятельной работы студентов}
    \begin{tabular}{|r|p{26ex}|l|r|r|p{9ex}|}
    \hline
    № & 
    \multicolumn{1}{p{37ex}|}{\centering Наименование раздела, темы} & 
    \multicolumn{1}{c|}{Виды} & 
    \multicolumn{1}{p{6ex}|}{\centering Объем часов} & 
    \multicolumn{1}{p{14ex}|}{\centering Используемые ресурсы} & 
    Контроль\\%
\hline}{\\ \hline\end{tabular}\end{table}}
    
%%% Настройка содержания
\AtBeginDocument{%
\makeatletter 
\def\@tocline#1#2#3#4#5#6#7{\relax
  \ifnum #1>\c@tocdepth % then omit
  \else
    \par \addpenalty\@secpenalty\addvspace{#2}%
    \begingroup \hyphenpenalty\@M
    \@ifempty{#4}{%
      \@tempdima\csname r@tocindent\number#1\endcsname\relax
    }{%
      \@tempdima#4\relax
    }%
    \parindent\z@ \leftskip#3\relax \advance\leftskip\@tempdima\relax
    \rightskip\@pnumwidth plus4em \parfillskip-\@pnumwidth
    #5\leavevmode\hskip-\@tempdima
      \ifcase #1
       \or\or \hskip 1em \or \hskip 2em \else \hskip 3em \fi%
      #6\nobreak\relax
    \dotfill\hbox to\@pnumwidth{\@tocpagenum{#7}}\par
    \nobreak
    \endgroup
  \fi}
}%AtBeginDocument

\begin{document}

\thispagestyle{empty}

\noindent
\begin{center}
Министерство образования и науки Российской Федерации \\
Федеральное государственное автономное образовательное \\
учреждение высшего образования\\
«СЕВЕРО-ВОСТОЧНЫЙ ФЕДЕРАЛЬНЫЙ УНИВЕРСИТЕТ \\имени М.\,К.~АММОСОВА» \\
Институт математики и информатики \\
Кафедра информационных технологий

\vspace{12mm}
\begin{flushright}
\parbox{80mm}{
УТВЕРЖДАЮ\\
Директор ИМИ\\[2mm]
\ulfield{}{}{/\,В.\,И.~Афанасьева\,/}{}\\
\datefield
\\[20mm]
}
\end{flushright}


\textbf{РАБОЧАЯ ПРОГРАММА ДИСЦИПЛИНЫ}
\\[2mm]
\textbf{Б1.В.ДВ.3.1\ -- Программирование на языке Питон} 
\\[5mm]

для программы магистратуры\\
по направлению подготовки \\
09.04.01 -- Информатика и вычислительная техника
\\[15mm]

\begin{tabular}{p{0.3\textwidth}p{0.3\textwidth}p{0.3\textwidth}}
  ОДОБРЕНО &  ОДОБРЕНО  & РЕКОМЕНДОВАНО \\
  Заведующий кафедрой \newline разработчика &
  Заведующий выпускающей кафедрой ИТ&
  Нормоконтроль в составе ОП пройден \\
  \ulfield{}{}{\uline{/\hspace{30mm}/}}{} &
  \ulfield{}{}{\uline{/\hspace{30mm}/}}{} &
  \ulfield{}{}{\uline{/\hspace{30mm}/}}{} \\
  Протокол № \uline{\hspace{13pt}} от\newline \datefield[small] & 
  Протокол № \uline{\hspace{13pt}} от\newline \datefield[small] & 
  Протокол № \uline{\hspace{13pt}} от\newline \datefield[small] 
\end{tabular}
\par\vfill\vspace{6mm}
Якутск -- 2016

\end{center}


\newpage


\begin{center}
\section{АННОТАЦИЯ}
  {\bf к рабочей программе дисциплины\\
  Б1.В.ДВ.3.1\ -- Программирование на языке Питон} \\
  Трудоемкость \uline{~2~} з.~е.
\end{center}


\subsection{Цель освоения и краткое содержание дисциплины}
  
  Целью изучение дициплины <<Программирование на языке Питон>> является: изучение и получение практических навыков использования языка Питон (Python), в том числе для анализа данных.
  


\subsection{Перечень планируемых результатов обучения по дисциплине, соотнесенных с планируемыми результатами освоения образовательной программы.}

  \begin{longtable}{|p{70mm}|p{84mm}|}
  \caption{Перечень планируемых результатов обучения}\\
  \hline
  \centering
  Планируемые результаты освоения программы (содержание и коды компетенций) & 
  \centering\arraybackslash
  Планируемые результаты обучения по~дисциплине
  \\
  \hline
  
  ПК-4 : владением существующими методами и алгоритмами решения задач распознавания и обработки данных, \par 
  
  ПК-12 : способностью выбирать методы и разрабатывать алгоритмы решения задач управления и проектирования объектов автоматизации
  & 
  В результате изучения дисциплины обучающийся должен:\newline
  \emph{знать:}
  \begin{itemize}
    \item основы синтаксиса Python; 
    \item основные возможности стандартной библиотеки Python; 
    \item основные принципы функционального подхода к программированию; 
    \item основные принципы многозвенной архитектуры веб-приложений; 
  \end{itemize}
  

  \emph{уметь:}
  \begin{itemize}
    \item создавать консольные приложения на Python; 
    \item использовать элементы объектно-ориентированного и функционального программирования; 
    \item пользоваться средствами библиотек для анализа данных и визуализации результатов; 
    \item концептуально разделять представление, бизнес-логику и модели данных для веб-приложений; 
  \end{itemize}
  

  \emph{владеть навыками:}
  \begin{itemize}
    \item установки пакетов средствами пакетного менеждера pip; 
    \item использования IPython Notebook (Jupyter Notebook) для интерактивного анализа данных. 
  \end{itemize}
  
  \\
  \hline
  \end{longtable}


\subsection{Место дисциплины в~структуре образовательной программы}

  \begin{table}[H]
  \setlength\arraycolsep{3pt}
  \caption{Содержательно-логические связи дисциплины}
  \begin{tabular}{|l|p{18ex}|*{2}{p{23ex}|}}
  \hline
  \multicolumn{1}{|c|}{\multirow{2}{13ex}{\centering Индекс \linebreak дисциплины}} &
  \multicolumn{1}{c|}{\multirow{2}{18ex}{\centering Наименование \linebreak дисциплины}} & 
  \multicolumn{2}{p{46ex}|}{\centering Коды учебных дисциплин, практик} \\
  \cline{3-4}
   & & 
  \centering на которые опирается содержание дисциплины & 
  \centering\arraybackslash для которых содержание дисциплины выступает опорой
  \\ \hline
  Б1.В.ДВ.3.1 & Программирование на языке Питон 
  & 
  
  Б1.В.ОД.2.1 -- Объектно-ориентированное программирование 
  & 
  
  Б1.В.ДВ.7.1 -- Визуализация в научных исследованиях 
  \\ \hline
  \end{tabular}
  \end{table}


\subsection{Язык преподавания} 
  Русский.
  



\newpage

\section{Объем дисциплины в зачетных единицах с указанием количества академических часов, выделенных на контактную работу обучающихся с преподавателем (по~видам учебных занятий) и~на~самостоятельную работу обучающихся}

\begin{table}[H]
\caption{Выписка из учебного плана} 
\begin{tabular}{|p{9cm}|c|c|}
\hline
Код и название дисциплины по учебному плану & \multicolumn{2}{p{6cm}|}{Б1.В.ДВ.3.1\ -- Программирование на языке Питон }\\
\hline
Курс изучения &\multicolumn{2}{c|}{ 1 }\\
\hline
Семестр(ы) изучения &\multicolumn{2}{c|}{ 2 }\\
\hline
Форма промежуточной аттестации (зачет/экзамен) &\multicolumn{2}{c|}{ зачет }\\
\hline
Курсовой проект / курсовая работа (указать вид работы при наличии в учебном плане), семестр выполнения &\multicolumn{2}{c|}{ }\\
\hline
Трудоемкость (в ЗЕТ) &\multicolumn{2}{c|}{ 2 (2) }\\
\hline
{\bf Трудоемкость (в часах)} (сумма строк №1, 2, 3), в~т.~ч.:& \multicolumn{2}{c|}{72}\\
\hline
\textbf{№\,1. Контактная работа обучающихся с преподавателем (КР),} в часах:
& \multicolumn{1}{p{3cm}|}{\centering Объем аудиторной работы, в часаx}
& \multicolumn{1}{p{3cm}|}{\centering\arraybackslash В~т.\,ч. с~применением ДОТ или ЭО, в~часах}\\
\hline  
Объем работы (в часах) (1.1.+1.2.+1.3.)& 25 & \\
\hline
1.1. Занятия лекционного типа (лекции) & 6 & \\
\hline
1.2. Занятия семинарского типа, всего, в т.ч.: & & \\
\hline
- семинары (практические занятия, коллоквиумы и~т.~п.)  & 14 & \\
\hline
- лабораторные работы& – & \\
\hline
- практикумы & & \\
\hline
1.3. КСР (контроль самостоятельной работы, консультации)& 5 & \\
\hline
{\bf №\,2. Самостоятельная работа обучающихся (СРС) (в часах)}& \multicolumn{2}{c|}{47}\\
\hline
{\bf №\,3. Количество часов на экзамен (при наличии экзамена в учебном плане)}& \multicolumn{2}{c|}{–}\\
\hline
\end{tabular}
\end{table}



\newpage
\section{Содержание дисциплины, структурированное по~темам с~указанием отведенного на~них количества академических часов и~видов учебных занятий}
\subsection{Распределение часов по~темам и~видам учебных занятий}
\begin{longtable}{|>{\raggedright\arraybackslash}p{59mm}|c|c|c|c|c|c|c|c|c|c|c|}
\caption{}
\\
\hline
 & & 
\multicolumn{9}{c|}{Контактная работа, в часах} & 
\\
\cline{3-11} 
\raisebox{18mm}{Тема}&
\rotleft{Всего часов} &
\rotleft{Лекции} &
\rotleft{из них с прим-м  ЭО и ДОТ} &
\rotleft{\parbox{5cm}{\raggedright\arraybackslash Семинары  (практические занятия, коллоквиумы)}} &
\rotleft{из них с прим-м  ЭО и ДОТ} &
\rotleft{Лабораторные работы} &
\rotleft{из них с прим-м  ЭО и ДОТ} &
\rotleft{Практикумы} &
\rotleft{из них с прим-м  ЭО и ДОТ} &
\rotleft{КСР (консультации)} & 
\rotleft{Часы СРС}
\\
\hline
Тема 1. Введение в Python                           & 21 & 2 & 1 & 3 & 2 & 0 & 0 & 0 & 0 & 1 & 12 \\ 
\hline
Тема 2. Python как скриптовый и интерактивный язык  & 12 & 0 & 0 & 3 & 2 & 0 & 0 & 0 & 0 & 1 & 6 \\ 
\hline
Тема 3. Веб-приложения на Python                    & 18 & 2 & 0 & 3 & 0 & 0 & 0 & 0 & 0 & 1 & 12 \\ 
\hline
Тема 4. Анализ данных с использованием pandas       & 20 & 2 & 1 & 3 & 2 & 0 & 0 & 0 & 0 & 1 & 11 \\ 
\hline
Тема 5. Визуализация в matplotlib                   & 9 & 0 & 0 & 2 & 0 & 0 & 0 & 0 & 0 & 1 & 6 \\ 
\hline
ВСЕГО ЧАСОВ & 72 & 6 & 2 & 14 & 6 & 0 & 0 & 0 & 0 & 5 & 47 \\ 

\hline
\end{longtable}

\subsection{Содержание тем программы дисциплины} 


\textbf{Тема 1. Введение в Python                          }\\
Основы синтаксиса. Циклы, ветвления. Основные атомарные типы, строки,
кортежи, списки, словари. Функции. Области видимости имен. Работа с
файлами. Объекты. Аннотации. Пакеты и модули. Ссылочная прозрачность,
побочные эффекты и чистые функции. Функциональный аспект языка Питон.
Модуль functools.


\textbf{Тема 2. Python как скриптовый и интерактивный язык }\\
Стандартная библиотека. Модули math, os.path. Менеджер пакетов pip.
Установка, обновление и удаление пакетов. Портал PyPI. Сайты,
изоляция при помощи virtualenv. IPython. Jupyter Notebook (IPython
Notebook).


\textbf{Тема 3. Веб-приложения на Python                   }\\
Трехзвенная архитектура веб-приложений. MVC, MVVC. Фреймворк Django.
Микрофреймворк Flask. HTML-шаблоны. Шаблонизатор Jinja2.
Уровень представления. URL-маршрутизация. Уровень модели. Фреймворк
Pyramid (Pylons).


\textbf{Тема 4. Анализ данных с использованием pandas      }\\
Библиотека NumPy. Библиотека SciPy. Библиотека pandas. Кадры данных
(dataframes). Применение функций к кадрам данных. Очистка данных.
Многомерные данные. Агрегация. Линейная регрессия.


\textbf{Тема 5. Визуализация в matplotlib                  }\\
Библиотека matplotlib. Графики для одномерных данных. Графики для
двумерных данных.

 

\subsection{Формы и методы проведения занятий, применяемые учебные технологии}
При проведении занятий и организации СРС используются традиционныетехнологии сообщающего обучения, предполагающие передачу информациив~готовом виде: проведение лекционных занятий, самостоятельная работас~источниками. Предусмотрено использование активных и интерактивных формобучения с целью формирования и развития профессиональных навыковстудентов - выполнение практических работ с применением компьютерныхтехнологий.



\section{Перечень учебно-методического обеспечения для самостоятельной работы обучающихся по дисциплине}
\begin{longtable}{|l|>{\raggedright\arraybackslash}p{40mm}|>{\raggedright\arraybackslash}p{54mm}|c|>{\raggedright\arraybackslash}p{30mm}|}
\hline
№ & \centering Наименование раздела (темы) дисциплины & 
\centering Вид СРС & \multicolumn{1}{p{14mm}|}{\centering Трудо\-емкость (в часах)} & \centering\arraybackslash Формы и методы контроля\\
\hline
1 & Введение в Python                           & Прохождение онлайн-курса, покрывающего основы Python & 12 & Предъявление веб-страницы с информацией о прохождении теста \\ 
\hline
2 & Python как скриптовый и интерактивный язык  & Сдача домашнего задания & 6 & Публикация кода в репозитории на сайте GitHub \\ 
\hline
3 & Веб-приложения на Python                    & Разработка простого сайта & 12 & Публикация кода в репозитории на сайте GitHub \\ 
\hline
4 & Анализ данных с использованием pandas       & Прохождение вводной части курса на сайте edx.org & 11 & Предъявление веб-страницы с информацией о прохождении, публикация кода в репозитории на сайте GitHub \\ 
\hline
5 & Визуализация в matplotlib                   & Построение графиков & 6 & Публикация блокнота IPython с графиками в репозитории на сайте GitHub \\ 
\hline
 & ИТОГО                                       &  & 47 &  \\ 

\hline
\end{longtable}


\section{Методические указания для обучающихся по освоению дисциплины}
В связи с небольшим объемом аудиторных часов, важное значение в освоении
дисциплины занимает самостоятельная работа. Она предполагает в том числе
и сдачу частей онлайн-курсов, некоторые из них на английском языке. Это
требует самостоятельности и ответственности.
\par
В диагностическом разделе дисциплины приведены тесты по каждому модулю
дисциплины, которые необходимо выполнить для закрепления теоретических
знаний.
\par
Последовательная и добросовестная самостоятельная работа является 
основой для выработки практических навыков использования гибкого и
мультипарадигменного языка программирования, который с успехом может
быть применен для решения различных задач в областях деятельности,
предполагаемых стандартом подготовки по направлению
«Информатика и вычислительная техника».



\subsubsection*{Рейтинговый регламент по дисциплине}
\begin{longtable}{|>{\raggedright\arraybackslash}p{110mm}|r|r|}
\hline
\centering\arraybackslash Вид выполняемой учебной работы (контролирующие мероприятия) & 
\multicolumn{1}{p{20mm}|}{\centering\arraybackslash Количество баллов (min)} & 
\multicolumn{1}{p{20mm}|}{\centering\arraybackslash Количество баллов (max)} \\
\hline
Посещаемость & 5 & 10 \\ 
\hline
Домашние задания, онлайн курсы & 25 & 45 \\ 
\hline
Практические занятия & 10 & 15 \\ 
\hline
Тестирование & 20 & 30 \\ 
\hline
Количество баллов для получения зачета (min-max) & 60 & 100 \\ 

\hline
\end{longtable}


\section{Фонд оценочных средств для проведения промежуточной аттестации обучающихся по дисциплине}

\subsection{Показатели, критерии и шкала оценивания}
\subsection{Типовые контрольные задания (вопросы) для промежуточной аттестации}
\subsection{Методические материалы, определяющие процедуры оценивания}


\newpage
\section{Перечень основной и дополнительной учебной литературы, необходимой для освоения дисциплины}

  \begin{longtable}{|l|p{7cm}|p{18mm}|c|p{32mm}|}
  \caption*{Перечень литературы}\\
  \hline
  № & 
  \centering\small\arraybackslash Автор, название, место издания, издательство, год издания учебной литературы, вид и характеристика иных информационных ресурсов &
  \multicolumn{1}{p{18mm}|}{\centering\small\arraybackslash Наличие грифа, вид грифа} &
  \multicolumn{1}{p{21mm}|}{\centering\small\arraybackslash НБ СВФУ, кафедральная библиотека и кол-во экземпляров} & 
  \centering\small\arraybackslash Электронные издания: точка доступа к ресурсу (наименование ЭБС, ЭБ СВФУ)\\
  \hline
  \multicolumn{5}{|c|}{Основная литература}\\
  \hline
  1 &\raggedright\arraybackslash Степанов, Ю.А. Алгоритмизация и программирование. [Электронный ресурс] — НФИ КемГУ, 2013.  &    &  ---  & ЭБС <<Лань>>: https:// e.lanbook.com/ 
  \\
  \hline
  
  \multicolumn{5}{|c|}{Дополнительная литература}\\
  \hline
  1 &\raggedright\arraybackslash Соловьев И.А., Червяков А.В., Репин А.Ю. Вычислительная математика на смартфонах, коммуникаторах и ноутбуках с использованием программных сред Python. [Электронный ресурс]  М.: Лань, 2011.  &    &  ---  & ЭБС <<Лань>>: https:// e.lanbook.com/ 
  \\
  \hline
  
  \end{longtable}
  
\section{Перечень ресурсов информационно-телекоммуникационной сети «Интернет» (далее сеть-Интернет), необходимых для освоения дисциплины}
\begin{enumerate}
  
  \item Онлайн-курс Using Python for Research. \\ Режим доступа: https://www.edx.org/course/using-python-research-harvardx-ph526x. 
  
  \item Компания Microsoft. Онлайн-курс Introduction to Python for Data Science. \\ Режим доступа: https://www.edx.org/course/introduction-python-data-science-microsoft-dat208x-3. 
  
  \item Компания Microsoft. Онлайн-курс Programming with Python for Data Science. \\ Режим доступа: https://www.edx.org/course/programming-python-data-science-microsoft-dat210x-1. 
  
  \item В. Дронов. Django: Практика создания Web-сайтов на Python. \\ Режим доступа: http:// www.litres.ru/vladimir-dronov/django-praktika-sozdaniya-web-saytov-na-python-19213409 
  
\end{enumerate}

\newpage
\section{Описание материально-технической базы, необходимой для осуществления образовательного процесса по дисциплине}
  
  
       Для проведения лекционных занятий требуется аудитория, оборудованная доской,  мультимедийным проектором с экраном. 
       Для проведения лабораторных занятий требуется компьютерный класс с подключением к интернету.
  
  


\section{Перечень информационных технологий, используемых при осуществлении образовательного процесса по дисциплине, включая перечень программного обеспечения
}

\subsection{Перечень информационных технологий, используемых при осуществлении образовательного процесса по дисциплине}

При осуществлении образовательного процесса по дисциплине используются следующие информационные технологии:
\begin{itemize}[nolistsep]
  
\item использование на занятиях электронных изданий (чтение лекций с использованием слайд-презентаций);
  
\item ведение учета посещаемости и выполнения учебных заданий в системе Google Docs;
  
\item разработка обучающимися программ на языке Python;
  
\item организация взаимодействия с обучающимися посредством электронной почты, специализированного образовательного форума Piazza;
  
\item компьютерное тестирование.
  
\end{itemize}

\subsection{Перечень программного обеспечения}
При осуществлении образовательного процесса по дисциплине используются следующее программное обеспечение:
\begin{itemize}[nolistsep]
  
\item язык Python версии 3.4 и новее;
  
\item менеджер пакетов pip для Python;
  
\item среда разработки JetBrains PyCharm;
  
\item менеджер версий Git;
  
\item интернет-браузер.
  
\end{itemize}



\newpage
\begin{center}
\section*{ЛИСТ АКТУАЛИЗАЦИИ РАБОЧЕЙ ПРОГРАММЫ ДИСЦИПЛИНЫ}
Б1.В.ДВ.3.1\ --- Программирование на языке Питон 
\end{center}

  \noindent
  \begin{tabular}{|p{15mm}|p{67mm}|p{25mm}|p{41mm}|}
    \hline
    \small\centering
    Учебный год 
    & \small\centering
    Внесенные изменения 
    & \small\centering
    Преподаватель (ФИО) 
    & \small\centering\arraybackslash
    Протокол заседания выпускающей кафедры (дата, номер), ФИО зав.кафедрой, подпись \\
    & & & \\\hline
    & & & \\\hline
    & & & \\\hline
    & & & \\\hline
    & & & \\\hline
    & & & \\\hline
    & & & \\\hline
    & & & \\\hline
    & & & \\\hline
    & & & \\\hline
    & & & \\\hline
    & & & \\\hline
    & & & \\\hline
    & & & \\\hline
    & & & \\\hline
    & & & \\\hline
    & & & \\\hline
    & & & \\\hline
    & & & \\\hline
    & & & \\\hline
    & & & \\\hline
    & & & \\\hline
    & & & \\\hline
    & & & \\\hline
    & & & \\\hline
    & & & \\\hline
    & & & \\\hline
    & & & \\\hline
    & & & \\\hline
    & & & \\\hline
    & & & \\\hline
    & & & \\\hline
    & & & \\\hline
    & & & \\\hline
    & & & \\\hline
    & & & \\\hline
  \end{tabular}

  \medskip\noindent\textit{В таблице указывается только характер изменений (например, изменение темы, списка источников по~теме или темам, средств промежуточного контроля) с~указанием пунктов рабочей программы. Само содержание изменений оформляется приложением по~сквозной нумерации.}

\newpage\tableofcontents

\end{document}