\documentclass[a4paper,12pt]{article}
  \usepackage[hmargin={30mm, 17mm}, vmargin={20mm,20mm}, nohead]{geometry}
  \usepackage{polyglossia, enumitem, array, longtable, multirow, tabularx, rotating, indentfirst, caption, float, ulem, titlesec }%}
  \setdefaultlanguage{russian} % выбор основного языка (для переносов)
  % шрифты 
  \defaultfontfeatures{Ligatures=TeX} % нужен для того, чтобы работали стандартные сочетания символов ---, -- << >> и т.п.
  \setmainfont{Times New Roman} 
  \setmonofont[Scale=0.9]{Consolas}
  \setsansfont{Trebuchet MS}

  
  \usepackage[vargreek-shape=unicode]{unicode-math}
  \setmathfont{latinmodern-math.otf}
  \setmathfont[range=\mathit/{latin,Latin}]{Times New Roman-Italic}
  \setmathfont[range=\mathit/{greek,Greek}]{Times New Roman}
  \setmathfont[range=\mathup]{Times New Roman}
  

  % pdf metadata
  \usepackage[
    pdfencoding=unicode,
    pdftitle={Рабочая программа дисциплины Теория автоматов и формальных языков},
    pdfauthor={составлена }, 
    bookmarksnumbered=true, 
      colorlinks=true, linktocpage=true, linkcolor=blue, pdfpagemode=UseOutlines]
  {hyperref}

  % формат заголовков, подписей и списков
  \titleformat*{\section}{\large\bfseries\centering}
  \titleformat*{\subsection}{\bfseries}
  \captionsetup[table]{singlelinecheck=off, font=small, labelsep=period, textfont=it, format=hang, justification=raggedleft}
  \setlist[enumerate,1]{nolistsep,labelindent=0pt,leftmargin=\parindent}
  \setlist[itemize,1]{nolistsep,labelindent=20pt,leftmargin=*,label=--}

  \makeatletter

  % заполняет ширину текста полем с подчеркиванием.
  % #1 - текст слева, #2 - на подчеркивании, #3 - справа
  \newcommand{\ulfield}[3]{
  \noindent
  \begin{tabularx}{\linewidth}{@{}l@{}X@{}l@{}}
  #1\if\relax\detokenize{#1}\relax\else\,~\vrule width 0pt\fi 
  & \uline{\vrule width 0pt\hfill#2\hfill\vrule width 0pt} & 
  \if\relax\detokenize{#3}\relax\else\vrule width 0pt~\,\fi #3
  \end{tabularx}
  }

  %%% Перенос составных слов
    \XeTeXinterchartokenstate=1
    \XeTeXcharclass `\- 24
    \XeTeXinterchartoks 24 0 = {\hskip\z@skip}
    \XeTeXinterchartoks 0 24 = {\nobreak}

  \newcommand\rotleft{\rotatebox{90}}
  
  \makeatother

  \newcommand{\datefield}[1][]{\if
  \relax\detokenize{#1}\relax
  «\uline{\hspace{22pt}}»~\uline{\hspace{90pt}}\,~20\uline{\hspace{20pt}}~г.\else 
  «\uline{\hspace{18pt}}»~\uline{\hspace{60pt}}\,~20\uline{\hspace{18pt}}~г.\fi
  }


%%% Настройка содержания
\AtBeginDocument{
  \makeatletter 
  \def\@tocline#1#2#3#4#5#6#7{\relax
  \ifnum #1>\c@tocdepth % then omit
  \else
    \par \addpenalty\@secpenalty\addvspace{#2}%
    \begingroup \hyphenpenalty\@M
    \@ifempty{#4}{\@tempdima\csname r@tocindent\number#1\endcsname\relax}{\@tempdima#4\relax}%
    \parindent\z@ \leftskip#3\relax \advance\leftskip\@tempdima\relax
    \rightskip\@pnumwidth plus4em \parfillskip-\@pnumwidth
    #5\leavevmode\hskip-\@tempdima
      \ifcase #1
       \or\or \hskip 1em \or \hskip 2em \else \hskip 3em \fi%
      #6\nobreak\relax
    \dotfill\hbox to\@pnumwidth{\@tocpagenum{#7}}\par
    \nobreak
    \endgroup
  \fi}
  \makeatother
}%AtBeginDocument

\begin{document}
\sloppy
\thispagestyle{empty}

\noindent
\begin{center}
Министерство образования и науки Российской Федерации \\
Федеральное государственное автономное образовательное \\
учреждение высшего образования\\
«СЕВЕРО-ВОСТОЧНЫЙ ФЕДЕРАЛЬНЫЙ УНИВЕРСИТЕТ \\
имени М.\,К.~АММОСОВА» \\
Институт математики и информатики \\
Кафедра информационных технологий

\vspace{12mm}
\begin{flushright}
\parbox{80mm}{
УТВЕРЖДАЮ\\
Директор ИМИ\\[2mm]
\ulfield{}{}{/\,В.\,И.~Афанасьева\,/}{}\\
\datefield
\\[20mm]
}
\end{flushright}


РАБОЧАЯ ПРОГРАММА ДИСЦИПЛИНЫ
\\[2mm]
\textbf{Б1.Б.19\ -- Теория автоматов и формальных языков} 
\\[5mm]

для программы бакалавриата\\
по направлению подготовки \\
02.03.02 -- Фундаментальная информатика и информационные технологии
\\[15mm]


\parbox{\textwidth}{
 Автор: Павлов А.В., к.\,ф.-м.\,н., доцент кафедры информационных технологий ИМИ, av.pavlov@s-vfu.ru

}
\bigskip


\begin{tabular}{|p{0.3\textwidth}|p{0.3\textwidth}|p{0.3\textwidth}|}
  \hline
  ОДОБРЕНО &  ОДОБРЕНО  & РЕКОМЕНДОВАНО \\
  Заведующий кафедрой \newline разработчика &
  Заведующий выпускающей кафедрой ИТ&
  Нормоконтроль в составе ОП пройден \\
  \ulfield{}{}{\uline{/\hspace{30mm}/}} &
  \ulfield{}{}{\uline{/\hspace{30mm}/}} &
  \ulfield{}{}{\uline{/\hspace{30mm}/}} \\
  Протокол № \uline{\hspace{13pt}} от\newline \datefield[small] & 
  Протокол № \uline{\hspace{13pt}} от\newline \datefield[small] 
  & 
  Протокол № \uline{\hspace{13pt}} от\newline \datefield[small] \\
  \hline
  \multicolumn{2}{|p{0.625\textwidth}|}{Рекомендовано к утверждению в составе ОП\newline 
  02.03.02 «Фундаментальная информатика и информационные технологии»\newline
  Председатель УМК ИМИ \uline{\hspace{21mm}} \mbox{/И.\,В.\, Николаева/}\newline 
  Протокол УМК № \uline{\hspace{12mm}} от \datefield[small]}& 
  Эксперт УМК ИМИ\newline
  \ulfield{}{}{\uline{/\hspace{30mm}/}}\newline\datefield[small]
  \\
  \hline
\end{tabular}
\par\vfill\vspace{6mm}
Якутск -- 2016

\end{center}


\newpage


\begin{center}
\section{АННОТАЦИЯ}
  {\bf к рабочей программе дисциплины\\
  Б1.Б.19\ -- Теория автоматов и формальных языков} \\
  Трудоемкость \uline{~3~} з.~е.
\end{center}


\subsection{Цель освоения и краткое содержание дисциплины}
  
  Изучение дисциплины <<Теория автоматов и формальных языков>> имеет следующие цели:
  \begin{itemize}
    \item дать введение в идеи и методы теории формальных языков; 
    \item ознакомить с основными способами задания и анализа регулярных языков; 
    \item ознакомить с основными способами задания и анализа контекстно-свободных языков. 
  \end{itemize}
  
  
  \textit{Краткое содержание дисциплины.} Регулярные языки. Иерархия Хомского. Контекстно-свободные языки. Языки, распознаваемые машиной Тьюринга. Неразрешимые языки.
  
  



\subsection{Перечень планируемых результатов обучения по дисциплине, соотнесенных с~планируемыми результатами освоения образовательной программы}

\begin{longtable}{|p{54mm}|p{100mm}|}
  \caption{Перечень планируемых результатов обучения}\\
  \hline
  \centering
  Планируемые результаты освоения программы (содержание и коды компетенций) & 
  \centering\arraybackslash
  Планируемые результаты обучения по~дисциплине
  \\
  \hline
  
  ОПК-2 : способность применять в профессиональной деятельности современные языки программирования и языки баз данных, методологии системной инженерии, системы автоматизации проектирования, электронные библиотеки и коллекции, сетевые технологии, библиотеки и пакеты программ, современные профессиональные стандарты информационных технологий, \par 
  
  ПК-2 : способность понимать, совершенствовать и применять современный математический аппарат, фундаментальные концепции и системные методологии, международные и профессиональные стандарты в области информационных технологий, \par 
  & 
  В результате изучения дисциплины обучающийся должен:\newline
  \emph{знать:}
  \begin{itemize}[leftmargin=12pt]
    \item определение, основные способы задания и свойства регулярных языков; 
    \item определение, основные способы задания и свойства контекстно-свободных языков; 
    \item алгоритмы, используемые для определения принадлежности заданной строки заданному регулярному или КС-языку. 
  \end{itemize}
  

  \emph{уметь:}
  \begin{itemize}[leftmargin=12pt]
    \item строить регулярные выражения для несложных регулярных языков; 
    \item понимать и проверять индуктивные доказательства свойств языков, автоматов и грамматик; 
    \item преобразовывать задания данного регулярного языка при помощи конечного автомата, грамматики, регулярного выражения друг в друга; 
    \item пользоваться в компьютерных программах несложными регулярными выражениями для поиска текста; 
    \item строить несложные машины Тьюринга. 
  \end{itemize}
  
  
  \\
  \hline
  ПК-6 : способность эффективно применять базовые математические знания и информационные технологии при решении проектно-технических и прикладных задач, связанных с развитием и использованием информационных технологий
  &
  \emph{владеть навыками:}
  \begin{itemize}[leftmargin=12pt]
    \item проверки принадлежности заданной строки языку данного конечного автомата или регулярного выражения; 
    \item чтения грамматик, заданных в форме Бэкуса-Наура и построения примеров строк, выводимых в данной грамматике. 
  \end{itemize}
  \\
  \hline
  \end{longtable}


\subsection{Место дисциплины в~структуре образовательной программы}

  \begin{table}[H]
  \setlength\arraycolsep{3pt}
  \caption{Содержательно-логические связи дисциплины}
  \begin{tabular}{|l|p{18ex}|*{2}{p{23ex}|}}
  \hline
  \multicolumn{1}{|c|}{\multirow{2}{13ex}{\centering Индекс \linebreak дисциплины}} &
  \multicolumn{1}{c|}{\multirow{2}{18ex}{\centering Наименование \linebreak дисциплины}} & 
  \multicolumn{2}{p{46ex}|}{\centering Коды учебных дисциплин, практик} \\
  \cline{3-4}
   & & 
  \centering на которые опирается содержание дисциплины & 
  \centering\arraybackslash для которых содержание дисциплины выступает опорой
  \\ \hline
  Б1.Б.19 & Теория автоматов и формальных языков 
  & 
  \raggedright
  
  Б1.Б.17 -- Дискретная математика, Б1.Б.18~-- Математическая логика и теория алгоритмов 
  & 
  \raggedright\arraybackslash
  
  Б1.В.ОД.11 -- Языки программирования и методы трансляции 
  \\ \hline
  \end{tabular}
  \end{table}


\subsection{Язык преподавания} 
  Русский.
  



\newpage

\section{Объем дисциплины в зачетных единицах с указанием количества академических часов, выделенных на контактную работу обучающихся с преподавателем (по~видам учебных занятий) и~на~самостоятельную работу обучающихся}

\begin{table}[H]
\caption{Выписка из учебного плана} 
\begin{tabular}{|p{9cm}|c|c|}
\hline
Код и название дисциплины по учебному плану & \multicolumn{2}{p{6cm}|}{Б1.Б.19\ -- Теория автоматов и формальных языков }\\
\hline
Курс изучения &\multicolumn{2}{c|}{ 2 }\\
\hline
Семестр(ы) изучения &\multicolumn{2}{c|}{ 4 }\\
\hline
Форма промежуточной аттестации (зачет/экзамен) &\multicolumn{2}{c|}{ зачет }\\
\hline
Курсовой проект / курсовая работа (указать вид работы при наличии в учебном плане), семестр выполнения &\multicolumn{2}{c|}{ }\\
\hline
Трудоемкость (в ЗЕТ) &\multicolumn{2}{c|}{ 3 (3) }\\
\hline
{\bf Трудоемкость (в часах)} (сумма строк №1, 2, 3), в~т.~ч.:& \multicolumn{2}{c|}{108}\\
\hline
\textbf{№\,1. Контактная работа обучающихся с преподавателем (КР),} в часах:
& \multicolumn{1}{p{3cm}|}{\centering Объем аудиторной работы, в часаx}
& \multicolumn{1}{p{3cm}|}{\centering\arraybackslash В~т.\,ч. с~применением ДОТ или ЭО, в~часах}\\
\hline  
Объем работы (в часах) (1.1.+1.2.+1.3.)& 57 & \\
\hline
1.1. Занятия лекционного типа (лекции) & 18 & \\
\hline
1.2. Занятия семинарского типа, всего, в т.ч.: & & \\
\hline
- семинары (практические занятия, коллоквиумы и~т.~п.)  & – & \\
\hline
- лабораторные работы& 34 & \\
\hline
- практикумы & & \\
\hline
1.3. КСР (контроль самостоятельной работы, консультации)& 5 & \\
\hline
{\bf №\,2. Самостоятельная работа обучающихся (СРС) (в часах)}& \multicolumn{2}{c|}{51}\\
\hline
{\bf №\,3. Количество часов на экзамен (при наличии экзамена в учебном плане)}& \multicolumn{2}{c|}{–}\\
\hline
\end{tabular}
\end{table}



\newpage
\section{Содержание дисциплины, структурированное по~темам с~указанием отведенного на~них количества академических часов и~видов учебных занятий}
\subsection{Распределение часов по~темам и~видам учебных занятий}
\begin{longtable}{|>{\raggedright\arraybackslash}p{59mm}|c|c|c|c|c|c|c|c|c|c|c|}
\caption{}
\\
\hline
 & & 
\multicolumn{9}{c|}{Контактная работа, в часах} & 
\\
\cline{3-11} 
\raisebox{18mm}{Тема}&
\rotleft{Всего часов} &
\rotleft{Лекции} &
\rotleft{из них с прим-м  ЭО и ДОТ} &
\rotleft{\parbox{5cm}{\raggedright\arraybackslash Семинары  (практические занятия, коллоквиумы)}} &
\rotleft{из них с прим-м  ЭО и ДОТ} &
\rotleft{Лабораторные работы} &
\rotleft{из них с прим-м  ЭО и ДОТ} &
\rotleft{Практикумы} &
\rotleft{из них с прим-м  ЭО и ДОТ} &
\rotleft{КСР (консультации)} & 
\rotleft{Часы СРС}
\\
\hline
Тема 1. Регулярные языки & 51 & 6 & 0 & 0 & 0 & 11 & 12 & 0 & 0 & 2 & 20 \\ 
\hline
Тема 2. КС-языки         & 51 & 6 & 0 & 0 & 0 & 13 & 11 & 0 & 0 & 2 & 19 \\ 
\hline
Тема 3. Элементы теории алгоритмов	    & 40 & 6 & 0 & 0 & 0 & 10 & 11 & 0 & 0 & 1 & 12 \\ 
\hline
ВСЕГО ЧАСОВ & 108 & 18 & 0 & 0 & 0 & 34 & 34 & 0 & 0 & 5 & 51 \\ 

\hline
\end{longtable}

\subsection{Содержание тем программы дисциплины} 


\textbf{Тема 1. Регулярные языки}\\
Регулярные языки. Мотивировки: задачи и приложения теории формальных языков. Использование регулярных выражений в задачах компьютерной обработки текстов. POSIX BRE, ERE. Детерминированные и недетерминированные конечные автоматы (КА). Теорема о детерминизации. Регулярные выражения.Регулярные выражения (РВ). Эквивалентность РВ и КА. Праволинейные грамматики. Грамматики. Иерархия Хомского. Праволинейные грамматики. Нормальный вид праволинейных грамматик. Эквивалентность праволинейных грамматик и КА. Свойства регулярных языков. Свойства замкнутости. Лемма о разрастании. Примеры нерегулярных языков. Программные распознаватели РВ.

\textbf{Тема 2. КС-языки        }\\
Примеры КС-языков. Форма Бэкуса-Наура. Практическое использование грамматик. Нормальная форма Хомского. Магазинные автоматы (МА). Эквивалентность МА и КС-грамматик. Свойства КС-языков. Алгоритм Кока-Янгера-Касами. Лемма о разрастании для КС-языков. Примеры не-КС языков. Иерархия Хомского.

\textbf{Тема 3. Элементы теории алгоритмов	   }\\
Машина Тьюринга (МТ) как распознаватель. Детерминированная и недетерминированная МТ. Сложность, классы P и NP. Разрешимые и вычислимые множества. Программы, печатающие сами себя. Универсальная машина, диагонализация и алгоритмическая неразрешимость. Неразрешимость задач самоприменимости и остановки. Понятие о сетях Петри и клеточных автоматах (*).
 

\subsection{Формы и методы проведения занятий, применяемые учебные технологии}
При проведении занятий и организации СРС используются традиционные технологии сообщающего обучения, предполагающие передачу информации в~готовом виде: проведение лекционных занятий, самостоятельная работа с~источниками. Предусмотрено использование активных и интерактивных форм обучения с целью формирования и развития профессиональных навыков студентов - выполнение практических работ с применением компьютерных технологий. 



\section{Перечень учебно-методического обеспечения для самостоятельной работы обучающихся по дисциплине}
\begin{longtable}{|l|>{\raggedright\arraybackslash}p{40mm}|>{\raggedright\arraybackslash}p{54mm}|c|>{\raggedright\arraybackslash}p{30mm}|}
\hline
№ & \centering Наименование раздела (темы) дисциплины & 
\centering Вид СРС & \multicolumn{1}{p{14mm}|}{\centering Трудо\-емкость (в часах)} & \centering\arraybackslash Формы и методы контроля\\
\hline
1 & Регулярные языки	               & Решение задач  & 20 & Сдача домашних заданий в Gradiance, JFLAP \\ 
\hline
2 & КС-языки						   & Решение задач  & 19 & Сдача домашних заданий в Gradiance, JFLAP \\ 
\hline
3 & Элементы теории алгоритмов  & Решение задач  & 12 & Сдача домашних заданий в Gradiance \\ 
\hline
 & ИТОГО                           &                & 51 &  \\ 

\hline
\end{longtable}


\section{Методические указания для обучающихся по освоению дисциплины}
Важное значение в освоении
дисциплины имеет самостоятельная работа. Она предполагает
в том числе выполнение в срок домашних работ в системе
онлайн-тестирования Grаdiance. Хотя тесты Gradiance
выглядят как тесты с выбором варианта, на деле требуется
решить традиционную задачу, а Gradiance задает вопросы
к различным случайно выбранным аспектам решения. Для
с3ачи каждого домашнего5задания предусмотрен срок,
после которого решение получает сначала неполные баллы,
а затем не получает баллов вообще. Своевременное выполнение
заданий требует самостоятельности и ответственности. При
возникновении трудностей следует задавать вопросы, в том числе
на форуме курса, где на вопросы могут отвечать сокурсники.
\par
Последовательное и добросовестное изучение курса является основой для
выработки углубленного понимания алгоритмических проблем генерации и анализа
структурированных текстов в областях деятельности, предполагаемых образовательным
стандартом.


\newpage
\subsubsection*{Рейтинговый регламент по дисциплине}
\begin{longtable}{|>{\raggedright\arraybackslash}p{110mm}|r|r|}
\hline
\centering\arraybackslash Вид выполняемой учебной работы (контролирующие мероприятия) & 
\multicolumn{1}{p{20mm}|}{\centering\arraybackslash Количество баллов (min)} & 
\multicolumn{1}{p{20mm}|}{\centering\arraybackslash Количество баллов (max)} \\
\hline
Посещаемость                    & 6 & 10 \\ 
\hline
Домашние задания, онлайн-тесты  & 30 & 50 \\ 
\hline
Контрольные работы              & 24 & 40 \\ 
\hline
Количество баллов для получения зачета (min--max) & 60 & 100 \\ 

\hline
\end{longtable}

\section{Фонд оценочных средств для проведения промежуточной аттестации обучающихся по дисциплине}

\subsection{Показатели, критерии и шкала оценивания}

\begin{longtable}{|@{\hspace{.7mm}}p{15mm}@{}|p{50mm}|@{}p{17mm}@{}|p{53mm}|p{14mm}|}
\hline
  \centering\small Коды оцениваемых компетенций
& \centering Показатель оценивания (дескриптор) (по п.1.2) 
& \centering\small Уровни освоения 
& \centering Критерий оценивания 
& \centering\small\arraybackslash Оценка
\\
\hline
\multirow{4}{15mm}{ОПК-2, ПК-2, ПК-6}
& \small
\emph{знать:}\newline
определение, основные способы задания и свойства регулярных языков; определение, основные способы задания и свойства контекстно-свободных языков; алгоритмы, используемые для определения принадлежности заданной строки заданному регулярному или КС-языку. \newline
\emph{уметь:}\newline
строить регулярные выражения для несложных регулярных языков; понимать и проверять индуктивные доказательства свойств языков, автоматов и грамматик; преобразовывать задания данного регулярного языка при помощи конечного автомата, грамматики, регулярного выражения друг в друга; пользоваться в компьютерных программах несложными регулярными выражениями для поиска текста; строить несложные машины Тьюринга. \newline
\emph{владеть навыками:}\newline
проверки принадлежности заданной строки языку данного конечного автомата 
& 
высокий & \raggedright\arraybackslash\small
способен выполнять все задачи из следующего списка:
строить конечные автоматы для языков с простыми закономерностями повторов;
преобразовать заданный недетерминированный КА в детерминированный;
написать extended regexp для структурированных фрагментов текста, включающих вложенные повторы, с использованием классов символов, квантификаторов и группировки;
преобразовывать РВ в эквивалентное КА и обратно;
приводить КС-грамматику к нормальной форме Хомского;
преобразовывать КС грамматику в эквивалентный МА;
строить машину Тьюринга, выполняющую простые манипуляции со строками на ленте, либо
арифметические действия (исключая деление) с аргументами. & зачтено 
\\
\cline{3-5} 
&
\multirow{3}{50mm}{\parbox{50mm}{%
или регулярного выражения; чтения грамматик, заданных в форме Бэкуса-Наура и построения примеров строк, выводимых в данной грамматике. 
}}
& базовый & не способен выполнить не более одного пункта из вышеперечисленного & зачтено 
\\
\cline{3-5}

& & мини\-мальный & не способен выполнить не более двух пункта из вышеперечисленного & зачтено 
\\

\cline{3-5}
& & не освоено & не способен выполнить три или более пунктов из вышеперечисленного & не зачтено 
\\

\hline

\end{longtable}



\subsection{Типовые контрольные задания (вопросы) для промежуточной аттестации}

\begin{longtable}{|p{15mm}|p{42mm}|p{17mm}|p{70mm}|}
\hline
\centering\small Коды оцениваемых компетенций  & \centering Оцениваемый показатель (ЗУВ) 
& \centering Тема  & \centering\arraybackslash Образец типового (тестового или практического) задания (вопроса)
\\
\hline
ПК-2, ПК-6 & 
знать определение, основные способы задания и свойства регулярных языков;
владеть навыками проверки принадлежности заданной строки языку данного конечного автомата или регулярного выражения; & 
1 & 
Нарисуйте диаграмму недетерминированного конечного автомата над алфавитом $\{a,b,c\}$
с множеством состояний $\{P, Q, R, S, T\}$, начальным состоянием $P$,
множеством финальных состояний $\{T\}$, и~функцией переходов $\delta$:\newline
{\small
$\delta(P,ε)=\{Q\}$,  $\delta(Q,ε)=\{R\}$,  $\delta(T,0)=\{T\}$,\newline
$\delta(R,0)=\{R,S\}$, $\delta(S,0)=\{S,T\}$\newline
$\delta(P,0)=\{P\}$,  $\delta(Q,2)=\{Q\}$, \newline
$\delta(T,1)=\{T\}$, $\delta(R,1)=\{R\}$,   $\delta(S,1)=\{S\}$.\newline
}
Перечислите все строки длины $3$, допускаемые данным автоматом.
Допускает ли этот автомат цепочку $002011$? Перечислите все состояния, в которых он может оказаться, прочитав данную цепочку. 
\\
\hline
ПК-2, ПК-6 & 
знать определение, основные способы задания и свойства контекстно-свободных языков; & 
2 & 
Постройте КС-грамматику для языка $\{0^m 1^n 2^k 0^{n+2} \mid m,k \geqslant 0,\: n>0\}$. 
\\
\hline
ПК-2, ПК-6 & 
знать алгоритмы, используемые для определения принадлежности заданной строки заданному регулярному или КС-языку. & 
2 & 
Заполните таблицу алгоритма Кока-Янгера-Касами для грамматики \newline
$S \to AB \mid BC;\quad A \to BA \mid a$,\newline
$B \to CC \mid b;\quad C \to AB \mid a$\newline
и строки $w = baaba$. 
\\
\hline
ОПК-2, ПК-6 & 
уметь строить регулярные выражения для несложных регулярных языков; & 
1 & 
Постройте ERE для последовательностей адресных строк в следующем формате.
Адресная строка состоит из названия улицы (одно русское слово, начинающееся с прописной буквы,
перед которым обязательно идет «{\tt ул.}»), номера дома $n$
(\mbox{$1 \leqslant n \leqslant 59$}), возможно, с дробью (число от 1 до 3), и номера
квартиры $m$, ($1 \leqslant m \leqslant 79$), перед которым обязательно идет
«{\tt кв.}». Улица, номер дома и номер квартиры разделяются запятой с пробелом. Адресные
строки в последовательности разделяются точкой с запятой и пробелом. Последовательность
завершается точкой. Пример:
«{\tt ул. Гороховая, 53, кв. 23; ул. Кржижановского, 27/2, кв. 2.}» 
\\
\hline
ПК-2, ПК-6 & 
уметь понимать и проверять индуктивные доказательства свойств языков, автоматов и грамматик; & 
1,2 & 
Дано индуктивное доказательство о языке данного автомата. Объясните, получается ли
каждый из отмеченных логических шагов доказательства в силу
а) свойств строк, б) свойств конечных автоматов или в) индуктивного предположения. 
\\
\hline
ПК-2, ПК -6 & 
уметь преобразовывать задания данного регулярного языка при помощи конечного автомата, грамматики, регулярного выражения друг в друга; & 
1,2 & 
Постройте конечный автомат, эквивалентный регулярному выражению $((01)^*5(01^*2 + 021^+)(303+44)^*)^*$. 
\\
\hline
ПК-2, ПК -6 & 
уметь строить несложные машины Тьюринга. & 
3 & 
Постройте машину Тьюринга, обращающую второе слово на ленте, содержащей два слова в алфавите $\{0,1\}$. 
\\
\hline
\end{longtable}
\subsubsection*{Вопросы к зачету}
\begin{enumerate}
\item
Регулярное выражение (“математический вариант”). Итерация, конкатенация, альтернатива.
\item
Регэкспы (регулярные выражения) POSIX: базовые и расширенные. Конструкции \verb!.!, \verb![abc]!,
\verb![a-c]!, \verb![^abm-z]!, \verb!*!, \verb!+!, \verb!{n, m}!, \verb!^!, \verb!$!, \verb!|!,
скобки, \texttt{\textbackslash}, \verb!?!.
\item
Детерминированные и недетерминированные конечные автоматы, определяемые ими языки.
\item
Теорема о детерминизации: преобразование НКА в ДКА
\item
Теорема об операциях над регулярными (автоматными) языками: конкатенация, итерация, объединение, пересечение, разность, дополнение, обращение (реверс).
\item
Теорема о существовании эквивалентного регулярного выражения для любого ДКА
\item
Теорема о преобразовании регулярного выражения в конечный автомат.
\item
Порождающие грамматики.
\item
Лемма о разрастании для регулярных языков.
\item
Контекстно-свободные грамматики. Примеры КС-грамматик: грамматика для $\{a^nb^n\}$. Грамматика для арифметических выражений, построенных из чисел.
\item
Деревья разбора. Левое и правое порождение (вывод).
\item
Нормальная форма Хомского. Устранение бесполезных символов.
\item
Нормальная форма Хомского. Удаление ε-правил.
\item
Нормальная форма Хомского. Удаление цепных правил.
\item
Нормальная форма Хомского. Устранение правил с терминалами в теле длины >1, устранение правил с телом длины >2 из нетерминалов.
\item
Алгоритм Кока-Янгера-Касами, его сложность (Галочкин и др., лекция 5)
\item
Магазинный автомат. Язык, допускаемый автоматом: при помощи пустого стека, при помощи конечного состояния. Эквивалентность.
\item
Построение магазинного автомата, эквивалентного заданной грамматике: конструкция, пример.
\item
Построение грамматики, эквивалентной данному магазинному автомату: построение, пример.
\item
Машина Тьюринга. Машины прибавления единицы, перестановки двух слова на ленте.
\item
Универсальная машина Тьюринга. Неразрешимость проблемы самоприменимости.
\item
Неразрешимость проблемы остановки.
\end{enumerate}



\subsection{Методические материалы, определяющие процедуры оценивания}

Форма промежуточной аттестации: зачет.
\par
Данный вид комплексного испытания предполагает последовательное выполнение
всех форм текущего контроля, таких, как онлайн-тесты по домашним заданиям на Gradiance
и~выполнение контрольных работ.
\par
Онлайн-тестирование. Данная форма контроля направлена на оценку основных
теоретических знаний обучающегося по мере освоения основных разделов дисциплины.
\par
Контрольные работы. В этой форме промежуточного контроля проверяется закрепленность
необходимых умений и навыков.



\newpage
\section{Перечень основной и дополнительной учебной литературы, необходимой для освоения дисциплины}

  \begin{longtable}{|l|p{7cm}|p{18mm}|c|p{32mm}|}
  \caption*{Перечень литературы}\\
  \hline
  № & 
  \centering\small\arraybackslash Автор, название, место издания, издательство, год издания учебной литературы, вид и характеристика иных информационных ресурсов &
  \multicolumn{1}{p{18mm}|}{\centering\small\arraybackslash Наличие грифа, вид грифа} &
  \multicolumn{1}{p{21mm}|}{\centering\small\arraybackslash НБ СВФУ, кафедральная библиотека и кол-во экземпляров} & 
  \centering\small\arraybackslash Электронные издания: точка доступа к ресурсу (наименование ЭБС, ЭБ СВФУ)\\
  \hline
  \multicolumn{5}{|c|}{Основная литература}\\
  \hline
  1 &\raggedright\arraybackslash Мозговой М. В. Классика программирования: алгоритмы, языки, автоматы, компиляторы: практический подход. СПб: Наука и Техника, 2006. 320 с. & & 10 &
  \\
  \hline
  
  \multicolumn{5}{|c|}{Дополнительная литература}\\
  \hline
  1 &\raggedright\arraybackslash Кормен Т.\,Х. Алгоритмы: построение и анализ. М.: МЦНМО, 1999 & & 1 & 
  \\
  \hline
  
  \end{longtable}
  
\section{Перечень ресурсов информационно-телекоммуникационной сети «Интернет» (далее сеть-Интернет), необходимых для освоения дисциплины}
\begin{enumerate}
  
  \item Хопкрофт Дж., Мотвани Р., Ульман Дж. Введение в теорию автоматов, языков и вычислений. Режим доступа:\\ https://books.google.ru/books?id=Th5ZTEpJQMoC 
  
  \item Пентус А.\,Е., Пентус М.\,Р. Теория формальных языков. М.: Изд-во ЦПИ при ММФ МГУ, 2004, 80 с. Режим доступа:\\ http://www.mccme.ru/free-books/pentus/pentus.pdf 
  
  \item Верещагин Н.\,К., Шень А.,Х. Вычислимые функции. М.: Изд-во МЦНМО, 2012. 160 с. Режим доступа: \\ http://www.mccme.ru/free-books/shen/shen-logic-part3-2.pdf 
  
  \item Система Gradiance. Режим доступа:\\ http://www.newgradiance.com/ 
  
  \item INTUIT: Пентус А. Е., Пентус М. Р. Курс <<Математическая теория формальных языков>>.  Режим доступа: http://www.intuit.ru/studies/courses/1064/170/info 
  
  \item Susan H. Rodger. JFLAP version 7.0. Режим доступа: http://www.cs.duke.edu/csed/jflap/ 
  
\end{enumerate}


\newpage
\section{Описание материально-технической базы, необходимой для осуществления образовательного процесса по дисциплине}
  
  
       Для проведения лекционных занятий требуется аудитория, оборудованная доской,  мультимедийным проектором с экраном. 
       Для проведения лабораторных занятий требуется компьютерный класс с подключением к интернету.
  
  


\section{Перечень информационных технологий, используемых при осуществлении образовательного процесса по дисциплине, включая перечень программного обеспечения
}

\subsection{Перечень информационных технологий, используемых при осуществлении образовательного процесса по дисциплине}

При осуществлении образовательного процесса по дисциплине используются следующие информационные технологии:
\begin{itemize}[nolistsep]
  
\item использование на занятиях электронных изданий (чтение лекций с использованием слайд-презентаций);
  
\item ведение учета посещаемости и выполнения учебных заданий в системе Google Docs;
  
\item организация взаимодействия с обучающимися посредством электронной почты, специализированного образовательного форума Piazza;
  
\item компьютерное тестирование.
  
\end{itemize}

\subsection{Перечень программного обеспечения}
При осуществлении образовательного процесса по дисциплине используются следующее программное обеспечение:
\begin{itemize}[nolistsep]
  
\item Виртуальная машина Java, например Oracle Java Runtime Environment;
  
\item свободно распространяемое программное обеспечение: JFLAP и ANTLR;
  
\item интернет-браузер.
  
\end{itemize}



\newpage
\begin{center}
\section*{ЛИСТ АКТУАЛИЗАЦИИ РАБОЧЕЙ ПРОГРАММЫ ДИСЦИПЛИНЫ}
Б1.Б.19\ --- Теория автоматов и формальных языков 
\end{center}

  \noindent
  \begin{tabular}{|p{15mm}|p{67mm}|p{25mm}|p{41mm}|}
    \hline
    \small\centering
    Учебный год 
    & \small\centering
    Внесенные изменения 
    & \small\centering
    Преподаватель (ФИО) 
    & \small\centering\arraybackslash
    Протокол заседания выпускающей кафедры (дата, номер), ФИО зав.кафедрой, подпись \\
    & & & \\\hline
    & & & \\\hline
    & & & \\\hline
    & & & \\\hline
    & & & \\\hline
    & & & \\\hline
    & & & \\\hline
    & & & \\\hline
    & & & \\\hline
    & & & \\\hline
    & & & \\\hline
    & & & \\\hline
    & & & \\\hline
    & & & \\\hline
    & & & \\\hline
    & & & \\\hline
    & & & \\\hline
    & & & \\\hline
    & & & \\\hline
    & & & \\\hline
    & & & \\\hline
    & & & \\\hline
    & & & \\\hline
    & & & \\\hline
    & & & \\\hline
    & & & \\\hline
    & & & \\\hline
    & & & \\\hline
    & & & \\\hline
    & & & \\\hline
    & & & \\\hline
    & & & \\\hline
    & & & \\\hline
    & & & \\\hline
    & & & \\\hline
    & & & \\\hline
  \end{tabular}

  \medskip\noindent\textit{В таблице указывается только характер изменений (например, изменение темы, списка источников по~теме или темам, средств промежуточного контроля) с~указанием пунктов рабочей программы. Само содержание изменений оформляется приложением по~сквозной нумерации.}

\newpage\tableofcontents

\end{document}