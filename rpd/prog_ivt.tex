\documentclass[a4paper,12pt]{article}
  \usepackage[hmargin={30mm, 17mm}, vmargin={20mm,20mm}, nohead]{geometry}
  \usepackage{polyglossia, enumitem, array, longtable, multirow, tabularx, rotating, indentfirst, caption, float, ulem, titlesec }%}
  \setdefaultlanguage{russian} % выбор основного языка (для переносов)
  % шрифты 
  \setmainfont{Times New Roman} 
  \setmonofont[Scale=0.9]{Consolas}
  \setsansfont{Trebuchet MS}
  \defaultfontfeatures{Ligatures=TeX} % нужен для того, чтобы работали стандартные сочетания символов ---, -- << >> и т.п.

  
  \usepackage[vargreek-shape=unicode]{unicode-math}
  \setmathfont{latinmodern-math.otf}
  \setmathfont[range=\mathit/{latin,Latin}]{Times New Roman-Italic}
  \setmathfont[range=\mathit/{greek,Greek}]{Times New Roman}
  \setmathfont[range=\mathup]{Times New Roman}
  

  % pdf metadata
  \usepackage[
    pdfencoding=unicode,
    pdftitle={Рабочая программа дисциплины Программирование},
    pdfauthor={составлена }, 
    bookmarksnumbered=true, 
      colorlinks=true, linktocpage=true, linkcolor=blue, pdfpagemode=UseOutlines]
  {hyperref}

  % формат заголовков, подписей и списков
  \titleformat*{\section}{\large\bfseries\centering}
  \titleformat*{\subsection}{\bfseries}
  \captionsetup[table]{singlelinecheck=off, font=small, labelsep=period, textfont=it, format=hang, justification=raggedleft}
  \setlist[enumerate,1]{nolistsep,labelindent=0pt,leftmargin=\parindent}
  \setlist[itemize,1]{nolistsep,labelindent=20pt,leftmargin=*,label=--}

  \makeatletter

  % заполняет ширину текста полем с подчеркиванием.
  % #1 - текст слева, #2 - на подчеркивании, #3 - справа
  \newcommand{\ulfield}[3]{
  \noindent
  \begin{tabularx}{\linewidth}{@{}l@{}X@{}l@{}}
  #1\if\relax\detokenize{#1}\relax\else\,~\vrule width 0pt\fi 
  & \uline{\vrule width 0pt\hfill#2\hfill\vrule width 0pt} & 
  \if\relax\detokenize{#3}\relax\else\vrule width 0pt~\,\fi #3
  \end{tabularx}
  }

  %%% Перенос составных слов
    \XeTeXinterchartokenstate=1
    \XeTeXcharclass `\- 24
    \XeTeXinterchartoks 24 0 = {\hskip\z@skip}
    \XeTeXinterchartoks 0 24 = {\nobreak}

  \newcommand\rotleft{\rotatebox{90}}
  
  \makeatother

  \newcommand{\datefield}[1][]{\if
  \relax\detokenize{#1}\relax
  «\uline{\hspace{22pt}}»~\uline{\hspace{90pt}}\,~20\uline{\hspace{20pt}}~г.\else 
  «\uline{\hspace{18pt}}»~\uline{\hspace{60pt}}\,~20\uline{\hspace{18pt}}~г.\fi
  }


%%% Настройка содержания
\AtBeginDocument{
  \makeatletter 
  \def\@tocline#1#2#3#4#5#6#7{\relax
  \ifnum #1>\c@tocdepth % then omit
  \else
    \par \addpenalty\@secpenalty\addvspace{#2}%
    \begingroup \hyphenpenalty\@M
    \@ifempty{#4}{\@tempdima\csname r@tocindent\number#1\endcsname\relax}{\@tempdima#4\relax}%
    \parindent\z@ \leftskip#3\relax \advance\leftskip\@tempdima\relax
    \rightskip\@pnumwidth plus4em \parfillskip-\@pnumwidth
    #5\leavevmode\hskip-\@tempdima
      \ifcase #1
       \or\or \hskip 1em \or \hskip 2em \else \hskip 3em \fi%
      #6\nobreak\relax
    \dotfill\hbox to\@pnumwidth{\@tocpagenum{#7}}\par
    \nobreak
    \endgroup
  \fi}
  \makeatother
}%AtBeginDocument

\begin{document}
\sloppy
\thispagestyle{empty}

\noindent
\begin{center}
Министерство образования и науки Российской Федерации \\
Федеральное государственное автономное образовательное \\
учреждение высшего образования\\
«СЕВЕРО-ВОСТОЧНЫЙ ФЕДЕРАЛЬНЫЙ УНИВЕРСИТЕТ \\
имени М.\,К.~АММОСОВА» \\
Институт математики и информатики \\
Кафедра информационных технологий

\vspace{12mm}
\begin{flushright}
\parbox{80mm}{
УТВЕРЖДАЮ\\
Директор ИМИ\\[2mm]
\ulfield{}{}{/\,В.\,И.~Афанасьева\,/}{}\\[1mm]
\datefield
\\[20mm]
}
\end{flushright}


РАБОЧАЯ ПРОГРАММА ДИСЦИПЛИНЫ
\\[2mm]
\textbf{Б1.В.ОД.3\ -- Программирование} 
\\[5mm]

для программы бакалавриата\\
по направлению подготовки \\
09.03.01 -- Информатика и вычислительная техника
\\[15mm]


\parbox{\textwidth}{
 Автор: Павлов А.В., к.\,ф.-м.\,н., доцент кафедры информационных технологий ИМИ, av.pavlov@s-vfu.ru

}
\bigskip


\begin{tabular}{|p{0.3\textwidth}|p{0.3\textwidth}|p{0.3\textwidth}|}
  \hline
  ОДОБРЕНО &  ОДОБРЕНО  & РЕКОМЕНДОВАНО \\
  Заведующий кафедрой \newline разработчика &
  Заведующий выпускающей кафедрой ИТ&
  Нормоконтроль в составе ОП пройден \\
  \ulfield{}{}{\uline{/\hspace{30mm}/}} &
  \ulfield{}{}{\uline{/\hspace{30mm}/}} &
  \ulfield{}{}{\uline{/\hspace{30mm}/}} \\
  Протокол № \uline{\hspace{13pt}} от\newline \datefield[small] & 
  Протокол № \uline{\hspace{13pt}} от\newline \datefield[small] 
  & 
  Протокол № \uline{\hspace{13pt}} от\newline \datefield[small] \\
  \hline
  \multicolumn{2}{|p{0.625\textwidth}|}{Рекомендовано к утверждению в составе ОП\newline 
  09.03.01 «Информатика и и вычислительная техника»\newline
  Председатель УМК ИМИ \uline{\hspace{21mm}} \mbox{/И.\,В.\, Николаева/}\newline 
  Протокол УМК № \uline{\hspace{12mm}} от \datefield[small]}& 
  Эксперт УМК ИМИ\newline
  \ulfield{}{}{\uline{/\hspace{30mm}/}}\newline\datefield[small]
  \\
  \hline
\end{tabular}
\par\vfill\vspace{6mm}
Якутск -- 2016

\end{center}


\newpage


\begin{center}
\section{АННОТАЦИЯ}
  {\bf к рабочей программе дисциплины\\
  Б1.В.ОД.3\ -- Программирование} \\
  Трудоемкость \uline{~10~} з.~е.
\end{center}


\subsection{Цель освоения и краткое содержание дисциплины}
  
  Целью изучения диcциплины <<Программирование>> является: овладение основами программирования на языках Питон (Python) и Си/Си++.
  
  
  \textit{Краткое содержание дисциплины.} Основы синтаксиса и семантики императивных языков высокого уровня Python и С++. Переменные, типы, выражения и присваивание. Условные и циклические структуры. Текстовый ввод/вывод. Функции и передача параметров. Структурированная декомпозиция.
  
  



\subsection{Перечень планируемых результатов обучения по дисциплине, соотнесенных с~планируемыми результатами освоения образовательной программы}

\begin{longtable}{|p{54mm}|p{100mm}|}
  \caption{Перечень планируемых результатов обучения}\\
  \hline
  \centering
  Планируемые результаты освоения программы (содержание и коды компетенций) & 
  \centering\arraybackslash
  Планируемые результаты обучения по~дисциплине
  \\
  \hline
  \endhead
  
  ОПК-2 : способностью осваивать методики использования программных средств для решения практических задач, \par 
  
  ПК-1 : способностью разрабатывать модели компонентов информационных систем, включая модели баз данных и модели и интерфейсов «человек – электронно-вычислительная машина»
  & 
  В результате изучения дисциплины обучающийся должен:\newline
  \emph{знать:}
  \begin{itemize}[leftmargin=12pt]
    \item способы передачи параметров в функцию в языке Си++; 
    \item несколько методов сортировки, включая быструю сортировку; 
  \end{itemize}
  

  \emph{уметь:}
  \begin{itemize}[leftmargin=12pt]
    \item анализировать, объяснять поведение, модифицировать, тестировать небольшие программы на языках Питон и Си++, использующие любую комбинацию следующих понятий: простые вычисления, ветвление, итерация, простой ввод-вывод, в том числе в/из текстовых файлов, массивы (списки), функции; 
    \item сравнивать два алгоритма для решения несложной задачи; 
  \end{itemize}
  

  \emph{владеть навыками:}
  \begin{itemize}[leftmargin=12pt]
    \item отладки программ на языке Си++ в современной инструментальной среде; 
    \item написания небольших программ на языках Питон и Си++ с использованием условного оператора и различных видов циклов; 
  \end{itemize}
  
  \\
  \hline
  \end{longtable}


\subsection{Место дисциплины в~структуре образовательной программы}

  \begin{table}[H]
  \setlength\arraycolsep{3pt}
  \caption{Содержательно-логические связи дисциплины}
  \begin{tabular}{|l|p{18ex}|*{2}{p{23ex}|}}
  \hline
  \multicolumn{1}{|c|}{\multirow{2}{13ex}{\centering Индекс \linebreak дисциплины}} &
  \multicolumn{1}{c|}{\multirow{2}{18ex}{\centering Наименование \linebreak дисциплины}} & 
  \multicolumn{2}{p{46ex}|}{\centering Коды учебных дисциплин, практик} \\
  \cline{3-4}
   & & 
  \centering на которые опирается содержание дисциплины & 
  \centering\arraybackslash для которых содержание дисциплины выступает опорой
  \\ \hline
  Б1.В.ОД.3 & Программирование 
  & 
  \raggedright
  
  --- 
  & 
  \raggedright\arraybackslash
  
  Б1.Б.23~-- Операционные системы,
  Б1.В.ОД.1~-- Компьютерные сети и телекоммуникации,
  Б1.В.ОД.4~-- Структуры и алгоритмы обработки данных,
  Б1.В.ОД.5~-- Web-программирование,
  Б1.В.ОД.6~-- Программирование .NET,
  Б1.В.ОД.7~-- Объектно-ориентированное программирование,
  Б1.В.ОД.8~-- Языки программирования и методы трансляции 
  \\ \hline
  \end{tabular}
  \end{table}


\subsection{Язык преподавания} 
  Русский.
  



\newpage

\section{Объем дисциплины в зачетных единицах с указанием количества академических часов, выделенных на контактную работу обучающихся с преподавателем (по~видам учебных занятий) и~на~самостоятельную работу обучающихся}

\begin{table}[H]
\caption{Выписка из учебного плана} 
\begin{tabular}{|p{9cm}|c|c|}
\hline
Код и название дисциплины по учебному плану & \multicolumn{2}{p{6cm}|}{Б1.В.ОД.3\ -- Программирование }\\
\hline
Курс изучения &\multicolumn{2}{c|}{ 1 }\\
\hline
Семестр(ы) изучения &\multicolumn{2}{c|}{ 1, 2 }\\
\hline
Форма промежуточной аттестации (зачет/экзамен) &\multicolumn{2}{c|}{ экзамен / экзамен }\\
\hline
Курсовой проект / курсовая работа (указать вид работы при наличии в учебном плане), семестр выполнения &\multicolumn{2}{c|}{ }\\
\hline
Трудоемкость (в ЗЕТ) &\multicolumn{2}{c|}{ 5 / 5 (10) }\\
\hline
{\bf Трудоемкость (в часах)} (сумма строк №1, 2, 3), в~т.~ч.:& \multicolumn{2}{c|}{180 / 180}\\
\hline
\textbf{№\,1. Контактная работа обучающихся с преподавателем (КР),} в часах:
& \multicolumn{1}{p{3cm}|}{\centering Объем аудиторной работы, в часаx}
& \multicolumn{1}{p{3cm}|}{\centering\arraybackslash В~т.\,ч. с~применением ДОТ или ЭО, в~часах}\\
\hline  
Объем работы (в часах) (1.1.+1.2.+1.3.)& 75 / 79 & \\
\hline
1.1. Занятия лекционного типа (лекции) & 17 / 18 & \\
\hline
1.2. Занятия семинарского типа, всего, в т.ч.: & & \\
\hline
- семинары (практические занятия, коллоквиумы и~т.~п.)  & 17 / 18 & \\
\hline
- лабораторные работы& 34 / 36 & \\
\hline
- практикумы & & \\
\hline
1.3. КСР (контроль самостоятельной работы, консультации)& 7 / 7 & \\
\hline
{\bf №\,2. Самостоятельная работа обучающихся (СРС) (в часах)}& \multicolumn{2}{c|}{69 / 65}\\
\hline
{\bf №\,3. Количество часов на экзамен (при наличии экзамена в учебном плане)}& \multicolumn{2}{c|}{36 / 36}\\
\hline
\end{tabular}
\end{table}



\newpage
\section{Содержание дисциплины, структурированное по~темам с~указанием отведенного на~них количества академических часов и~видов учебных занятий}
\subsection{Распределение часов по~темам и~видам учебных занятий}
\begin{longtable}{|>{\raggedright\arraybackslash}p{59mm}|c|c|c|c|c|c|c|c|c|c|c|}
\caption{}
\\
\hline
 & & 
\multicolumn{9}{c|}{Контактная работа, в часах} & 
\\
\cline{3-11} 
\raisebox{18mm}{Тема}&
\rotleft{Всего часов} &
\rotleft{Лекции} &
\rotleft{из них с прим-м  ЭО и ДОТ} &
\rotleft{\parbox{5cm}{\raggedright\arraybackslash Семинары  (практические занятия, коллоквиумы)}} &
\rotleft{из них с прим-м  ЭО и ДОТ} &
\rotleft{Лабораторные работы} &
\rotleft{из них с прим-м  ЭО и ДОТ} &
\rotleft{Практикумы} &
\rotleft{из них с прим-м  ЭО и ДОТ} &
\rotleft{КСР (консультации)} & 
\rotleft{Часы СРС}
\\
\hline
Тема 1. Язык Питон: простые типы и основы синтаксиса. & 54 & 7 & 0 & 7 & 0 & 14 & 0 & 0 & 0 & 2 & 24 \\ 
\hline
Тема 2. Язык Питон: структурированные типы.           & 56 & 6 & 0 & 6 & 0 & 12 & 0 & 0 & 0 & 2 & 30 \\ 
\hline
Тема 3. Язык Питон: функции и рекурсия.	              & 33 & 4 & 0 & 4 & 0 & 8 & 0 & 0 & 0 & 2 & 15 \\ 
\hline
Тема 4. Язык Си: простые типы и основы синтаксиса.    & 32 & 4 & 0 & 4 & 0 & 8 & 0 & 0 & 0 & 2 & 14 \\ 
\hline
Тема 5. Язык Си/Си++: структурированные типы.         & 30 & 4 & 0 & 4 & 0 & 7 & 0 & 0 & 0 & 2 & 13 \\ 
\hline
Тема 6. Язык Си: функции и рекурсия.	              & 30 & 4 & 0 & 4 & 0 & 7 & 0 & 0 & 0 & 2 & 13 \\ 
\hline
Тема 7. Алгоритмы сортировки.	                      & 27 & 3 & 0 & 3 & 0 & 7 & 0 & 0 & 0 & 1 & 13 \\ 
\hline
Тема 8. Язык Си++: классы. Знакомство с STL.          & 26 & 3 & 0 & 3 & 0 & 7 & 0 & 0 & 0 & 1 & 12 \\ 
\hline
ВСЕГО ЧАСОВ & 288 & 35 & 0 & 35 & 0 & 70 & 0 & 0 & 0 & 14 & 134 \\ 

\hline
\end{longtable}

\subsection{Содержание тем программы дисциплины} 


\textbf{Тема 1. Язык Питон: простые типы и основы синтаксиса.}\\
Что такое программирование? Компилируемые и интерпретируемые языки. Статическая и динамическая типизация. Знакомство с Питоном. Среда IDLE. Окно интерпретатора. REPL-цикл. Написание, редактирование и запуск программ. Числовые типы. Логический тип. Арифметические и логические операторы. Оператор присваивания. Модуль math. Программы линейной структуры. Блок-схемы. Операторы ветвления и цикла, роль отступов. Строковый тип. Перебор строк. Ввод с клавиатуры. Простое преобразование типов. Встроенные средства документации, функция help(). Функции dir().

\textbf{Тема 2. Язык Питон: структурированные типы.          }\\
Кортежи и списки в Питоне. Индексация, нотация диапазонов. Поиск минимального/максимального элемента. Строки. Unicode, Python 2 и Python 3. Словари. Семантика присваивания для сложных типов. Двумерные массивы как списки: индексация, ввод, заполнение случайными значениями. Перебор списков. Перебор словаря, методы keys(), values(), items().

\textbf{Тема 3. Язык Питон: функции и рекурсия.	             }\\
Функции. Видимость переменных. Передача параметров. Значения параметров по умолчанию. Рекурсия. Стандартная библиотека, менеджер пакетов pip.

\textbf{Тема 4. Язык Си: простые типы и основы синтаксиса.   }\\
Язык Си. Функция main(). Числовые типы. Объявления переменных. Компиляция программы на Си, заголовочные и библиотечные файлы. Стандартная библиотека. Логический тип. Арифметические и логические операторы. Оператор присваивания. Неявное преобразование типов. Блоки. Операторы ветвления и цикла, оператор выбора. Вычисление сумм и произведений. Символьный тип, символьные константы, коды символов. Unicode и широкий символьный тип. Ввод-вывод с текстовыми файлами.

\textbf{Тема 5. Язык Си/Си++: структурированные типы.        }\\
Указатели, типизированные и нетипизированные. Разыменование. Средства отладки Visual Studio. Массивы. Строки. Структуры. Перечисления. malloc() и new().

\textbf{Тема 6. Язык Си: функции и рекурсия.	             }\\
Функции. Передача параметров по ссылке и по значению. Декомпозиция задачи <<сверху вниз>>. Видимость переменных. Статические переменные. Глобальные переменные.

\textbf{Тема 7. Алгоритмы сортировки.	                     }\\
Сортировка пузырьком. Сортировка вставками. Сортировка слиянием. Быстрая сортировка.

\textbf{Тема 8. Язык Си++: классы. Знакомство с STL.         }\\
Перегрузка функций и перегрузка операторов. Классы и объекты; конструкторы и деструкторы. Методы. Классы потокового ввода-вывода. Некоторые контейнерные классы STL.
 

\subsection{Формы и методы проведения занятий, применяемые учебные технологии}
При проведении занятий и организации СРС используются традиционные технологии сообщающего обучения, предполагающие передачу информации в~готовом виде: проведение лекционных занятий, самостоятельная работа с~источниками. Предусмотрено использование активных и интерактивных форм обучения с целью формирования и развития профессиональных навыков студентов~--- выполнение лабораторных работ, подразумевающих применение компьютерных технологий. 



\section{Перечень учебно-методического обеспечения для самостоятельной работы обучающихся по дисциплине}
\begin{longtable}{|l|>{\raggedright\arraybackslash}p{40mm}|>{\raggedright\arraybackslash}p{30mm}|c|>{\raggedright\arraybackslash}p{54mm}|}
\hline
№ & \centering Наименование раздела (темы) дисциплины & 
\centering Вид СРС & \multicolumn{1}{p{14mm}|}{\centering Трудо\-емкость (в часах)} & \centering\arraybackslash Формы и методы контроля\endhead
\hline
1 & Язык Питон: простые типы и основы синтаксиса. & Решение задач  & 24 & Сдача домашних заданий лично, неоконченных лаб. работ \\ 
\hline
2 & Язык Питон: структурированные типы.           & Решение задач  & 30 & Сдача домашних заданий лично, неоконченных лаб. работ \\ 
\hline
3 & Язык Питон: функции и рекурсия.	             & Решение задач  & 15 & Сдача домашних заданий лично, неоконченных лаб. работ \\ 
\hline
4 & Язык Си: простые типы и основы синтаксиса.    & Решение задач  & 14 & Сдача домашних заданий лично, неоконченных лаб. работ \\ 
\hline
5 & Язык Си/Си++: структурированные типы.         & Решение задач  & 13 & Сдача домашних заданий лично, неоконченных лаб. работ \\ 
\hline
6 & Язык Си: функции и рекурсия.	                 & Решение задач  & 13 & Сдача домашних заданий лично, неоконченных лаб. работ \\ 
\hline
7 & Алгоритмы сортировки.	                     & Решение задач  & 13 & Сдача домашних заданий лично, неоконченных лаб. работ \\ 
\hline
8 & Язык Си++: классы. Знакомство с STL.          & Решение задач  & 12 & Сдача домашних заданий лично, неоконченных лаб. работ \\ 
\hline
 & ИТОГО                                         &                & 134 &  \\ 

\hline
\end{longtable}


\section{Методические указания для обучающихся по освоению дисциплины}
Важное значение в освоении дисциплины имеет самостоятельная работа. Ключевым
ее видом является самостоятельное написание программ. Только самостоятельное
практическое написание программ, поиск и исправление ошибок в них могут обеспечить
действительное понимание тем курса.
\par
Последовательное и добросовестное изучение курса является основой для
выработки навыков алгоритмизации, чтения и отладки текстов программ,
ключевых для данного направления подготовки.



\subsubsection*{Рейтинговый регламент по дисциплине}
\begin{longtable}{|>{\raggedright\arraybackslash}p{114mm}|r|r|}
\hline
\centering\arraybackslash Вид выполняемой учебной работы (контролирующие мероприятия) & 
\multicolumn{1}{p{19mm}|}{\centering\arraybackslash{}Кол-во баллов (min)} & 
\multicolumn{1}{p{19mm}|}{\centering\arraybackslash{}Кол-во баллов (max)} \\
\hline
Посещение занятий   & 6  & 10 \\ 
\hline
Лабораторные работы & 15 & 24 \\ 
\hline
Домашние задания    & 11 & 17 \\ 
\hline
Контрольные тесты   & 13 & 19 \\ 
\hline
\bf Кол-во баллов для допуска к экзамену в 1 сем. (min--max) & \bf 45 & \bf 70 \\ 
\hline
Посещение занятий   & 6  & 10 \\ 
\hline
Лабораторные работы & 12 & 18 \\ 
\hline
Домашние задания    & 16 & 25 \\ 
\hline
Контрольные тесты   & 11 & 17 \\ 
\hline
\bf Кол-во баллов для допуска к экзамену вo 2 сем. (min--max) & \bf 45 & \bf 70 \\ 

\hline
\end{longtable}

\section{Фонд оценочных средств для проведения промежуточной аттестации обучающихся по дисциплине}

\subsection{Показатели, критерии и шкала оценивания}

\begin{longtable}{|p{15mm}|p{45mm}|p{16mm}|p{51mm}|p{14mm}|}
\hline
  \centering\small Коды оцениваемых компетенций
& \centering Показатель оценивания (дескриптор) (по п.1.2) 
& \centering\small Уровни освоения 
& \centering Критерий оценивания 
& \centering\small\arraybackslash Оценка
\endhead
\hline

\multirow{4}{15mm}{ОПК-2, ПК-1}
&
\multirow{4}{45mm}{\parbox{45mm}{%
\vrule width 0pt height 10pt \emph{знать:}\newline
способы передачи параметров в функцию в языке Си++; несколько методов сортировки, включая быструю сортировку; \newline
\emph{уметь:}\newline
анализировать, объяснять поведение, модифицировать, тестировать небольшие программы на языках Питон и Си++, использующие любую комбинацию следующих понятий: простые вычисления, ветвление, итерация, простой ввод-вывод, в том числе в/из текстовых файлов, массивы (списки), функции; сравнивать два алгоритма для решения несложной задачи; \newline
\emph{владеть навыками:}\newline
отладки программ на языке Си++ в современной инструментальной среде; написания небольших программ на языках Питон и Си++ с использованием условного оператора и различных видов циклов; 
}}
& 
высокий & \raggedright\arraybackslash 
может правильно написать на языке Си функцию, изменяющую аргументы, переданные по ссылке, например заполняющую структуру;\newline
может объяснить механизм работы оператора присваивания для кортежей, списков, словарей в языке Питон;\newline
может реализовать сортировку пузырьком и быструю сортировку на языке Си, объяснить преимущество последней;\newline
может найти ошибку типа <<ошибка на единицу>> в некорректно реализованных на языках Си или Питон функциях сортировки списка или массива, агрегации списка или массива, обработки строки;\newline
может найти ошибку в циклической программе, используя отладку в инструментальной среде; может написать, запустить, протестировать и отладить на языках Си или Питон программу, вычисляющую агрегирующую функцию от строки или числового массива или списка, в том числе рекурсивно; \newline
может объяснить, какую агрегирующую функцию 
& отлично 
\\
\hline
&&&
от числового массива или списка вычисляет данная функция (в том числе рекурсивная), в пределах 20 строк;\newline
имея документацию, может написать вызов на языках Си и Питон библиотечной функции для конкретной задачи, правильно организовав подключение нужных пакетов/заголовочных файлов и передав параметры;\newline
может построить пример ввода, на котором предъявленная дефектная реализация простой программы обработки массива/списка на языках Си или Питон дает неверный результат;\newline
может пользоваться для отладки программы на Си++: точками останова, в том числе условными, пошаговым исполнением, шагом с заходом и с обходом по вызовам функций, просмотром значений переменных и памяти остановленной программы. &
\\ 
\cline{3-5}
& & базовый & может объяснить, почему изменяются значения переменных в вызывающей функции при передаче по ссылке на языке Си;\newline
может реализовать сортировку пузырьком на языке Си;\newline
может найти ошибку типа <<ошибка на единицу>> в некорректно реализованных на языках Си или Питон функциях сортировки списка или массива, & хорошо 
\\
\hline
& & & 
агрегации списка или массива, обработки строки;\newline
может написать, запустить, протестировать и отладить на языках Си или Питон программу, вычисляющую агрегирующую функцию от числового массива или списка;\newline
может объяснить, какую агрегирующую функцию от числового массива или списка вычисляет данная функция, в пределах 15 строк;\newline
имея документацию и примеры использования, может написать вызов на языках Си и Питон библиотечной функции для конкретной задачи, правильно организовав подключение нужных пакетов/заголовочных файлов и передав параметры;\newline
может вручную трассировать по тексту несложные циклические программы в пределах 4 итераций и правильно находить значения переменных после их завершения при различных входных данных;\newline
может пользоваться для отладки программы на Си++: точками останова, пошаговым исполнением, шагом с заходом и с обходом по вызовам функций, просмотром значений переменных остановленной программы. 
&
\\
\hline
& & мини\-мальный & может объяснить, почему изменяются значения переменных в вызывающей функции при передаче по ссылке;\newline
может написать, запустить, протестировать и отладить на языках Си или Питон программу, вычисляющую минимум или максимум чисел, удовлетворяющих некоторому условию, из числового массива или списка;\newline
может объяснить, какую агрегирующую функцию от числового массива или списка вычисляет данная функция, в пределах 15 строк;\newline
может вручную трассировать по тексту несложные циклические программы в пределах 4 итераций и правильно находить значения переменных после их завершения при различных входных данных;\newline
может пользоваться для отладки программы на Си++: точками останова, пошаговым исполнением, просмотром значений переменных остановленной программы. & удовл. 
\\

\cline{3-5}
& & не освоены & не способен выполнить три или более пунктов из вышеперечисленного & не зачтено 
\\

\hline

\end{longtable}



\newpage
\subsection{Типовые контрольные задания (вопросы) для промежуточной аттестации}

\begin{longtable}{|p{15mm}|p{42mm}|p{17mm}|p{70mm}|}
\hline
\centering\small Коды оцениваемых компетенций  & \centering Оцениваемый показатель (ЗУВ) 
& \centering Тема  & \centering\arraybackslash Образец типового (тестового или практического) задания (вопроса)
\endhead
\hline

ОПК-2, ПК-1 & 
знать способы передачи параметров в функцию в языке Си++; & 
6 & 
Какие значения будут напечатаны данной программой:
\begin{verbatim}
void f(int x) {
  printf("%%d",x); x++;}
void g(int &x) {
  printf("%%d",x); x++;}
int main(){
  int x = 1; f(x); g(x);
  return 0;}
\end{verbatim} 
\\
\hline
ОПК-2, ПК-1 & 
знать несколько методов сортировки, включая быструю сортировку; & 
7 & 
Сколько раз будет выполнена функция \texttt{swap()} для массива \texttt{A\,=\,\{3,2,5,4,1\}} при следующей реализации сортировки пузырьком:
\begin{verbatim}
for (i=0; i<N; ++i)
  for (j=i; j<N; ++j)
    if (a[i]>=a[j])
      swap(A, i, j);
\end{verbatim}
Как можно улучшить эту реализацию? 
\\
\hline
ОПК-2, ПК-1 & 
уметь анализировать, объяснять поведение, модифицировать, тестировать и отлаживать небольшие программы на языках Питон и Си++, использующие любую комбинацию следующих понятий: простые вычисления, ветвление, итерация, простой ввод-вывод, в том числе в/из текстовых файлов, массивы (списки), функции; & 
1, 4 & 
Даны действительные числа $a$, $b$, $c$, причем $a\ne 0$. Даны ординаты $y_1$, $y_2$, $y_3$ точек $L$, $M$, $N$ на прямой $ax+by+c=0$. Вывести координаты точек $L$, $M$, $N$ в порядке слева направо (или сверху вниз, если прямая вертикальна). 
\\
\hline
ОПК-2, ПК-1 & 
уметь анализировать, объяснять поведение, модифицировать, тестировать и отлаживать небольшие программы на языках Питон и Си++, использующие любую комбинацию следующих понятий: простые вычисления, ветвление, итерация, простой ввод-вывод, в том числе в/из текстовых файлов, массивы (списки), функции; & 
2 & 
Приведите пример списка {\tt L} из трех чисел, для которого данный код напечатает 2:\newline
\verb!i, M, iM = 0, 0, -1!\newline
\verb!for x in L:!\newline
\verb!  if x%%2 == 1:!\newline
\verb!    if x > M:!\newline
\verb!      M = x!\newline
\verb!      iM = i!\newline
\verb!  i += 1!\newline
\verb!if iM >= 0:!\newline
\verb!  print(iM) ! 
\\
\hline
ОПК-2, ПК-1 & 
анализировать, объяснять поведение, модифицировать, тестировать небольшие программы на языках Питон и Си++, использующие любую комбинацию следующих понятий: простые вычисления, ветвление, итерация, простой ввод-вывод, в том числе в/из текстовых файлов, массивы (списки), функции; & 
3 & 
Имеется некорректно написанная рекурсивная функция, подсчитывающая количество вхождений числа {\tt x} в список {\tt A}:\newline
\verb!def occurs(x, A):!\newline
\verb!    if A[0] == x:!\newline
\verb!        return 1+occurs(x, A[1:])!\newline
\verb!    return occurs(x, A[1:])!\newline
Исправьте ее 
\\
\hline
ОПК-2, ПК-1 & 
анализировать, объяснять поведение, модифицировать, тестировать небольшие программы на языках Питон и Си++, использующие любую комбинацию следующих понятий: простые вычисления, ветвление, итерация, простой ввод-вывод, в том числе в/из текстовых файлов, массивы (списки), функции; & 
6 & 
Имеется некорректно написанная рекурсивная функция, подсчитывающая количество вхождений числа {\tt x} в массив {\tt A} длины {\tt n}:\newline
\verb!int occurs(int x, int n, int* A) {!\newline
\verb!  if (A[0] == x)!\newline
\verb!    return 1+occurs(x, n-1, A+1);!\newline
\verb!  return occurs(x, n-1, A+1);!\newline
\verb!}!\newline
Исправьте ее 
\\
\hline
ОПК-2, ПК-1 & 
владеть навыками отладки программ на языке Си++ в современной инструментальной среде; & 
4 & 
Имеется программа нахождения трех максимальных членов числовой последовательности,
предполагается, что среди чисел могут быть одинаковые. \newline
\verb!#include <stdio.h>!\newline
\verb!int main()!\newline
\verb!{!\newline
\verb!  int i, n, x, max1, max2, max3;!\newline
\verb!  scanf("%d", &n);!\newline
\verb!  if (n < 3) return -1;!\newline
\verb!  scanf("%d", &x);!\newline
\verb!  max1 = max2 = max3 = x;!\newline
\verb!  for (i=1; i<n; ++i){!\newline
\verb!    scanf(" %d", &x);!\newline
\verb!    if (x > max1) max1 = x; else!\newline
\verb!    if (x >= max2) max2 = x; else !\newline
\verb!    if (x >= max3) max3 = x;!\newline
\verb!  }!\newline
\verb!  printf("max=%d,max2=%d,max3=%d",!\newline
\verb!         max1, max2, max3);!\newline
\verb!  return 0;!\newline
\verb!}!\newline
Приведите пример входных данных, для которых она работает правильно,
и пример входных данных, для которых она работает неправильно. Исправьте программу. 
\\
\hline
ОПК-2, ПК-1 & 
владеть навыками написания небольших программ на языках Питон и Си++ с использованием условного оператора и различных видов циклов; & 
1, 4 & 
Натуральное число называется совершенным, если оно равно сумме всех своих делителей, включая единицу, но исключая себя. Например, число 28 --- совершенное: \[1 + 2 + 4 + 7 + 14 = 28.\] Напечатать все совершенные числа, меньшие заданного с клавиатуры числа $N$. 
\\
\hline
ОПК-2, ПК-1 & 
владеть навыками написания небольших программ на языках Питон и Си++ с использованием условного оператора и различных видов циклов; & 
2, 5 & 
Ввести строку. Гарантируется, что в ней только пробелы и латинские буквы. Подсчитать число слов в строке. 
\\
\hline
ОПК-2, ПК-1 & 
владеть навыками написания небольших программ на языках Питон и Си++ с использованием условного оператора и различных видов циклов; & 
2 & 
Ввести с клавиатуры положительное целое $n$, а затем $n$ целых чисел. Найти любое наиболее часто встречающееся число, используя словарь. 
\\
\hline
ОПК-2, ПК-1 & 
уметь сравнивать два алгоритма для решения несложной задачи; & 
5 & 
Сравните два алгоритма для подсчета количества различных чисел в массиве
{\tt A} длины {\tt n} неотрицательных чисел, меньших {\tt N}:\newline
\verb!const int N = ...;!\newline
\verb!int uniq1(int n, int *A) {!\newline
\verb!  int i, j, res = 0;!\newline
\verb!  for (i=0; i < n; ++i){!\newline
\verb!    bool last = true;!\newline
\verb!    for (j=i+1; j < n; ++j)!\newline
\verb!      if (A[i]==A[j]) last=false;!\newline
\verb!    if (last) res++;!\newline
\verb!  }!\newline
\verb!  return res;!\newline
\verb!}!\newline
\verb!int uniq2(int n, int *A) {!\newline
\verb!  int i, res = 0;!\newline
\verb!  bool* h = new bool[N];!\newline
\verb!  for (i=0; i<N; ++i) h[i] = 0;!\newline
\verb!  for (i=0; i<n; ++i) h[A[i]] = 1;!\newline
\verb!  for (i=0; i<n; ++i) res += h[i];!\newline
\verb!  return res;!\newline
\verb!}!\newline
Приведите примеры, когда каждый из алгоритмов совершает меньше действий, чем второй. При каких условиях лучше тот или другой? 
\\
\hline
ОПК-2, ПК-1 & 
владеть навыками написания небольших программ на языках Питон и Си++ с использованием условного оператора и различных видов циклов; & 
8 & 
Со стандартного ввода вводится число $n>0$, затем $n$ целых чисел.
Найдите максимум отрицательных чисел, с обязательным использованием
следующих элементов: потокового ввода (\texttt{<iostream>}), векторов и итераторов (\texttt{<vector>}).
Если отрицательные числа не вводились, выведите \texttt{"ERROR"}. 
\\
\hline
\end{longtable}
\subsubsection*{Вопросы к экзамену за I семестр}
\begin{enumerate}
\item
Типы в языке Питон. Динамическая типизация. \texttt{int}, \texttt{float}, \texttt{bool}, \texttt{str}, \texttt{NoneType}. Явное преобразование типов. Функции \texttt{type()}, \texttt{dir()}, \texttt{help()}.
\item
Числовые типы \texttt{int}, \texttt{bool}, \texttt{float}. Преобразование числовых типов друг в друга. Числовые операторы, в том числе степень, деление нацело, остаток по модулю, побитовые сдвиги влево и вправо.
\item
Логический тип \texttt{bool}. Преобразование числовых, строковых значений и \texttt{None} в \texttt{bool}. Логические операторы.
\item
Операторы \texttt{if}, \texttt{while}. Отступы и блоки. Пример.
\item
Оператор \texttt{while}. Нахождение НОД двух положительных целых чисел.
\item
Кортежи, присваивание и распаковка кортежей. Списки и строки. Присваивание списков. Изменяемые и неизменяемые типы. Функция \texttt{append()}.  Длина. Индексация, отрицательные индексы, диапазоны, шаг. Пропуск начального, конечного значений диапазона индексов. Оператор \texttt{in}. Сложение и умножение на число для кортежей, списков и строк.
\item
Списки. Присваивание списков. Функция \texttt{append()}. Индексация, длина, отрицательные индексы, диапазоны, шаг. Пропуск начального, конечного значений диапазона индексов. Ввод числового списка с клавиатуры. Перечисления, отбор значений, примеры.
\item
Списки. Присваивание списков. Ввод числового списка с клавиатуры. Порождение списка случайных чисел заданной длины. Нахождение максимума, минимума и среднего арифметического для числового списка.
\item
Оператор \texttt{for}. Объекты \texttt{range}. Начальное значение, шаг. Перебор символов строки. Перебор строк файла.
\item
Двумерные массивы в виде списков. Инициализация нулевой матрицы $m \times n$. Ввод матрицы с клавиатуры, $m$ строк по $n$ значений в строке.
\item
Функции. Параметры. Видимость переменных, глобальные и локальные переменные. Примеры.
\item
Словари. Допустимые типы ключей. Проверка наличия ключа в словаре. Перебор словаря, функция \texttt{items()}.
\item
Рекурсия. Примеры: рекурсивное вычисления максимума, минимума, факториала, чисел Фибоначчи; печать ключей бинарного дерева, заданного в виде словаря.
\item
Пузырьковая сортировка. Количество действий для массива длины $n$.
\item
Быстрая сортировка. Упорядочение числового массива на три части относительно выбранного срединного значения.
\end{enumerate}
{}
\subsubsection*{Вопросы к экзамену за II семестр}
\begin{enumerate}
\item
Состав языка Си. Алфавит языка. Идентификаторы. Ключевые слова. Комментарии.
\item
Типы данных. Стандартные типы (целые, со знаком и без, вещественные, \texttt{char}, \texttt{bool}).
\item
Пользовательские типы данных. Переименование типов. Перечисления (\texttt{enum}), структуры (\texttt{struct}), объединения (\texttt{union}).
\item
Операции. Операторы. Составной оператор.
\item
Алгебраические выражения, их типы и правила вычисления. Функции стандартной библиотеки из \texttt{<math.h>}.
\item
Логические выражения, их типы и правила вычисления.
\item
Инициализация переменных (стандартных, строк, массивов).
\item
Условные операторы. Оператор безусловного перехода. Оператор выбора.
\item
Программы циклической структуры. Оператор цикла с параметром (\texttt{for}).
\item
Программы циклической структуры. Оператор цикла с предусловием (\texttt{while}).
\item
Программы циклической структуры. Оператор цикла с постусловием \texttt{do}-\texttt{while} (\texttt{repeat}).
\item
Структурированные типы данных. Массивы. Описание типа. Действия над массивами. Действия над элементами массива.
\item
Функции стандартной библиотеки для работы со строками и символами.
\item
Динамические структуры. Очереди.
\item
Текстовые файлы.
\item
Бинарные файлы.
\item
Объектно-ориентированное программирование. Классы. Наследование. Конструкторы и деструкторы.
\item
Перегрузка операций.
\item
Перегрузка функций, шаблоны функций.
\item
Динамические структуры. Дерево.
\item
Рекуррентные формулы. Вычисление конечных сумм и произведений.
\item
Рекуррентные формулы. Приближенное Вычисление бесконечных сумм и произведений.
\item
Перевод целого числа из десятичной системы счисления (СС) в десятчную СС. Перевод целого числа из десятичной системы счисления в $p$-ичную СС.
\item
Решето Эратосфена.
\item
Сортировка методом обмена (пузырька).
\item
Сортировка методом выбора.
\item
Сортировка методом вставки.
\item
Быстрая сортировка Хоара.
\item
Задачи поиска. Линейный поиск. Бинарный поиск.
\item
Реализация динамических структур с помощью массивов. Стеки.
\item
Обратная польская запись.
\item
Схема Горнера. Перевод числа из p-ичной системы счисления в десятичную.
\item
Определение площади произвольного n-угольника (без самопересечений).
\item
Перевод целого числа из двоичной СС в десятчную СС.
\item
Перевод целого числа из десятичной СС в $p$-ичную СС.
\item
Перевод целого числа из p-чной СС в десятичную СС.
\item
Задачи поиска. Бинарный поиск.
\item
Задача о восьми ферзях.
\item
Схема Горнера. Вычисление значения многочлена $n$-й степени.
\end{enumerate}



\subsection{Методические материалы, определяющие процедуры оценивания}

\textit{Лабораторные работы.}\/ Во время лабораторных занятий по каждой теме
обучающиеся должны самостоятельно написать программы для задач в
описании лабораторной работы.  Решения сдаются лично преподавателю или отправляются
через веб-браузер в автоматизированную проверяющую систему <<Мультиметр>>. Задания очередной
лабораторной работы могут быть сданы не позднее следующего лабораторного занятия.
\par
\textit{Домашние задания.}\/ Домашние задания выполняются по индивидуальным вариантам.
Срок сдачи очередного комплекта домашних заданий --- как правило, не позднее двух
недель после его выдачи. Каждое задание сдается преподавателю лично.
\par
\textit{Онлайн-тестирование.}\/ Данная форма текущего контроля направлена на оценку основных
теоретических знаний обучающегося по мере освоения разделов дисциплины. Предполагает
ответы на вопросы теста через веб-формы, например Google Forms, либо
исправление и сдачу выданных <<заготовок>> (незаконечнных программ или
программ с дефектами) в автоматизированной проверяющей системе <<Мультиметр>>.
\par
\textit{Форма промежуточной аттестации: экзамен.}\/ К экзамену допускаются студенты,
выполнившие обязательный минимум учебной работы и набравшие в семестре не менее 45~баллов.
Данный вид комплексного испытания предполагает ответ по билету, содержащему один
теоретический и два практических вопроса. Последние предполагают написание программы.
На экзамене можно набрать до 30 баллов: 9 за теоретический вопрос, 9 и 12 баллов за
практические вопросы.



\newpage
\section{Перечень основной и дополнительной учебной литературы, необходимой для освоения дисциплины}

  \begin{longtable}{|l|p{7cm}|p{18mm}|c|p{32mm}|}
  \caption*{Перечень литературы}\\
  \hline
  № & 
  \centering\small\arraybackslash Автор, название, место издания, издательство, год издания учебной литературы, вид и характеристика иных информационных ресурсов &
  \multicolumn{1}{p{18mm}|}{\centering\small\arraybackslash Наличие грифа, вид грифа} &
  \multicolumn{1}{p{21mm}|}{\centering\small\arraybackslash НБ СВФУ, кафедральная библиотека и кол-во экземпляров} & 
  \centering\small\arraybackslash Электронные издания: точка доступа к ресурсу (наименование ЭБС, ЭБ СВФУ)\\
  \hline
  \multicolumn{5}{|c|}{Основная литература}\\
  \hline
  1 &\raggedright\arraybackslash Мозговой М. В. Классика программирования: алгоритмы, языки, автоматы, компиляторы: практический подход. СПб: Наука и Техника, 2006. 320 с.  &   &  10  &  
  \\
  \hline
  2 &\raggedright\arraybackslash Павловская Т. А. С/С++. СПб.: Питер, 2013.  &  & 13 &  
  \\
  \hline
  
  \multicolumn{5}{|c|}{Дополнительная литература}\\
  \hline
  1 &\raggedright\arraybackslash Кормен Т.\,Х. Лейзерсон Ч., Ривест Р. Алгоритмы: построение и анализ. М.: МЦНМО, 1999.  &   &  1  &  
  \\
  \hline
  2 &\raggedright\arraybackslash Кнут, Д. Искусство программирования Т.\,1: Основные алгоритмы. М. 2004.  &   &  3 &  
  \\
  \hline
  3 &\raggedright\arraybackslash Страуструп Б., Язык программирования С++.  М.: Бином, 1999.   &   & 1 & \thispagestyle{empty} 
  \\
  \hline
  
  \end{longtable}
  
\section{Перечень ресурсов информационно-телекоммуникационной сети «Интернет» (далее~--- сеть Интернет), необходимых для освоения дисциплины}
\begin{enumerate}
  
  \item Шень А.\,Х. Программирование: теоремы и задачи. М.: МЦНМО, 2004. 160 с. \\\texttt{http://www.mccme.ru/free-books/shen/shen-progbook.pdf} 
  
  \item Компания Sololearn. Мобильное приложение <<Учим Python>>. \\\texttt{https://play.google.com/store/apps/details?id=com.sololearn.python} или \\\texttt{https://itunes.apple.com/us/app/id953972812} или \\\texttt{http://www.windowsphone.com/s?appid=7bb32109-b882-49c9-8fa2-a500b79a19ca} 
  
  \item Онлайн-игра для обучения программированию CodeCombat. \\\texttt{https://codecombat.com/} 
  
  \item Игра для обучения программированию Colobot. \\\texttt{https://colobot.info/} 
  
\end{enumerate}


\newpage
\section{Описание материально-технической базы, необходимой для осуществления образовательного процесса по дисциплине}
  
  
       Для проведения лекционных занятий требуется аудитория, оборудованная доской,  мультимедийным проектором с экраном. 
       Для проведения лабораторных занятий требуется компьютерный класс с подключением к интернету.
  
  


\section{Перечень информационных технологий, используемых при осуществлении образовательного процесса по дисциплине, включая перечень программного обеспечения
}

\subsection{Перечень информационных технологий, используемых при осуществлении образовательного процесса по дисциплине}

При осуществлении образовательного процесса по дисциплине используются следующие информационные технологии:
\begin{itemize}[nolistsep]
  
\item использование на занятиях электронных изданий (чтение лекций с использованием слайд-презентаций);
  
\item ведение учета посещаемости и выполнения учебных заданий в системе Google Docs;
  
\item написание программ на языках высокого уровня в инструментальных средах;
  
\item сдача программ в автоматической проверяющей системе <<Мультиметр>>;
  
\item организация взаимодействия с обучающимися посредством электронной почты, специализированного образовательного форума Piazza;
  
\item компьютерное тестирование.
  
\end{itemize}

\subsection{Перечень программного обеспечения}
При осуществлении образовательного процесса по дисциплине используются следующее программное обеспечение:
\begin{itemize}[nolistsep]
  
\item Язык программирования Python, среда IDLE;
  
\item Среда Visual Studio с компилятором Visual C++, или среда CodeBlocks с компилятором GNU C++;
  
\item интернет-браузер.
  
\end{itemize}



\newpage
\begin{center}
\section*{ЛИСТ АКТУАЛИЗАЦИИ РАБОЧЕЙ ПРОГРАММЫ ДИСЦИПЛИНЫ}
Б1.В.ОД.3\ --- Программирование 
\end{center}

  \noindent
  \begin{tabular}{|p{15mm}|p{67mm}|p{25mm}|p{41mm}|}
    \hline
    \small\centering
    Учебный год 
    & \small\centering
    Внесенные изменения 
    & \small\centering
    Преподаватель (ФИО) 
    & \small\centering\arraybackslash
    Протокол заседания выпускающей кафедры (дата, номер), ФИО зав.кафедрой, подпись \\
    & & & \\\hline
    & & & \\\hline
    & & & \\\hline
    & & & \\\hline
    & & & \\\hline
    & & & \\\hline
    & & & \\\hline
    & & & \\\hline
    & & & \\\hline
    & & & \\\hline
    & & & \\\hline
    & & & \\\hline
    & & & \\\hline
    & & & \\\hline
    & & & \\\hline
    & & & \\\hline
    & & & \\\hline
    & & & \\\hline
    & & & \\\hline
    & & & \\\hline
    & & & \\\hline
    & & & \\\hline
    & & & \\\hline
    & & & \\\hline
    & & & \\\hline
    & & & \\\hline
    & & & \\\hline
    & & & \\\hline
    & & & \\\hline
    & & & \\\hline
    & & & \\\hline
    & & & \\\hline
    & & & \\\hline
    & & & \\\hline
    & & & \\\hline
    & & & \\\hline
  \end{tabular}

  \medskip\noindent\textit{В таблице указывается только характер изменений (например, изменение темы, списка источников по~теме или темам, средств промежуточного контроля) с~указанием пунктов рабочей программы. Само содержание изменений оформляется приложением по~сквозной нумерации.}

\newpage\tableofcontents

\end{document}