\documentclass[a4paper,12pt]{article}
  \usepackage[hmargin={30mm, 17mm}, vmargin={20mm,20mm}, nohead]{geometry}
  \usepackage{polyglossia, enumitem, array, longtable, multirow, tabularx, rotating, indentfirst, caption, float, ulem, titlesec }%}
  \setdefaultlanguage{russian} % выбор основного языка (для переносов)
  % шрифты 
  \defaultfontfeatures{Ligatures=TeX} % нужен для того, чтобы работали стандартные сочетания символов ---, -- << >> и т.п.
  \setmainfont{Times New Roman} 
  \setmonofont[Scale=0.9]{Consolas}
  \setsansfont{Trebuchet MS}

  

  % pdf metadata
  \usepackage[
    pdfencoding=unicode,
    pdftitle={Рабочая программа дисциплины Математические основы защиты информации},
    pdfauthor={составлена }, 
    bookmarksnumbered=true, 
      colorlinks=true, linktocpage=true, linkcolor=blue, pdfpagemode=UseOutlines]
  {hyperref}

  % формат заголовков, подписей и списков
  \titleformat*{\section}{\large\bfseries\centering}
  \titleformat*{\subsection}{\bfseries}
  \captionsetup[table]{singlelinecheck=off, font=small, labelsep=period, textfont=it, format=hang, justification=raggedleft}
  \setlist[enumerate,1]{nolistsep,labelindent=0pt,leftmargin=\parindent}
  \setlist[itemize,1]{nolistsep,labelindent=20pt,leftmargin=*,label=--}

  \makeatletter

  % заполняет ширину текста полем с подчеркиванием.
  % #1 - текст слева, #2 - на подчеркивании, #3 - справа, #4 - подпись под подчеркнутым
  \newcommand{\ulfield}[4]{
  \noindent
  \begin{tabularx}{\linewidth}{@{}l@{}X@{}l@{}}
  #1\if\relax\detokenize{#1}\relax\else\,~\vrule width 0pt\fi 
  & \uline{\vrule width 0pt\hfill#2\hfill\vrule width 0pt} & 
  \if\relax\detokenize{#3}\relax\else\vrule width 0pt~\,\fi #3
  \\
  & {\scriptsize \vrule width 0pt\hfill#4\hfill\vrule width 0pt}
  \end{tabularx}
  }

  %%% Перенос составных слов
    \XeTeXinterchartokenstate=1
    \XeTeXcharclass `\- 24
    \XeTeXinterchartoks 24 0 = {\hskip\z@skip}
    \XeTeXinterchartoks 0 24 = {\nobreak}

  \newcommand\rotleft{\rotatebox{90}}
  
  \makeatother

  \newcommand{\datefield}[1][]{\if
  \relax\detokenize{#1}\relax«\uline{\hspace{22pt}}»~\uline{\hspace{90pt}}\,~20\uline{\hspace{20pt}}~г.\else «\uline{\hspace{18pt}}»~\uline{\hspace{60pt}}\,~20\uline{\hspace{18pt}}~г.\fi
  }


%%% Настройка содержания
\AtBeginDocument{
  \makeatletter 
  \def\@tocline#1#2#3#4#5#6#7{\relax
  \ifnum #1>\c@tocdepth % then omit
  \else
    \par \addpenalty\@secpenalty\addvspace{#2}%
    \begingroup \hyphenpenalty\@M
    \@ifempty{#4}{\@tempdima\csname r@tocindent\number#1\endcsname\relax}{\@tempdima#4\relax}%
    \parindent\z@ \leftskip#3\relax \advance\leftskip\@tempdima\relax
    \rightskip\@pnumwidth plus4em \parfillskip-\@pnumwidth
    #5\leavevmode\hskip-\@tempdima
      \ifcase #1
       \or\or \hskip 1em \or \hskip 2em \else \hskip 3em \fi%
      #6\nobreak\relax
    \dotfill\hbox to\@pnumwidth{\@tocpagenum{#7}}\par
    \nobreak
    \endgroup
  \fi}
  \makeatother
}%AtBeginDocument

\begin{document}
\sloppy
\thispagestyle{empty}

\noindent
\begin{center}
Министерство образования и науки Российской Федерации \\
Федеральное государственное автономное образовательное \\
учреждение высшего образования\\
«СЕВЕРО-ВОСТОЧНЫЙ ФЕДЕРАЛЬНЫЙ УНИВЕРСИТЕТ \\
имени М.\,К.~АММОСОВА» \\
Институт математики и информатики \\
Кафедра информационных технологий

\vspace{12mm}
\begin{flushright}
\parbox{80mm}{
УТВЕРЖДАЮ\\
Директор ИМИ\\[2mm]
\ulfield{}{}{/\,В.\,И.~Афанасьева\,/}{}\\
\datefield
\\[20mm]
}
\end{flushright}


\textbf{РАБОЧАЯ ПРОГРАММА ДИСЦИПЛИНЫ}
\\[2mm]
\textbf{Б1.В.ОД.1.3\ -- Математические основы защиты информации} 
\\[5mm]

для программы магистратуры\\
по направлению подготовки \\
09.04.01 -- Информатика и вычислительная техника
\\[15mm]

\begin{tabular}{p{0.3\textwidth}p{0.3\textwidth}p{0.3\textwidth}}
  ОДОБРЕНО &  ОДОБРЕНО  & РЕКОМЕНДОВАНО \\
  Заведующий кафедрой \newline разработчика &
  Заведующий выпускающей кафедрой ИТ&
  Нормоконтроль в составе ОП пройден \\
  \ulfield{}{}{\uline{/\hspace{30mm}/}}{} &
  \ulfield{}{}{\uline{/\hspace{30mm}/}}{} &
  \ulfield{}{}{\uline{/\hspace{30mm}/}}{} \\
  Протокол № \uline{\hspace{13pt}} от\newline \datefield[small] & 
  Протокол № \uline{\hspace{13pt}} от\newline \datefield[small] & 
  Протокол № \uline{\hspace{13pt}} от\newline \datefield[small] 
\end{tabular}
\par\vfill\vspace{6mm}
Якутск -- 2016

\end{center}


\newpage


\begin{center}
\section{АННОТАЦИЯ}
  {\bf к рабочей программе дисциплины\\
  Б1.В.ОД.1.3\ -- Математические основы защиты информации} \\
  Трудоемкость \uline{~4~} з.~е.
\end{center}


\subsection{Цель освоения и краткое содержание дисциплины}
  
  Целью изучение дициплины <<Математические основы защиты информации>> является: Дать представление о математических основах наиболее значимых алгоритмов, применяемых для защиты информации.
  


\subsection{Перечень планируемых результатов обучения по дисциплине, соотнесенных с~планируемыми результатами освоения образовательной программы}

\begin{longtable}{|p{45mm}|p{109mm}|}
  \caption{Перечень планируемых результатов обучения}\\
  \hline
  \centering
  Планируемые результаты освоения программы (содержание и коды компетенций) & 
  \centering\arraybackslash
  Планируемые результаты обучения по~дисциплине
  \\
  \hline
  
  ОК-6 : способностью проявлять инициативу, в том числе в ситуациях риска, брать на себя всю полноту ответственности, \par 
  
  ОПК-1 : способностью воспринимать математические, естественнонаучные, социально-экономические и профессиональные знания, умением самостоятельно приобретать, развивать и применять их для решения нестандартных задач, в том числе в новой или незнакомой среде и в междисциплинарном контексте, \par 
  
  ОПК-5 : владением методами и средствами получения, хранения, переработки и трансляции информации посредством современных компьютерных технологий, в том числе в глобальных компьютерных сетях, \par 
  
  ПК-7 : применением перспективных методов исследования и решения профессиональных задач на основе знания мировых тенденций развития вычислительной техники и информационных технологий
  & 
  В результате изучения дисциплины обучающийся должен:\newline
  \emph{знать:}
  принципы работы алгоритмов Диффи-Хеллмана и RSA; основные проблемы реализации и известные атаки на один из них
  

  \emph{уметь:}
  объяснять разницу между симметричными и асимметричными криптосистемами; оценивать надежность системы при помощи теоретико-сложностных оценок
  

  \emph{владеть навыками:}
  применения и реализации основных эффективных теоретико-числовых алгоритмов, включая нахождение НОД, арифметику в кольцах вычетов, нахождение обратного и степени вычета mod $p$.
  
  \\
  \hline
  \end{longtable}


\subsection{Место дисциплины в~структуре образовательной программы}

  \begin{table}[H]
  \setlength\arraycolsep{3pt}
  \caption{Содержательно-логические связи дисциплины}
  \begin{tabular}{|l|p{18ex}|*{2}{p{23ex}|}}
  \hline
  \multicolumn{1}{|c|}{\multirow{2}{13ex}{\centering Индекс \linebreak дисциплины}} &
  \multicolumn{1}{c|}{\multirow{2}{18ex}{\centering Наименование \linebreak дисциплины}} & 
  \multicolumn{2}{p{46ex}|}{\centering Коды учебных дисциплин, практик} \\
  \cline{3-4}
   & & 
  \centering на которые опирается содержание дисциплины & 
  \centering\arraybackslash для которых содержание дисциплины выступает опорой
  \\ \hline
  Б1.В.ОД.1.3 & Математические основы защиты информации 
  & 
  
   
  & 
  
   
  \\ \hline
  \end{tabular}
  \end{table}


\subsection{Язык преподавания} 
  Русский.
  



\newpage

\section{Объем дисциплины в зачетных единицах с указанием количества академических часов, выделенных на контактную работу обучающихся с преподавателем (по~видам учебных занятий) и~на~самостоятельную работу обучающихся}

\begin{table}[H]
\caption{Выписка из учебного плана} 
\begin{tabular}{|p{9cm}|c|c|}
\hline
Код и название дисциплины по учебному плану & \multicolumn{2}{p{6cm}|}{Б1.В.ОД.1.3\ -- Математические основы защиты информации }\\
\hline
Курс изучения &\multicolumn{2}{c|}{ 1 }\\
\hline
Семестр(ы) изучения &\multicolumn{2}{c|}{ 2 }\\
\hline
Форма промежуточной аттестации (зачет/экзамен) &\multicolumn{2}{c|}{ экзамен }\\
\hline
Курсовой проект / курсовая работа (указать вид работы при наличии в учебном плане), семестр выполнения &\multicolumn{2}{c|}{ }\\
\hline
Трудоемкость (в ЗЕТ) &\multicolumn{2}{c|}{ 4 (4) }\\
\hline
{\bf Трудоемкость (в часах)} (сумма строк №1, 2, 3), в~т.~ч.:& \multicolumn{2}{c|}{144}\\
\hline
\textbf{№\,1. Контактная работа обучающихся с преподавателем (КР),} в часах:
& \multicolumn{1}{p{3cm}|}{\centering Объем аудиторной работы, в часаx}
& \multicolumn{1}{p{3cm}|}{\centering\arraybackslash В~т.\,ч. с~применением ДОТ или ЭО, в~часах}\\
\hline  
Объем работы (в часах) (1.1.+1.2.+1.3.)& 54 & \\
\hline
1.1. Занятия лекционного типа (лекции) & 15 & \\
\hline
1.2. Занятия семинарского типа, всего, в т.ч.: & & \\
\hline
- семинары (практические занятия, коллоквиумы и~т.~п.)  & – & \\
\hline
- лабораторные работы& 34 & \\
\hline
- практикумы & & \\
\hline
1.3. КСР (контроль самостоятельной работы, консультации)& 5 & \\
\hline
{\bf №\,2. Самостоятельная работа обучающихся (СРС) (в часах)}& \multicolumn{2}{c|}{54}\\
\hline
{\bf №\,3. Количество часов на экзамен (при наличии экзамена в учебном плане)}& \multicolumn{2}{c|}{36}\\
\hline
\end{tabular}
\end{table}



\newpage
\section{Содержание дисциплины, структурированное по~темам с~указанием отведенного на~них количества академических часов и~видов учебных занятий}
\subsection{Распределение часов по~темам и~видам учебных занятий}
\begin{longtable}{|>{\raggedright\arraybackslash}p{59mm}|c|c|c|c|c|c|c|c|c|c|c|}
\caption{}
\\
\hline
 & & 
\multicolumn{9}{c|}{Контактная работа, в часах} & 
\\
\cline{3-11} 
\raisebox{18mm}{Тема}&
\rotleft{Всего часов} &
\rotleft{Лекции} &
\rotleft{из них с прим-м  ЭО и ДОТ} &
\rotleft{\parbox{5cm}{\raggedright\arraybackslash Семинары  (практические занятия, коллоквиумы)}} &
\rotleft{из них с прим-м  ЭО и ДОТ} &
\rotleft{Лабораторные работы} &
\rotleft{из них с прим-м  ЭО и ДОТ} &
\rotleft{Практикумы} &
\rotleft{из них с прим-м  ЭО и ДОТ} &
\rotleft{КСР (консультации)} & 
\rotleft{Часы СРС}
\\
\hline
Тема 1. Шифрование и криптоанализ на простейших примерах & 31 & 3 & 1 & 0 & 0 & 7 & 5 & 0 & 0 & 1 & 14 \\ 
\hline
Тема . Основы теории информации и кодирования           & 24 & 3 & 1 & 0 & 0 & 7 & 5 & 0 & 0 & 1 & 7 \\ 
\hline
Тема 3. Хэширование и односторонние функции              & 31 & 3 & 1 & 0 & 0 & 7 & 5 & 0 & 0 & 1 & 14 \\ 
\hline
Тема 4. Криптография с открытым ключом                   & 29 & 3 & 1 & 0 & 0 & 7 & 5 & 0 & 0 & 1 & 12 \\ 
\hline
Тема 5. Анализ данных и визуализация в Питоне       & 21 & 3 & 0 & 0 & 0 & 6 & 4 & 0 & 0 & 1 & 7 \\ 
\hline
ВСЕГО ЧАСОВ & 108 & 15 & 4 & 0 & 0 & 34 & 24 & 0 & 0 & 5 & 54 \\ 

\hline
\end{longtable}

\subsection{Содержание тем программы дисциплины} 


\textbf{Тема 1. Шифрование и криптоанализ на простейших примерах}\\


\textbf{Тема . Основы теории информации и кодирования          }\\


\textbf{Тема 3. Хэширование и односторонние функции             }\\


\textbf{Тема 4. Криптография с открытым ключом                  }\\


\textbf{Тема 5. Анализ данных и визуализация в Питоне      }\\

 

\subsection{Формы и методы проведения занятий, применяемые учебные технологии}




\section{Перечень учебно-методического обеспечения для самостоятельной работы обучающихся по дисциплине}
\begin{longtable}{|l|>{\raggedright\arraybackslash}p{40mm}|>{\raggedright\arraybackslash}p{54mm}|c|>{\raggedright\arraybackslash}p{30mm}|}
\hline
№ & \centering Наименование раздела (темы) дисциплины & 
\centering Вид СРС & \multicolumn{1}{p{14mm}|}{\centering Трудо\-емкость (в часах)} & \centering\arraybackslash Формы и методы контроля\\

\hline
\end{longtable}


\section{Методические указания для обучающихся по освоению дисциплины}



\subsubsection*{Рейтинговый регламент по дисциплине}
\begin{longtable}{|>{\raggedright\arraybackslash}p{110mm}|r|r|}
\hline
\centering\arraybackslash Вид выполняемой учебной работы (контролирующие мероприятия) & 
\multicolumn{1}{p{20mm}|}{\centering\arraybackslash Количество баллов (min)} & 
\multicolumn{1}{p{20mm}|}{\centering\arraybackslash Количество баллов (max)} \\

\hline
\end{longtable}

\newpage
\section{Фонд оценочных средств для проведения промежуточной аттестации обучающихся по дисциплине}

\subsection{Показатели, критерии и шкала оценивания}

\begin{longtable}{|p{15mm}|p{53mm}|p{16mm}|p{43mm}|p{14mm}|}
\hline
  \centering\small Коды оцениваемых компетенций
& \centering Показатель оценивания (дескриптор) (по п.1.2) 
& \centering\small Уровни освоения 
& \centering Критерий оценивания 
& \centering\small\arraybackslash Оценка
\\
\hline

\end{longtable}



\subsection{Типовые контрольные задания (вопросы) для промежуточной аттестации}

\begin{longtable}{|p{15mm}|p{42mm}|p{17mm}|p{70mm}|}
\hline
\centering\small Коды оцениваемых компетенций  & \centering Оцениваемый показатель (ЗУВ) 
& \centering Тема  & \centering\arraybackslash Образец типового (тестового или практического) задания (вопроса)
\\
\hline

\end{longtable}



\subsection{Методические материалы, определяющие процедуры оценивания}




\newpage
\section{Перечень основной и дополнительной учебной литературы, необходимой для освоения дисциплины}

  \begin{longtable}{|l|p{7cm}|p{18mm}|c|p{32mm}|}
  \caption*{Перечень литературы}\\
  \hline
  № & 
  \centering\small\arraybackslash Автор, название, место издания, издательство, год издания учебной литературы, вид и характеристика иных информационных ресурсов &
  \multicolumn{1}{p{18mm}|}{\centering\small\arraybackslash Наличие грифа, вид грифа} &
  \multicolumn{1}{p{21mm}|}{\centering\small\arraybackslash НБ СВФУ, кафедральная библиотека и кол-во экземпляров} & 
  \centering\small\arraybackslash Электронные издания: точка доступа к ресурсу (наименование ЭБС, ЭБ СВФУ)\\
  \hline
  \multicolumn{5}{|c|}{Основная литература}\\
  \hline
  1 &\raggedright\arraybackslash Элементы теории обыкновенных представлений и характеров конечных групп с приложениями в криптографии.  СПб. : Лань, 2015.  &   &  ---  &  ЭБС <<Лань>>, режим доступа: http://e.lanbook .com/book/65044 
  \\
  \hline
  2 &\raggedright\arraybackslash Бабенко Л. К., Параллельные алгоритмы для решения задач защиты информации. М.: Горячая линия-Телеком, 2014  &   &  1  &  
  \\
  \hline
  
  \multicolumn{5}{|c|}{Дополнительная литература}\\
  \hline
  1 &\raggedright\arraybackslash Левин Максим. PGP: Кодирование и шифрование информации с открытым ключом. М: Майор, 2001                    &   &  1  &  
  \\
  \hline
  2 &\raggedright\arraybackslash Глухов М. М., Круглов И. А., Пичкур А. Б., Черемушкин А.В. Введение в теоретико-числовые методы криптографии. СПб. : Лань, 2011. &   & --- &  ЭБС <<Лань>>, режим доступа: http://e.lanbook .com/book/1540 
  \\
  \hline
  3 &\raggedright\arraybackslash Кормен Т. Х. Алгоритмы. вводный курс. М.: Вильямс, 2015                                                     &   &  1  &  
  \\
  \hline
  
  \end{longtable}
  
\section{Перечень ресурсов информационно-телекоммуникационной сети «Интернет» (далее сеть-Интернет), необходимых для освоения дисциплины}
\begin{enumerate}
  
  \item Анисимов В.\,В. Криптографические методы защиты информации. Режим доступа:\\ https://sites.google.com/site/anisimovkhv/learning/kripto 
  
  \item Лапонина О.\,Р. Криптографические основы безопасности.  Режим доступа:\\ http://www.intuit.ru/studies/courses/28/28/info 
  
\end{enumerate}


\newpage
\section{Описание материально-технической базы, необходимой для осуществления образовательного процесса по дисциплине}
  
  
       Для проведения лекционных занятий требуется аудитория, оборудованная доской,  мультимедийным проектором с экраном. 
       Для проведения лабораторных занятий требуется компьютерный класс с подключением к интернету.
  
  


\section{Перечень информационных технологий, используемых при осуществлении образовательного процесса по дисциплине, включая перечень программного обеспечения
}

\subsection{Перечень информационных технологий, используемых при осуществлении образовательного процесса по дисциплине}

При осуществлении образовательного процесса по дисциплине используются следующие информационные технологии:
\begin{itemize}[nolistsep]
  
\item использование на занятиях электронных изданий (чтение лекций с использованием слайд-презентаций);
  
\item ведение учета посещаемости и выполнения учебных заданий в системе Google Docs;
  
\item разработка обучающимися программ на языках Python и Си++;
  
\item организация взаимодействия с обучающимися посредством электронной почты, специализированного образовательного форума Piazza;
  
\item компьютерное тестирование.
  
\end{itemize}

\subsection{Перечень программного обеспечения}
При осуществлении образовательного процесса по дисциплине используются следующее программное обеспечение:
\begin{itemize}[nolistsep]
  
\item язык Python версии 3 и новее;
  
\item среда разработки JetBrains PyCharm;
  
\item среда разработки Visual Studio;
  
\item интернет-браузер.
  
\end{itemize}



\newpage
\begin{center}
\section*{ЛИСТ АКТУАЛИЗАЦИИ РАБОЧЕЙ ПРОГРАММЫ ДИСЦИПЛИНЫ}
Б1.В.ОД.1.3\ --- Математические основы защиты информации 
\end{center}

  \noindent
  \begin{tabular}{|p{15mm}|p{67mm}|p{25mm}|p{41mm}|}
    \hline
    \small\centering
    Учебный год 
    & \small\centering
    Внесенные изменения 
    & \small\centering
    Преподаватель (ФИО) 
    & \small\centering\arraybackslash
    Протокол заседания выпускающей кафедры (дата, номер), ФИО зав.кафедрой, подпись \\
    & & & \\\hline
    & & & \\\hline
    & & & \\\hline
    & & & \\\hline
    & & & \\\hline
    & & & \\\hline
    & & & \\\hline
    & & & \\\hline
    & & & \\\hline
    & & & \\\hline
    & & & \\\hline
    & & & \\\hline
    & & & \\\hline
    & & & \\\hline
    & & & \\\hline
    & & & \\\hline
    & & & \\\hline
    & & & \\\hline
    & & & \\\hline
    & & & \\\hline
    & & & \\\hline
    & & & \\\hline
    & & & \\\hline
    & & & \\\hline
    & & & \\\hline
    & & & \\\hline
    & & & \\\hline
    & & & \\\hline
    & & & \\\hline
    & & & \\\hline
    & & & \\\hline
    & & & \\\hline
    & & & \\\hline
    & & & \\\hline
    & & & \\\hline
    & & & \\\hline
  \end{tabular}

  \medskip\noindent\textit{В таблице указывается только характер изменений (например, изменение темы, списка источников по~теме или темам, средств промежуточного контроля) с~указанием пунктов рабочей программы. Само содержание изменений оформляется приложением по~сквозной нумерации.}

\newpage\tableofcontents

\end{document}