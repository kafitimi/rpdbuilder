\documentclass[a4paper,12pt]{article}
\usepackage[hmargin={30mm, 17mm}, vmargin={20mm,20mm}, nohead]{geometry}
\usepackage{polyglossia, enumitem, array, multirow, tabularx, rotating, indentfirst, caption, float, ulem, titlesec, longtable }%<= additionalpackages =>}
\setdefaultlanguage{russian} % выбор основного языка (для переносов)
\defaultfontfeatures{Ligatures=TeX} % нужен для того, чтобы работали стандартные сочетания символов ---, -- << >> и т.п.

% шрифты 
\setmainfont{Times New Roman} 
\setmonofont[Scale=0.9]{Consolas}
\setsansfont{Trebuchet MS}

%{ if math=='1' %}
\usepackage[vargreek-shape=unicode]{unicode-math}
\setmathfont{latinmodern-math.otf}
\setmathfont[range=\mathit/{latin,Latin}]{Times New Roman-Italic}
\setmathfont[range=\mathit/{greek,Greek}]{Times New Roman}
\setmathfont[range=\mathup]{Times New Roman}
%{ endif %}

% pdf metadata
\usepackage[
  pdfencoding=unicode,
  pdftitle={Рабочая программа дисциплины <= discipline =>},
  pdfauthor={составлена <= author =>}, 
  bookmarksnumbered=true, 
    colorlinks=true, linktocpage=true, linkcolor=blue, pdfpagemode=UseOutlines]
{hyperref}


\titleformat*{\section}{\large\bfseries\centering}
\titleformat*{\subsection}{\bfseries}

\makeatletter

% заполняет ширину текста полем с подчеркиванием.
% #1 - текст слева, #2 - на подчеркивании, #3 - справа, #4 - подпись под подчеркнутым
\newcommand{\ulfield}[4]{
\noindent
\begin{tabularx}{\linewidth}{@{}l@{}X@{}l@{}}
#1\if\relax\detokenize{#1}\relax\else\,~\vrule width 0pt\fi 
& \uline{\vrule width 0pt\hfill#2\hfill\vrule width 0pt} & 
\if\relax\detokenize{#3}\relax\else\vrule width 0pt~\,\fi #3
\\
& {\scriptsize \vrule width 0pt\hfill#4\hfill\vrule width 0pt}
\end{tabularx}
}

\newcommand{\datefield}[1][]{
\if\relax\detokenize{#1}\relax%
«\uline{\hspace{22pt}}»~\uline{\hspace{90pt}}\,~20\uline{\hspace{20pt}}~г.\else
«\uline{\hspace{18pt}}»~\uline{\hspace{60pt}}\,~20\uline{\hspace{18pt}}~г.\fi
}

\setlist[enumerate,1]{nolistsep,labelindent=0pt,leftmargin=\parindent}
\setlist[itemize,1]{nolistsep,labelindent=0pt,leftmargin=*,label=--}

\newcommand\rotleft{\rotatebox{90}}

\captionsetup[table]{singlelinecheck=off, font=small, labelsep=period, textfont=it, format=hang, justification=raggedleft}


\newenvironment{contentstable}{
\begin{table}[H]
\caption{Структура и содержание дисциплины}
\begin{tabular}{|l|l|>{\raggedright}p{25ex}|*{5}{r|}p{8ex}|p{9ex}|}
\hline
\multicolumn{1}{|l|}     {~\raisebox{-12pt}[0pt][0pt]{\multirow{2}{*}{\rotleft{\raisebox{-4pt}[0pt][0pt]{№ п/п}}}}} &
\multicolumn{1}{l|}      {~\raisebox{-5pt}[0pt][0pt]{\multirow{2}{*}{\rotleft{Недели семестра}}}} &
\multicolumn{1}{c|}      {\multirow{2}{*}[-5ex]{\parbox{25ex}{\centering ~\linebreak Раздел дисциплины, \linebreak содержание}}} & 
\multicolumn{4}{p{17ex}|}{\centering Виды учебной работы, трудоем- кость (в~часах)} &
\multicolumn{1}{c|}      {~~\raisebox{18pt}[0pt][0pt]{\multirow{2}{*}{\rotleft{\parbox{25ex}{\centering КСР (в часах)}}}}} &
\multicolumn{1}{c|}      {\raisebox{25pt}[0pt]     [0pt]{\multirow{2}{*}{\rotleft{\parbox{28ex}{\centering Применяемые образова\-тельные (интер\-активные) технологии*}}}}} &
\multicolumn{1}{c|}      {~~\raisebox{12pt}{\multirow{2}{*}{\rotleft{\parbox{22ex}{\centering Форма текущего контроля}}}}} \\
\cline{4-7}
 & & &
~\raisebox{-2pt}{\rotleft{Лекций~} }& 
~\raisebox{-2pt}{\rotleft{Практич.~}} &
~\raisebox{-2pt}{\rotleft{Лаборат.~}} &
~\raisebox{-2pt}{\rotleft{СРС~}} &
 & & \\
\hline}{\\ \hline\end{tabular}\end{table}}

\newenvironment{srstable}{
\begin{table}[H]
    \caption{Содержание самостоятельной работы студентов}
    \begin{tabular}{|r|p{26ex}|l|r|r|p{9ex}|}
    \hline
    № & 
    \multicolumn{1}{p{37ex}|}{\centering Наименование раздела, темы} & 
    \multicolumn{1}{c|}{Виды} & 
    \multicolumn{1}{p{6ex}|}{\centering Объем часов} & 
    \multicolumn{1}{p{14ex}|}{\centering Используемые ресурсы} & 
    Контроль\\%
\hline}{\\ \hline\end{tabular}\end{table}}
    
\newenvironment{littable}{
    \begin{table}[H]
    \caption{Карта обеспеченности литературой}
    \begin{tabular}{|c|>{\raggedright}p{18.5em}|>{\raggedright}p{14ex}|>{\raggedleft}p{9ex}|p{9ex}|}
    \hline
    № & 
    \centering {\small Автор, название, место издания, издательство, 
    год издания учебной литературы, вид и~характеристика иных информационных ресурсов} &
    \centering {\small Наличие грифа, вид грифа} & 
    \centering {\small Кол-во экз. в библиотеке СВФУ} & 
    \centering {\small Кол-во экз. в библиотеке кафедры} \tabularnewline 
    \hline}{\\ \hline\end{tabular}\end{table}}

\newenvironment{internettable}{
    \begin{table}[H]
    \caption{Интернет-ресурсы}
    \begin{tabular}{|c|>{\raggedright}p{9em}|>{\raggedright}p{7.7em}|>{\raggedright}p{12ex}|p{24ex}|}
    \hline
    № & 
    \centering Наименование интернет-ресурса & 
    \centering Автор, разработчики & 
    \centering Формат документа & 
    \centering \raisebox{-8pt}{Ссылка (URL)} 
    \tabularnewline \hline}{\\ \hline\end{tabular}\end{table}}


%%% Перенос составных слов
\XeTeXinterchartokenstate=1
\XeTeXcharclass `\- 24
\XeTeXinterchartoks 24 0 = {\hskip\z@skip}
\XeTeXinterchartoks 0 24 = {\nobreak}

%%% Настройка содержания
\AtBeginDocument{%
\makeatletter 
\def\@tocline#1#2#3#4#5#6#7{\relax
  \ifnum #1>\c@tocdepth % then omit
  \else
    \par \addpenalty\@secpenalty\addvspace{#2}%
    \begingroup \hyphenpenalty\@M
    \@ifempty{#4}{%
      \@tempdima\csname r@tocindent\number#1\endcsname\relax
    }{%
      \@tempdima#4\relax
    }%
    \parindent\z@ \leftskip#3\relax \advance\leftskip\@tempdima\relax
    \rightskip\@pnumwidth plus4em \parfillskip-\@pnumwidth
    #5\leavevmode\hskip-\@tempdima
      \ifcase #1
       \or\or \hskip 1em \or \hskip 2em \else \hskip 3em \fi%
      #6\nobreak\relax
    \dotfill\hbox to\@pnumwidth{\@tocpagenum{#7}}\par
    \nobreak
    \endgroup
  \fi}
}%AtBeginDocument



\begin{document}

\thispagestyle{empty}

\noindent
\begin{center}
Министерство образования и науки Российской Федерации \\
Федеральное государственное автономное образовательное \\
учреждение высшего образования\\
«СЕВЕРО-ВОСТОЧНЫЙ ФЕДЕРАЛЬНЫЙ УНИВЕРСИТЕТ ИМЕНИ М.\,К.~АММОСОВА» \\
Институт математики и информатики \\
Кафедра информационных технологий

\vspace{12mm}
\begin{flushright}
\parbox{80mm}{
УТВЕРЖДАЮ\\
Директор ИМИ\\[2mm]
\ulfield{}{}{/\,В.\,И.~Афанасьева\,/}{}\\
\datefield
\\[20mm]
}
\end{flushright}


\textbf{РАБОЧАЯ ПРОГРАММА ДИСЦИПЛИНЫ}
\\[2mm]
\textbf{<= disciplinecode =>\ -- <= discipline =>} 
\\[5mm]

для программы <= level =>\\
по направлению подготовки \\
<= plancode => -- <= planname =>
\\[15mm]

\begin{tabular}{p{0.3\textwidth}p{0.3\textwidth}p{0.3\textwidth}}
  ОДОБРЕНО &  ОДОБРЕНО  & РЕКОМЕНДОВАНО \\
  Заведующий кафедрой \newline разработчика &
  Заведующий выпускающей кафедрой ИТ&
  Нормоконтроль в составе ОП пройден \\
  \ulfield{}{}{\uline{/\hspace{30mm}/}}{} &
  \ulfield{}{}{\uline{/\hspace{30mm}/}}{} &
  \ulfield{}{}{\uline{/\hspace{30mm}/}}{} \\
  Протокол № \uline{\hspace{13pt}} от\newline \datefield[small] & 
  Протокол № \uline{\hspace{13pt}} от\newline \datefield[small] & 
  Протокол № \uline{\hspace{13pt}} от\newline \datefield[small] 
\end{tabular}
\par\vfill\vspace{6mm}
Якутск -- <= year =>

\end{center}


\newpage


\begin{center}
\section{АННОТАЦИЯ}
{\bf к рабочей программе дисциплины\\
<= disciplinecode =>\ -- <=discipline =>} \\
Трудоемкость \uline{~<= zet =>~} з.~е.
\end{center}


\subsection{Цель освоения и краткое содержание дисциплины}
Целью освоения дициплины <<<= discipline =>>> является


\subsection{Перечень планируемых результатов обучения по дисциплине, соотнесенных с планируемыми результатами освоения образовательной программы.}
\begin{longtable}{|p{8cm}|p{8cm}|}
\caption{Перечень планируемых результатов обучения}\\
\hline
\centering
Планируемые результаты освоения программы (содержание и коды компетенций) & 
\centering\arraybackslash
Планируемые результаты обучения по дисциплине
\\
\hline
%{ for code in Компетенцииs %}
<= code => : <= Компетенцииs[code] =>%{ if not loop.last %}, %{ endif %}
%{endfor %}
& \\
\hline
\end{longtable}


\subsection{Место дисциплины в~структуре образовательной программы}

  \begin{table}[H]
  \setlength\arraycolsep{3pt}
  \caption{Содержательно-логические связи дисциплины}
  \begin{tabular}{|l|p{18ex}|*{2}{p{23ex}|}}
  \hline
  \multicolumn{1}{|c|}{\multirow{2}{13ex}{\centering Индекс \linebreak дисциплины}} &
  \multicolumn{1}{c|}{\multirow{2}{18ex}{\centering Наименование \linebreak дисциплины}} & 
  \multicolumn{2}{p{46ex}|}{\centering Коды учебных дисциплин, практик} \\
  \cline{3-4}
   & & 
  \centering на которые опирается содержание дисциплины & 
  \centering\arraybackslash для которых содержание дисциплины выступает опорой
  \\ \hline
  <=disciplinecode=> & <= discipline => & <= basedon => & <= baseof =>
  \\ \hline
  \end{tabular}
  \end{table}

\subsection{Язык преподавания} 

  Русский.

\newpage

\section{Объем дисциплины в зачетных единицах с указанием количества академических часов, выделенных на контактную работу обучающихся с преподавателем (по~видам учебных занятий) и~на~самостоятельную работу обучающихся}

\begin{table}[H]
\caption{Выписка из учебного плана} 
\begin{tabular}{|p{9cm}|c|c|}
\hline
Код и название дисциплины по учебному плану & \multicolumn{2}{p{6cm}|}{<= disciplinecode =>\ -- <=discipline => }\\
\hline
Курс изучения &\multicolumn{2}{c|}{ <= Курсs => }\\
\hline
Семестр(ы) изучения &\multicolumn{2}{c|}{ <= Семs => }\\
\hline
Форма промежуточной аттестации (зачет/экзамен) &\multicolumn{2}{c|}{ <= формаконтроляs => }\\
\hline
Курсовой проект / курсовая работа (указать вид работы при наличии в учебном плане), семестр выполнения &\multicolumn{2}{c|}{ }\\
\hline
Трудоемкость (в ЗЕТ) &\multicolumn{2}{c|}{ <= ЗЕТs => (<= zet =>) }\\
\hline
{\bf Трудоемкость (в часах)} (сумма строк №1,2,3), в~т.ч.:& \multicolumn{2}{c|}{<= ВсегоЧасs =>}\\
\hline
№\,1. Контактная работа обучающихся с преподавателем (КР), в часах:
& \multicolumn{1}{p{3cm}|}{\centering Объем аудиторной работы, в часаx}
& \multicolumn{1}{p{3cm}|}{\centering\arraybackslash В~т.\,ч. с~применением ДОТ или ЭО, в~часах}\\
\hline  
Объем работы (в часах) (1.1.+1.2.+1.3.)& <= ВсегоАудЧасs => & \\
\hline
1.1. Занятия лекционного типа (лекции) & <= Лекs => & \\
\hline
1.2. Занятия семинарского типа, всего, в т.ч.: & & \\
\hline
- семинары (практические занятия, коллоквиумы и~т.~п.)  & <= Прs => & \\
\hline
- лабораторные работы& <= Лабs => & \\
\hline
- практикумы & & \\
\hline
1.3. КСР (контроль самостоятельной работы, консультации)& <= КСРs => & \\
\hline
{\bf №\,2. Самостоятельная работа обучающихся (СРС) (в часах)}& \multicolumn{2}{c|}{<= СРСs =>}\\
\hline
{\bf №\,3. Количество часов на экзамен (при наличии экзамена в учебном плане)}& \multicolumn{2}{c|}{<= ЧасЭкзs =>}\\
\hline
\end{tabular}
\end{table}

\newpage
\section{Содержание дисциплины, структурированное по~темам с~указанием отведенного на~них количества академических часов и~видов учебных занятий}
\subsection{Распределение часов по~темам и~видам учебных занятий}
\begin{longtable}{|p{59mm}|c|c|c|c|c|c|c|c|c|c|c|}
\caption{}\\
\hline
 & 
     & 
\multicolumn{9}{c|}{Контактная работа, в часах} & 
    \\
\cline{3-11} 
\raisebox{18mm}{Тема}&
\rotleft{Всего часов} &
\rotleft{Лекции} &
\rotleft{из них с прим-м  ЭО и ДОТ} &
\rotleft{\parbox{5cm}{\raggedright\arraybackslash Семинары  (практические занятия, коллоквиумы)}} &
\rotleft{из них с прим-м  ЭО и ДОТ} &
\rotleft{Лабораторные работы} &
\rotleft{из них с прим-м  ЭО и ДОТ} &
\rotleft{Практикумы} &
\rotleft{из них с прим-м  ЭО и ДОТ} &
\rotleft{КСР (консультации)} & 
\rotleft{Часы СРС}\\
\hline
a & 28 & 28 & 28 & 28 & 28 & 28 & 28 & 28 & 28 & 28 & 28 
\\
\hline
\end{longtable}



\subsection{Содержание тем программы дисциплины} 


\section{Перечень учебно-методического обеспечения для самостоятельной работы обучающихся по дисциплине}
\begin{longtable}{|l|p{40mm}|p{54mm}|c|p{30mm}|}
\hline
№ & \centering Наименование раздела (темы) дисциплины & 
\centering Вид СРС & \multicolumn{1}{p{14mm}|}{\centering Трудо\-емкость (в часах)} & \centering\arraybackslash Формы и методы контроля\\
\hline
\end{longtable}


\section{Методические указания для обучающихся по освоению дисциплины}

\section{Фонд оценочных средств для проведения промежуточной аттестации обучающихся по дисциплине}

\section{Перечень основной и дополнительной учебной литературы, необходимой для освоения дисциплины}

\begin{longtable}{|l|p{7cm}|c|c|p{32mm}|}
\hline
№ & 
Автор, название, место издания, издательство, год издания учебной литературы, вид и характеристика иных информационных ресурсов &
\multicolumn{1}{p{25mm}|}{Наличие грифа, вид грифа} &
\multicolumn{1}{p{25mm}|}{НБ СВФУ, кафедральная библиотека и кол-во экземпляров} & 
\multicolumn{1}{p{25mm}|}{Электронные издания: точка доступа к ресурсу (наименование ЭБС, ЭБ СВФУ)}\\
\hline
\multicolumn{5}{|c|}{Основная литература}\\

\hline
<= forloop.index => & <= lit.name => & <= list.grif => & <= list.grif => & <= list.quantity => & <= list.url =>
\\
\hline

\multicolumn{5}{|c|}{Основная литература}\\
\hline

\hline
<= forloop.index => & <= lit.name => & <= list.grif => & <= list.grif => & <= list.quantity => & <= list.url =>
\\
\hline

\end{longtable}

\section{Перечень ресурсов информационно-телекоммуникационной сети «Интернет» (далее сеть-Интернет), необходимых для освоения дисциплины}

\section{Перечень информационных технологий, используемых при осуществлении образовательного процесса по дисциплине, включая перечень программного обеспечения и информационных справочных систем (при необходимости)}

\section{Описание материально-технической базы, необходимой для осуществления образовательного процесса по дисциплине}


\subsection{Перечень информационных технологий, используемых при осуществлении образовательного процесса по дисциплине}
При осуществлении образовательного процесса по дисциплине используются следующие информационные технологии:
\begin{itemize}[nolistsep]
\item <= tech =>
\end{itemize}

\subsection{Перечень программного обеспечения}
\begin{itemize}[nolistsep]
\item <= software =>
\end{itemize}



\newpage
\begin{center}
\section*{ЛИСТ АКТУАЛИЗАЦИИ РАБОЧЕЙ ПРОГРАММЫ ДИСЦИПЛИНЫ}
<= disciplinecode =>\ --- <= discipline => 
\end{center}

  \noindent
  \begin{tabular}{|p{15mm}|p{67mm}|p{25mm}|p{41mm}|}
    \hline
    \small\centering
    Учебный год 
    & \small\centering
    Внесенные изменения 
    & \small\centering
    Преподаватель (ФИО) 
    & \small\centering\arraybackslash
    Протокол заседания выпускающей кафедры (дата, номер), ФИО зав.кафедрой, подпись \\
    & & & \\\hline
    & & & \\\hline
    & & & \\\hline
    & & & \\\hline
    & & & \\\hline
    & & & \\\hline
    & & & \\\hline
    & & & \\\hline
    & & & \\\hline
    & & & \\\hline
    & & & \\\hline
    & & & \\\hline
    & & & \\\hline
    & & & \\\hline
    & & & \\\hline
    & & & \\\hline
    & & & \\\hline
    & & & \\\hline
    & & & \\\hline
    & & & \\\hline
    & & & \\\hline
    & & & \\\hline
    & & & \\\hline
    & & & \\\hline
    & & & \\\hline
    & & & \\\hline
    & & & \\\hline
    & & & \\\hline
    & & & \\\hline
    & & & \\\hline
    & & & \\\hline
    & & & \\\hline
    & & & \\\hline
    & & & \\\hline
    & & & \\\hline
    & & & \\\hline
  \end{tabular}

  \medskip\noindent\textit{В таблице указывается только характер изменений (например, изменение темы, списка источников по~теме или темам, средств промежуточного контроля) с~указанием пунктов рабочей программы. Само содержание изменений оформляется приложением по~сквозной нумерации.}

\newpage\tableofcontents

\end{document}
