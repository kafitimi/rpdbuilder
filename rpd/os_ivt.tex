\documentclass[a4paper,12pt]{article}
  \usepackage[hmargin={30mm, 17mm}, vmargin={20mm,20mm}, nohead]{geometry}
  \usepackage{polyglossia, enumitem, array, longtable, multirow, tabularx, rotating, indentfirst, caption, float, ulem, titlesec }%}
  \setdefaultlanguage{russian} % выбор основного языка (для переносов)
  % шрифты 
  \setmainfont{Times New Roman} 
  \setmonofont[Scale=0.9]{Consolas}
  \setsansfont{Trebuchet MS}
  \defaultfontfeatures{Ligatures=TeX} % нужен для того, чтобы работали стандартные сочетания символов ---, -- << >> и т.п.

  

  % pdf metadata
  \usepackage[
    pdfencoding=unicode,
    pdftitle={Рабочая программа дисциплины Операционные системы},
    pdfauthor={составлена }, 
    bookmarksnumbered=true, 
      colorlinks=true, linktocpage=true, linkcolor=blue, pdfpagemode=UseOutlines]
  {hyperref}

  % формат заголовков, подписей и списков
  \titleformat*{\section}{\large\bfseries\centering}
  \titleformat*{\subsection}{\bfseries}
  \captionsetup[table]{singlelinecheck=off, font=small, labelsep=period, textfont=it, format=hang, justification=raggedleft}
  \setlist[enumerate,1]{nolistsep,labelindent=0pt,leftmargin=\parindent}
  \setlist[itemize,1]{nolistsep,labelindent=20pt,leftmargin=*,label=--}

  \makeatletter

  % заполняет ширину текста полем с подчеркиванием.
  % #1 - текст слева, #2 - на подчеркивании, #3 - справа
  \newcommand{\ulfield}[3]{
  \noindent
  \begin{tabularx}{\linewidth}{@{}l@{}X@{}l@{}}
  #1\if\relax\detokenize{#1}\relax\else\,~\vrule width 0pt\fi 
  & \uline{\vrule width 0pt\hfill#2\hfill\vrule width 0pt} & 
  \if\relax\detokenize{#3}\relax\else\vrule width 0pt~\,\fi #3
  \end{tabularx}
  }

  %%% Перенос составных слов
    \XeTeXinterchartokenstate=1
    \XeTeXcharclass `\- 24
    \XeTeXinterchartoks 24 0 = {\hskip\z@skip}
    \XeTeXinterchartoks 0 24 = {\nobreak}

  \newcommand\rotleft{\rotatebox{90}}
  
  \makeatother

  \newcommand{\datefield}[1][]{\if
  \relax\detokenize{#1}\relax
  «\uline{\hspace{22pt}}»~\uline{\hspace{90pt}}\,~20\uline{\hspace{20pt}}~г.\else 
  «\uline{\hspace{18pt}}»~\uline{\hspace{60pt}}\,~20\uline{\hspace{18pt}}~г.\fi
  }


%%% Настройка содержания
\AtBeginDocument{
  \makeatletter 
  \def\@tocline#1#2#3#4#5#6#7{\relax
  \ifnum #1>\c@tocdepth % then omit
  \else
    \par \addpenalty\@secpenalty\addvspace{#2}%
    \begingroup \hyphenpenalty\@M
    \@ifempty{#4}{\@tempdima\csname r@tocindent\number#1\endcsname\relax}{\@tempdima#4\relax}%
    \parindent\z@ \leftskip#3\relax \advance\leftskip\@tempdima\relax
    \rightskip\@pnumwidth plus4em \parfillskip-\@pnumwidth
    #5\leavevmode\hskip-\@tempdima
      \ifcase #1
       \or\or \hskip 1em \or \hskip 2em \else \hskip 3em \fi%
      #6\nobreak\relax
    \dotfill\hbox to\@pnumwidth{\@tocpagenum{#7}}\par
    \nobreak
    \endgroup
  \fi}
  \makeatother
}%AtBeginDocument

\begin{document}
\sloppy
\thispagestyle{empty}

\noindent
\begin{center}
Министерство образования и науки Российской Федерации \\
Федеральное государственное автономное образовательное \\
учреждение высшего образования\\
«СЕВЕРО-ВОСТОЧНЫЙ ФЕДЕРАЛЬНЫЙ УНИВЕРСИТЕТ \\
имени М.\,К.~АММОСОВА» \\
Институт математики и информатики \\
Кафедра информационных технологий

\vspace{12mm}
\begin{flushright}
\parbox{80mm}{
УТВЕРЖДАЮ\\
Директор ИМИ\\[2mm]
\ulfield{}{}{/\,В.\,И.~Афанасьева\,/}{}\\[1mm]
\datefield
\\[20mm]
}
\end{flushright}


РАБОЧАЯ ПРОГРАММА ДИСЦИПЛИНЫ
\\[2mm]
\textbf{Б1.Б.23\ -- Операционные системы} 
\\[5mm]

для программы бакалавриата\\
по направлению подготовки \\
09.03.01 -- Информатика и вычислительная техника
\\[15mm]


\parbox{\textwidth}{
 Автор: Павлов Александр Викторович, к.~ф.-м.~н., ---, доцент кафедры информационных технологий ИМИ СВФУ

}
\bigskip


\begin{tabular}{|p{0.3\textwidth}|p{0.3\textwidth}|p{0.3\textwidth}|}
  \hline
  ОДОБРЕНО &  ОДОБРЕНО  & РЕКОМЕНДОВАНО \\
  Заведующий кафедрой \newline разработчика &
  Заведующий выпускающей кафедрой ИТ&
  Нормоконтроль в составе ОП пройден \\
  \ulfield{}{}{\uline{/\hspace{30mm}/}} &
  \ulfield{}{}{\uline{/\hspace{30mm}/}} &
  \ulfield{}{}{\uline{/\hspace{30mm}/}} \\
  Протокол № \uline{\hspace{13pt}} от\newline \datefield[small] & 
  Протокол № \uline{\hspace{13pt}} от\newline \datefield[small] 
  & 
  Протокол № \uline{\hspace{13pt}} от\newline \datefield[small] \\
  \hline
  \multicolumn{2}{|p{0.625\textwidth}|}{Рекомендовано к утверждению в составе ОП\newline 
  09.03.01 «Информатика и вычислительная техника»\newline
  Председатель УМК ИМИ \uline{\hspace{21mm}} \mbox{/И.\,В.\, Николаева/}\newline 
  Протокол УМК № \uline{\hspace{12mm}} от \datefield[small]}& 
  Эксперт УМК ИМИ\newline
  \ulfield{}{}{\uline{/\hspace{30mm}/}}\newline\datefield[small]
  \\
  \hline
\end{tabular}
\par\vfill\vspace{6mm}
Якутск -- 2016

\end{center}


\newpage


\begin{center}
\section{АННОТАЦИЯ}
  {\bf к рабочей программе дисциплины\\
  Б1.Б.23\ -- Операционные системы} \\
  Трудоемкость \uline{~6~} з.~е.
\end{center}


\subsection{Цель освоения и краткое содержание дисциплины}
  
  Целью изучения диcциплины <<Операционные системы>> является: изучение принципов работы и архитектуры современных операционных систем (ОС).
  
  
  \textit{Краткое содержание дисциплины.}  Обзор основных понятий и истории развития ОС. Напоминания из курса организации ЭВМ.
   Методы структурирования ОС. Процессы и параллелелизм, структуры данных ОС. Планирование и диспетчеризация.
   Управление памятью. Проблемы безопасности. Виртуализация. Основы организации сетевого взаимодействия.
  
  
  



\subsection{Перечень планируемых результатов обучения по дисциплине, соотнесенных с~планируемыми результатами освоения образовательной программы}

\begin{longtable}{|p{54mm}|p{100mm}|}
  \caption{Перечень планируемых результатов обучения}\\
  \hline
  \centering
  Планируемые результаты освоения программы (содержание и коды компетенций) & 
  \centering\arraybackslash
  Планируемые результаты обучения по~дисциплине
  \\
  \hline
  \endhead
  
  ОПК-1 : способностью инсталлировать программное и аппаратное обеспечение для информационных и автоматизированных систем, \par 
  
  ОПК-4 : способностью участвовать в настройке и наладке программно-аппаратных комплексов, \par 
  
  ПК-2 : способностью разрабатывать компоненты аппаратно-программных комплексов и баз данных, используя современные инструментальные средства и технологии программирования, \par 
  
  ПК-3 : способностью обосновывать принимаемые проектные решения, осуществлять постановку и выполнять эксперименты по проверке их корректности и эффективности
  & 
  В результате изучения дисциплины обучающийся должен:\newline
  \emph{знать:}
  \begin{itemize}[leftmargin=12pt]
    \item этапы эволюции операционных систем; 
    \item способы представления основных видов данных в памяти ЭВМ, понятия стекового кадра, соглашения о вызовах функций, статической и динамической компоновки; 
    \item назначение, основные принципы организации и функционирования современных ОС; 
  \end{itemize}
  

  \emph{уметь:}
  \begin{itemize}[leftmargin=12pt]
    \item писать небольшие программы, использующие специфичные для данной ОС системные сервисы; 
    \item пользоваться документацией функций ОС для корректного использования ее сервисов в собственных программах; 
    \item устанавливать не менее двух различных современных ОС; 
  \end{itemize}
  

  \emph{владеть навыками:}
  чтения, компиляции и отладки программ, использующих специфичные для данной ОС системные сервисы.
  
  \\
  \hline
  \end{longtable}


\subsection{Место дисциплины в~структуре образовательной программы}

  \begin{table}[H]
  \setlength\arraycolsep{3pt}
  \caption{Содержательно-логические связи дисциплины}
  \begin{tabular}{|l|p{18ex}|*{2}{p{23ex}|}}
  \hline
  \multicolumn{1}{|c|}{\multirow{2}{13ex}{\centering Индекс \linebreak дисциплины}} &
  \multicolumn{1}{c|}{\multirow{2}{18ex}{\centering Наименование \linebreak дисциплины}} & 
  \multicolumn{2}{p{46ex}|}{\centering Коды учебных дисциплин, практик} \\
  \cline{3-4}
   & & 
  \centering на которые опирается содержание дисциплины & 
  \centering\arraybackslash для которых содержание дисциплины выступает опорой
  \\ \hline
  Б1.Б.23 & Операционные системы 
  & 
  \raggedright
  
  Б1.В.ОД.3~--- Программирование 
  & 
  \raggedright\arraybackslash
  
  Б1.В.ДВ.4.1~---Администрирование ОС Windows,
  Б1.В.ДВ.6.1~---Администрирование ОС Linux 
  \\ \hline
  \end{tabular}
  \end{table}


\subsection{Язык преподавания} 
  Русский.
  



\newpage

\section{Объем дисциплины в зачетных единицах с указанием количества академических часов, выделенных на контактную работу обучающихся с преподавателем (по~видам учебных занятий) и~на~самостоятельную работу обучающихся}

\begin{table}[H]
\caption{Выписка из учебного плана} 
\begin{tabular}{|p{9cm}|c|c|}
\hline
Код и название дисциплины по учебному плану & \multicolumn{2}{p{6cm}|}{Б1.Б.23\ -- Операционные системы }\\
\hline
Курс изучения &\multicolumn{2}{c|}{ 2 }\\
\hline
Семестр(ы) изучения &\multicolumn{2}{c|}{ 3, 4 }\\
\hline
Форма промежуточной аттестации (зачет/экзамен) &\multicolumn{2}{c|}{ зачет / экзамен }\\
\hline
Курсовой проект / курсовая работа (указать вид работы при наличии в учебном плане), семестр выполнения &\multicolumn{2}{c|}{ }\\
\hline
Трудоемкость (в ЗЕТ) &\multicolumn{2}{c|}{ 2 / 4 (6) }\\
\hline
{\bf Трудоемкость (в часах)} (сумма строк №1, 2, 3), в~т.~ч.:& \multicolumn{2}{c|}{72 / 144}\\
\hline
\textbf{№\,1. Контактная работа обучающихся с преподавателем (КР),} в часах:
& \multicolumn{1}{p{3cm}|}{\centering Объем аудиторной работы, в часаx}
& \multicolumn{1}{p{3cm}|}{\centering\arraybackslash В~т.\,ч. с~применением ДОТ или ЭО, в~часах}\\
\hline  
Объем работы (в часах) (1.1.+1.2.+1.3.)& 53 / 59 & \\
\hline
1.1. Занятия лекционного типа (лекции) & 17 / 18 & \\
\hline
1.2. Занятия семинарского типа, всего, в т.ч.: & & \\
\hline
- семинары (практические занятия, коллоквиумы и~т.~п.)  & – / – & \\
\hline
- лабораторные работы& 34 / 36 & \\
\hline
- практикумы & & \\
\hline
1.3. КСР (контроль самостоятельной работы, консультации)& 2 / 5 & \\
\hline
{\bf №\,2. Самостоятельная работа обучающихся (СРС) (в часах)}& \multicolumn{2}{c|}{19 / 49}\\
\hline
{\bf №\,3. Количество часов на экзамен (при наличии экзамена в учебном плане)}& \multicolumn{2}{c|}{– / 36}\\
\hline
\end{tabular}
\end{table}



\newpage
\section{Содержание дисциплины, структурированное по~темам с~указанием отведенного на~них количества академических часов и~видов учебных занятий}
\subsection{Распределение часов по~темам и~видам учебных занятий}
\begin{longtable}{|>{\raggedright\arraybackslash}p{59mm}|c|c|c|c|c|c|c|c|c|c|c|}
\caption{}
\\
\hline
 & & 
\multicolumn{9}{c|}{Контактная работа, в часах} & 
\\
\cline{3-11} 
\raisebox{18mm}{Тема}&
\rotleft{Всего часов} &
\rotleft{Лекции} &
\rotleft{из них с прим-м ЭО и ДОТ} &
\rotleft{\parbox{5cm}{\raggedright\arraybackslash Семинары  (практические занятия, коллоквиумы)}} &
\rotleft{из них с прим-м ЭО и ДОТ} &
\rotleft{Лабораторные работы} &
\rotleft{из них с прим-м ЭО и ДОТ} &
\rotleft{Практикумы} &
\rotleft{из них с прим-м ЭО и ДОТ} &
\rotleft{КСР (консультации)} & 
\rotleft{Часы СРС}
\\
\hline
Тема 1. Обзор основных понятий курса. 		   & 29 & 6 & 0 & 0 & 0 & 12 & 0 & 0 & 0 & 0 & 11 \\ 
\hline
Тема 2. Напоминания из курса организации ЭВМ.  & 30 & 6 & 0 & 0 & 0 & 12 & 0 & 0 & 0 & 1 & 11 \\ 
\hline
Тема 3. Вызов функций и компоновка программ.   & 26 & 5 & 0 & 0 & 0 & 10 & 0 & 0 & 0 & 1 & 10 \\ 
\hline
Тема 4. Процессы и параллелизм. 			   & 31 & 6 & 0 & 0 & 0 & 12 & 0 & 0 & 0 & 2 & 11 \\ 
\hline
Тема 5. Управление памятью.                    & 30 & 6 & 0 & 0 & 0 & 12 & 0 & 0 & 0 & 2 & 10 \\ 
\hline
Тема 6. Ассорти: Командная строка Linux. Файловые системы. Безопасность. Виртуализация. Сеть.  & 34 & 6 & 0 & 0 & 0 & 12 & 0 & 0 & 0 & 1 & 15 \\ 
\hline
ВСЕГО ЧАСОВ & 180 & 35 & 0 & 0 & 0 & 70 & 0 & 0 & 0 & 7 & 68 \\ 

\hline
\end{longtable}

\subsection{Содержание тем программы дисциплины} 


\textbf{Тема 1. Обзор основных понятий курса. 		  }\\
Назначение ОС. История ОС. Функции современной ОС. Принципы организации ОС,
микроядерные и монолитные архитектуры. Процессы. Режим ядра и режим пользователя.
Прерывания. Программные интерфейсы, предоставляемые операционными системами.

\textbf{Тема 2. Напоминания из курса организации ЭВМ. }\\
Представление данных в памяти машины: целочисленные типы, числа с плавающей
точкой IEEE 754, представление текста, кодовые страницы, Unicode и UTF-8. Устройство
процессора, CISC и RISC. Архитектура x86. Работа с памятью, режимы адресации.
Машинный код и ассемблер. Стек.

\textbf{Тема 3. Вызов функций и компоновка программ.  }\\
Функции на Си. Стековые кадры и соглашения передачи. Внешние объекты в Си-файлах:
объявления и определения. Объектные файлы. Компоновка (связывание). Статическое
и динамическое связывание.

\textbf{Тема 4. Процессы и параллелизм. 			  }\\
Процессы и потоки, их состояния, структуры данных ОС для управления ими. Потоки в ядре
и в пользовательском пространстве. Межпроцессное взаимодействие. Механизмы взаимного
исключения/синхронизации. Классические проблемы межпроцессного взаимодействия.
Блокирующийся и неблокирующийся ввод/вывод. Тупики.
Алгоритмы планирования FIFO, RR, циклическое с приоритетами, лотерейное.

\textbf{Тема 5. Управление памятью.                   }\\
Сегментная и страничная организация памяти, виртуальная память. Кэширование и
его побочные эффекты. Структуры ОС для управления памятью. Алгоритмы замещения страниц.

\textbf{Тема 6. Ассорти: Командная строка Linux. Файловые системы. Безопасность. Виртуализация. Сеть. }\\
Командная строка Linux. Текстовые фильтры и перенаправление ввода-вывода.
Файловая система FAT32. Файловая система ext2. Проблемы безопасности. Управление доступом,
права и привилегии, системная политика Windows. Пакетные менеджеры и службы обновления.
Виртуализация, паравиртуализация. Сетевые сервисы ОС. Протоколы удаленного доступа.
 

\subsection{Формы и методы проведения занятий, применяемые учебные технологии}
При проведении занятий и организации СРС используются традиционные технологии сообщающего обучения, предполагающие передачу информации в~готовом виде: проведение лекционных занятий, самостоятельная работа с~источниками. Предусмотрено использование активных и интерактивных форм обучения с целью формирования и развития профессиональных навыков студентов~--- выполнение лабораторных работ, подразумевающих применение компьютерных технологий. 



\section{Перечень учебно-методического обеспечения для самостоятельной работы обучающихся по дисциплине}
\begin{longtable}{|l|>{\raggedright\arraybackslash}p{40mm}|>{\raggedright\arraybackslash}p{54mm}|c|>{\raggedright\arraybackslash}p{30mm}|}
\hline
№ & \centering Наименование раздела (темы) дисциплины & 
\centering Вид СРС & \multicolumn{1}{p{14mm}|}{\centering Трудо\-емкость (ч)} & \centering\arraybackslash Формы и методы контроля\\
\hline
1 & Обзор основных понятий курса. 		  & Завершение лаб. работ & 11 & Сдача отчета по л.р. \\ 
\hline
2 & Напоминания из курса организации ЭВМ.   & Завершение лаб. работ & 11 & Сдача отчета по л.р. \\ 
\hline
3 & Вызов функций и компоновка программ.    & Завершение лаб. работ & 10 & Сдача отчета по л.р. \\ 
\hline
4 & Процессы и параллелизм. 			      & Завершение лаб. работ & 11 & Сдача отчета по л.р. \\ 
\hline
5 & Управление памятью.                     & Завершение лаб. работ & 10 & Сдача отчета по л.р. \\ 
\hline
6 & Ассорти: Командная строка Linux. Файловые системы. Безопасность. Виртуализация. Сеть.  & Завершение лаб. работ, реферат & 15 & Сдача отчета по л.р., защита реферата. \\ 
\hline
 & ИТОГО                                   &                       & 68 &  \\ 

\hline
\end{longtable}


\section{Методические указания для обучающихся по освоению дисциплины}
Ключевым видом работы студента в данном курсе является самостоятельное написание программ.
Лабораторные работы, как правило, составляются так, что за одно занятие полностью выполнить
все задания трудно. Предполагается самостоятельное завершение заданий, составление
и отправка отчета по лабораторной работе. Также в конце второго семестра изучения
дисциплины предлагается написание и защита реферата по выбранной теме.
\par
Чтение документации, исходного кода таких сложных исторически сложившихся программных
комплексов, как операционные системы, должно развить и закрепить умения и навыки, имеющие
важное значение для профессиональной деятельности.



\subsubsection*{Рейтинговый регламент по дисциплине}
\begin{longtable}{|>{\raggedright\arraybackslash}p{110mm}|r|r|}
\hline
\centering\arraybackslash Вид выполняемой учебной работы (контролирующие мероприятия) & 
\multicolumn{1}{p{20mm}|}{\centering\arraybackslash Количество баллов (min)} & 
\multicolumn{1}{p{20mm}|}{\centering\arraybackslash Количество баллов (max)} \\
\hline
Посещение занятий   & 6  & 10 \\ 
\hline
Лабораторные работы & 33 & 55 \\ 
\hline
Контрольные тесты   & 21 & 35 \\ 
\hline
\bf Кол-во баллов для зачета в 1 сем. (min--max) & \bf 60 & \bf 100 \\ 
\hline
Посещение занятий   & 6  & 10 \\ 
\hline
Лабораторные работы & 25 & 32 \\ 
\hline
Контрольные тесты   & 14 & 18 \\ 
\hline
Защита реферата     &  0 & 10 \\ 
\hline
\bf Кол-во баллов для допуска к экзамену вo 2 сем. (min--max) & \bf 45 & \bf 70 \\ 

\hline
\end{longtable}

\newpage
\section{Фонд оценочных средств для проведения промежуточной аттестации обучающихся по дисциплине}

\subsection{Показатели, критерии и шкала оценивания}

\begin{longtable}{|p{15mm}|p{48mm}|p{16mm}|p{48mm}|p{14mm}|}
\hline
  \centering\small Коды оцениваемых компетенций
& \centering Показатель оценивания (дескриптор) (по п.1.2) 
& \centering\small Уровни освоения 
& \centering Критерий оценивания 
& \centering\small\arraybackslash Оценка
\\
\hline


ПК-2, ПК-3 & знать этапы эволюции операционных систем;
знать способы представления основных видов данных в памяти ЭВМ, понятия стекового кадра, соглашения о вызовах функций, статической и динамической компоновки;
знать назначение, основные принципы организации и функционирования современных ОС; & освоено & может сравнить оригинальный Unix, Linux, MS-DOS и Windows ветки NT,
назвав их главные отличия и приблизительно даты выпуска;
может объяснить работу стековых кадров при рекурсивном вызове функции в Си;
может объяснить понятие соглашения о вызовах функции;
может объяснить различие между режимом ядра и режимом пользователя;
может объяснить причины использования и механизм работы прерываний; & зачтено (1-й семестр)
\\
\hline

ПК-2, ПК-3 & знать этапы эволюции операционных систем;
знать способы представления основных видов данных в памяти ЭВМ, понятия стекового кадра, соглашения о вызовах функций, статической и динамической компоновки;
знать назначение, основные принципы организации и функционирования современных ОС; & не освоено & не может обоснованно сравнить Linux, MS-DOS и Windows ветки NT;
не может объяснить работу стековых кадров при рекурсивном вызове функции в Си;
не может объяснить понятие соглашения о вызовах функции;
не может объяснить различие между режимом ядра и режимом пользователя;
не может объяснить причины использования и механизм работы прерываний; & не зачтено (1-й семестр)
\\
\hline

ОПК-1, ОПК-4, ПК-2, ПК-3 & знать назначение, основные принципы организации и функционирования современных ОС;
знать назначение, архитектуру и состав операционных систем;
знать основные принципы организации и функционирования современных ОС;
уметь писать небольшие программы, использующие специфичные для данной ОС системные сервисы;
уметь пользоваться документацией функций операционной системы для корректного использования ее сервисов в собственных программах;
уметь устанавливать не менее двух различных современных ОС;
владеть навыками чтения, компиляции и отладки программ, использующих специфичные для данной ОС системные сервисы. & высокий & может объяснить механизм вытесняющей многозадачности;
может сравнить механизмы вызова системных сервисов в Windows и Linux;
может правильно вызывать функции Windows API/unistd.h, для работы с файлами и процессами в Windows/Linux, имея документацию MSDN/man;
может сравнивать и противопоставлять различные подходы к планированию, выделению памяти, замещению страниц, организации файловой системы;
может объяснить принцип страничной организации памяти;
может по псевдокоду правильно определить один из классических алгоритмов замещения страниц или планирования;
может организовать выделение раздела с заданными характеристиками при установке операционных систем Windows и Linux;
может описать проблемы, возникающие при совместном параллельном доступе к данным и обосновать необходимость механизмов синхронизации;
может правильно описать структуру метаданных FAT32 для файла с заданными характеристиками;
может описать преимущества, предоставляемые виртуализацией, и сравнивать различные механизмы виртуализации. & отлично (2-й сем.)
\\
\hline

ОПК-1, ОПК-4, ПК-2, ПК-3 & знать назначение, основные принципы организации и функционирования современных ОС;
знать назначение, архитектуру и состав операционных систем;
знать основные принципы организации и функционирования современных ОС;
уметь писать небольшие программы, использующие специфичные для данной ОС системные сервисы;
уметь пользоваться документацией функций операционной системы для корректного использования ее сервисов в собственных программах;
уметь устанавливать не менее двух различных современных ОС;
владеть навыками чтения, компиляции и отладки программ, использующих специфичные для данной ОС системные сервисы. & базовый & может объяснить основные цели операционных систем;
может объяснить разницу с точки зрения прикладного программиста между вытесняющей и кооперативной многозадачностью;
может объяснить разницу между режимом ядра и режимом пользователя;
может объяснять число и тип параметров в вызовах Windows API или функций, соответствующих системным вызовам Linux, используя справку;
может объяснить принцип страничной организации памяти;
может описать хотя бы три алгоритма замещения страниц;
может описать хотя бы три алгоритма планирования;
может сравнивать и противопоставлять различные подходы к организации хранения файлов;
может описать преимущества, предоставляемые виртуализацией. & хорошо (2-й сем.)
\\
\hline

ОПК-1, ОПК-4, ПК-2, ПК-3 & знать назначение, основные принципы организации и функционирования современных ОС;
знать назначение, архитектуру и состав операционных систем;
знать основные принципы организации и функционирования современных ОС;
уметь писать небольшие программы, использующие специфичные для данной ОС системные сервисы;
уметь пользоваться документацией функций операционной системы для корректного использования ее сервисов в собственных программах;
уметь устанавливать не менее двух различных современных ОС;
владеть навыками чтения, компиляции и отладки программ, использующих специфичные для данной ОС системные сервисы. & мини\-мальный & может объяснить основные цели операционных систем;
может объяснить разницу между режимом ядра и режимом пользователя;
может объяснить понятие системного вызова;
может объяснять число и тип параметров в вызовах Windows API или функций, соответствующих системным вызовам Linux, используя справку;
может описать хотя бы два алгоритма замещения страниц;
может описать хотя бы два алгоритма планирования; & удовл. (2-й сем.)
\\
\hline

ОПК-1, ОПК-4, ПК-2, ПК-3 & знать назначение, основные принципы организации и функционирования современных ОС;
знать назначение, архитектуру и состав операционных систем;
уметь устанавливать не менее двух различных современных ОС;
владеть навыками чтения, компиляции и отладки программ, использующих специфичные для данной ОС системные сервисы. & не освоено & не знает основных целей операционных систем; или
не может объяснить разницу между режимом ядра и режимом пользователя; или
не знает понятия прерывания; или
не может объяснить понятия виртуальной памяти; или
не может объяснить необходимости в файловой системе; или
не может правильно прочесть из документации прототип вызова функции Си, реализующей такие функции, как получение списка файлов в каталоге или запуск процесса; & неудовл. (2-й сем.)
\\
\hline


\end{longtable}



\subsection{Типовые контрольные задания (вопросы) для промежуточной аттестации}

\begin{longtable}{|p{15mm}|p{42mm}|p{17mm}|p{70mm}|}
\hline
\centering\small Коды оцениваемых компетенций  & \centering Оцениваемый показатель (ЗУВ) 
& \centering Тема  & \centering\arraybackslash Образец типового (тестового или практического) задания (вопроса)
\endhead
\hline

ПК-3 & 
знать этапы эволюции операционных систем; & 
1 & 
Расположите в хронологическом порядке: Linux, Windows XP, MS-DOS, Windows 95, Unix, BSD. 
\\
\hline
ПК-2 & 
знать способы представления основных видов данных в памяти ЭВМ, & 
2 & 
Какими двумя байтами в памяти представляется значение типа short int, соостветствующее числу -2000? 
\\
\hline
ПК-2 & 
знать понятия стекового кадра, соглашения о вызовах функций, статической и динамической компоновки; & 
3 & 
Объясните разницу между объявлением и определением переменной в языке Си. 
\\
\hline
ПК-3 & 
знать основные принципы организации и функционирования современных ОС; & 
4 & 
Циклическое планирование. Лотерейное планирование. 
\\
\hline
ОПК-4, ПК-2 & 
уметь писать небольшие программы, использующие специфичные для данной ОС системные сервисы;
уметь пользоваться документацией функций ОС для корректного использования ее сервисов в собственных программах; & 
6 & 
Напишите программу для Windows, перечисляющую скрытые файлы в текущей папке. 
\\
\hline
ОПК-4, ПК-2 & 
знать назначение, основные принципы организации и функционирования современных ОС; & 
5 & 
Алгоритм NRU. Алгоритм LRU, трудности его реализации. 
\\
\hline
ОПК-1 & 
уметь устанавливать не менее двух различных современных ОС; & 
6 & 
опишите разметку двух разделов для Windows с разделением всего объема жесткого диска пополам,
опишите настройку разметки разделов при установке Ubuntu. 
\\
\hline
\end{longtable}
\subsubsection*{Вопросы к экзамену за II семестр}
\begin{enumerate}
\item
Понятие операционной системы, назначение ОС, функции типичной современной ОС.
Основные виды операционных систем. Очерк истории развития ОС.
\item
Основные понятия ОС. Процессы и ресурсы; понятие прикладного программного интерфейса.
Режим ядра и режим пользователя. Прерывания.
\item
Структура ОС. Ядро, микроядро. Прерывания.
\item
Кодирование данных: беззнаковые и знаковые целые, число с плавающей точкой,
строка в однобайтовой кодовой странице, UNICODE, UTF-8.
\item
Стековые кадры. Передача параметорв в функции. Соглашения вызова.
\item
Компиляция многофайлового проекта в Си. Внешние функции и переменные.
Объектные и библиотечные файлы.	Статическое и динамическое связывание.
\item
Процессы и потоки. Структуры ОС (очереди готовности,
очереди драйверов, блоки управления процессами).
\item
Диспетчеризация и переключение контекста, роль прерываний.
\item
Параллельное исполнение. Потоки, сравнение с процессами, различные
подходы к реализации потоков в ОС.
\item
Проблемы параллельного доступа к данным. Состояния состязания.
Проблема критической секции (взаимного исключения).
\item
Механизмы синхронизации. Атомарные инструкции (TSL, XCHG), семафоры, мьютексы, мониторы.
\item
Модельные задачи межпроцессного взаимодействия: задача об обедающих философах.
\item
Модельные задачи межпроцессного взаимодействия: задача производителей и потребителей.
\item
Планирование и диспетчеризация. Алгоритмы планирования: циклическое и приоритетное планирование,
изменение приоритетов по вводу-выводу; планирование процессов и потоков.
\item
Тупики. Обнаружение тупиков.
\item
Управление памятью. Обзор видов физической памяти и аппаратных средств управления памятью.
\item
Подкачка, фрагментация. Виртуальное адресное пространство; страничная и сегментная организация памяти,
страничная ошибка.
\item
Методы учета свободной памяти.
\item
Алгоритмы замещения страниц.
\item
Файловые системы. Загрузочный сектор диска (MBR), таблицы разделов.
\item
Файловая система FAT32. Зарезервированные сектора. Кластеры. Таблица размещения файлов.
Каталоги (папки). Хранение длинного имени файла. Свободное место. Удаление файла.
Создание файла нулевой длины.
\item
Файловая система ext2. VFS. Узлы inode. Суперблоки. Организация каталогов. dentry. Блокировки совместного доступа
к файлам.
\item
Права доступа к файлам в классической модели UNIX. chmod, chown и chgrp. бит SETUID.
\item
Права доступа к файлам в NTFS. Списки контроля доступа (ACL), наследование, <<создатель-владелец>>.
Элементы запрета.
\item
Изоляция chroot. Виртуализация. Аппаратная виртуализация. Паравиртуализация. Docker-контейнеры.
\item
Сетевое взаимодействие TCP/IP. Сокеты в стиле Беркли.
\end{enumerate}



\subsection{Методические материалы, определяющие процедуры оценивания}

\textit{Лабораторные работы.}\/ Во время лабораторных занятий по каждой теме
обучающиеся должны самостоятельно написать программы, решающие задачу, поставленную в
описании лабораторной работы, либо выполнить указанные действия с целью исследовать
тот или иной аспект и составить отчет о проделанной работе. Задания очередной
лабораторной работы могут быть сданы не позднее следующего лабораторного занятия.
\par
\par
\textit{Онлайн-тестирование.}\/ Данная форма текущего контроля направлена на оценку основных
теоретических знаний обучающегося по мере освоения разделов дисциплины. Предполагает
ответы на вопросы теста через веб-формы, например Google Forms, либо
исправление и сдачу выданных <<заготовок>> (незаконечнных программ или
программ с дефектами) в автоматизированной проверяющей системе <<Мультиметр>>.
\par
\textit{Форма промежуточной аттестации: экзамен.}\/ Для получения зачета
необходимо набрать не менее 60 баллов за текущую работу в семестре, включая
посещение занятий, выполнение лабораторных работ и онлайн-тестирование.
\par
\textit{Форма промежуточной аттестации: экзамен.}\/ К экзамену допускаются студенты,
выполнившие обязательный минимум учебной работы и набравшие в семестре не менее 45~баллов.
Данный вид комплексного испытания предполагает ответ по билету, содержащему один
теоретический и два практических вопроса. Последние предполагают написание программы.
На экзамене можно набрать до 30 баллов: 9 за теоретический вопрос, 9 и 12 баллов за
практические вопросы.



\newpage
\section{Перечень основной и дополнительной учебной литературы, необходимой для освоения дисциплины}

  \begin{longtable}{|l|p{7cm}|p{18mm}|c|p{32mm}|}
  \caption*{Перечень литературы}\\
  \hline
  № & 
  \centering\small\arraybackslash Автор, название, место издания, издательство, год издания учебной литературы, вид и характеристика иных информационных ресурсов &
  \multicolumn{1}{p{18mm}|}{\centering\small\arraybackslash Наличие грифа, вид грифа} &
  \multicolumn{1}{p{21mm}|}{\centering\small\arraybackslash НБ СВФУ, кафедральная библиотека и кол-во экземпляров} & 
  \centering\small\arraybackslash Электронные издания: точка доступа к ресурсу (наименование ЭБС, ЭБ СВФУ)\\
  \hline
  \multicolumn{5}{|c|}{Основная литература}\\
  \hline
  1 &\raggedright\arraybackslash Орлов С.А., Цилькер Б.Я. Организация ЭВМ и систем. СПб.: Питер, 2014.  &  МОН РФ  &  11  &  
  \\
  \hline
  2 &\raggedright\arraybackslash Таненбаум Э., Современные операционные системы. СПб.: Питер, 2013.     &          &  35  &  
  \\
  \hline
  
  \multicolumn{5}{|c|}{Дополнительная литература}\\
  \hline
  1 &\raggedright\arraybackslash Хэвиленд К., Системное программирование в UNIX. Руководство программиста по разработке ПО. М.: ДМК, 2010  &   &  10  &  
  \\
  \hline
  
  \end{longtable}
  
\section{Перечень ресурсов информационно-телекоммуникационной сети «Интернет» (далее~--- сеть Интернет), необходимых для освоения дисциплины}
\begin{enumerate}
  \raggedright
  
  \item Microsoft. Разработка классических приложений // Центр разработки для Windows. \texttt{https://developer.microsoft.com/ru-ru/windows/desktop/develop} 
  
  \item Microsoft. Утилиты Windows Systinternals // TechNet. \texttt{https://technet.microsoft.com/ru-ru/sysinternals/} 
  
  \item OS Development Wiki. \texttt{http://wiki.osdev.org/} 
  
  \item Информационный портал OSZone.net. \texttt{http://www.oszone.net/} 
  
  \item Linux man pages. // die.net. \texttt{https://linux.die.net/man/} 
  
\end{enumerate}


\newpage
\section{Описание материально-технической базы, необходимой для осуществления образовательного процесса по дисциплине}
  
  
       Для проведения лекционных занятий требуется аудитория, оборудованная доской,  мультимедийным проектором с экраном. 
       Для проведения лабораторных занятий требуется компьютерный класс с подключением к интернету.
  
  


\section{Перечень информационных технологий, используемых при осуществлении образовательного процесса по дисциплине, включая перечень программного обеспечения
}

\subsection{Перечень информационных технологий, используемых при осуществлении образовательного процесса по дисциплине}

При осуществлении образовательного процесса по дисциплине используются следующие информационные технологии:
\begin{itemize}[nolistsep]
  
\item использование на занятиях электронных изданий (чтение лекций с использованием слайд-презентаций);
  
\item ведение учета посещаемости и выполнения учебных заданий в системе Google Docs;
  
\item написание программ на языках высокого уровня в инструментальных средах;
  
\item организация взаимодействия с обучающимися посредством электронной почты, специализированного образовательного форума Piazza;
  
\item компьютерное тестирование.
  
\end{itemize}

\subsection{Перечень программного обеспечения}
При осуществлении образовательного процесса по дисциплине используются следующее программное обеспечение:
\begin{itemize}[nolistsep]
  
\item Среда Visual Studio с компилятором Visual C++, или среда CodeBlocks с компилятором GCC;
  
\item сервер или виртуальные машины Linux с установленным компилятором GCC/g++;
  
\item интернет-браузер.
  
\end{itemize}



\newpage
\begin{center}
\section*{ЛИСТ АКТУАЛИЗАЦИИ РАБОЧЕЙ ПРОГРАММЫ ДИСЦИПЛИНЫ}
Б1.Б.23\ --- Операционные системы 
\end{center}

  \noindent
  \begin{tabular}{|p{15mm}|p{67mm}|p{25mm}|p{41mm}|}
    \hline
    \small\centering
    Учебный год 
    & \small\centering
    Внесенные изменения 
    & \small\centering
    Преподаватель (ФИО) 
    & \small\centering\arraybackslash
    Протокол заседания выпускающей кафедры (дата, номер), ФИО зав.кафедрой, подпись \\
    & & & \\\hline
    & & & \\\hline
    & & & \\\hline
    & & & \\\hline
    & & & \\\hline
    & & & \\\hline
    & & & \\\hline
    & & & \\\hline
    & & & \\\hline
    & & & \\\hline
    & & & \\\hline
    & & & \\\hline
    & & & \\\hline
    & & & \\\hline
    & & & \\\hline
    & & & \\\hline
    & & & \\\hline
    & & & \\\hline
    & & & \\\hline
    & & & \\\hline
    & & & \\\hline
    & & & \\\hline
    & & & \\\hline
    & & & \\\hline
    & & & \\\hline
    & & & \\\hline
    & & & \\\hline
    & & & \\\hline
    & & & \\\hline
    & & & \\\hline
    & & & \\\hline
    & & & \\\hline
    & & & \\\hline
    & & & \\\hline
    & & & \\\hline
    & & & \\\hline
  \end{tabular}

  \medskip\noindent\textit{В таблице указывается только характер изменений (например, изменение темы, списка источников по~теме или темам, средств промежуточного контроля) с~указанием пунктов рабочей программы. Само содержание изменений оформляется приложением по~сквозной нумерации.}

\newpage\tableofcontents

\end{document}