\documentclass[a4paper,12pt]{article}
  \usepackage[hmargin={30mm, 17mm}, vmargin={20mm,20mm}, nohead]{geometry}
  \usepackage{polyglossia, enumitem, array, longtable, multirow, tabularx, rotating, indentfirst, caption, float, ulem, titlesec }%}
  \setdefaultlanguage{russian} % выбор основного языка (для переносов)
  % шрифты 
  \defaultfontfeatures{Ligatures=TeX} % нужен для того, чтобы работали стандартные сочетания символов ---, -- << >> и т.п.
  \setmainfont{Times New Roman} 
  \setmonofont[Scale=0.9]{Consolas}
  \setsansfont{Trebuchet MS}

  

  % pdf metadata
  \usepackage[
    pdfencoding=unicode,
    pdftitle={Рабочая программа дисциплины Объектно-ориентированное программирование},
    pdfauthor={составлена }, 
    bookmarksnumbered=true, 
      colorlinks=true, linktocpage=true, linkcolor=blue, pdfpagemode=UseOutlines]
  {hyperref}

  % формат заголовков, подписей и списков
  \titleformat*{\section}{\large\bfseries\centering}
  \titleformat*{\subsection}{\bfseries}
  \captionsetup[table]{singlelinecheck=off, font=small, labelsep=period, textfont=it, format=hang, justification=raggedleft}
  \setlist[enumerate,1]{nolistsep,labelindent=0pt,leftmargin=\parindent}
  \setlist[itemize,1]{nolistsep,labelindent=20pt,leftmargin=*,label=--}

  \makeatletter

  % заполняет ширину текста полем с подчеркиванием.
  % #1 - текст слева, #2 - на подчеркивании, #3 - справа
  \newcommand{\ulfield}[3]{
  \noindent
  \begin{tabularx}{\linewidth}{@{}l@{}X@{}l@{}}
  #1\if\relax\detokenize{#1}\relax\else\,~\vrule width 0pt\fi 
  & \uline{\vrule width 0pt\hfill#2\hfill\vrule width 0pt} & 
  \if\relax\detokenize{#3}\relax\else\vrule width 0pt~\,\fi #3
  \end{tabularx}
  }

  %%% Перенос составных слов
    \XeTeXinterchartokenstate=1
    \XeTeXcharclass `\- 24
    \XeTeXinterchartoks 24 0 = {\hskip\z@skip}
    \XeTeXinterchartoks 0 24 = {\nobreak}

  \newcommand\rotleft{\rotatebox{90}}
  
  \makeatother

  \newcommand{\datefield}[1][]{\if
  \relax\detokenize{#1}\relax
  «\uline{\hspace{22pt}}»~\uline{\hspace{90pt}}\,~20\uline{\hspace{20pt}}~г.\else 
  «\uline{\hspace{18pt}}»~\uline{\hspace{60pt}}\,~20\uline{\hspace{18pt}}~г.\fi
  }


%%% Настройка содержания
\AtBeginDocument{
  \makeatletter 
  \def\@tocline#1#2#3#4#5#6#7{\relax
  \ifnum #1>\c@tocdepth % then omit
  \else
    \par \addpenalty\@secpenalty\addvspace{#2}%
    \begingroup \hyphenpenalty\@M
    \@ifempty{#4}{\@tempdima\csname r@tocindent\number#1\endcsname\relax}{\@tempdima#4\relax}%
    \parindent\z@ \leftskip#3\relax \advance\leftskip\@tempdima\relax
    \rightskip\@pnumwidth plus4em \parfillskip-\@pnumwidth
    #5\leavevmode\hskip-\@tempdima
      \ifcase #1
       \or\or \hskip 1em \or \hskip 2em \else \hskip 3em \fi%
      #6\nobreak\relax
    \dotfill\hbox to\@pnumwidth{\@tocpagenum{#7}}\par
    \nobreak
    \endgroup
  \fi}
  \makeatother
}%AtBeginDocument

\begin{document}
\sloppy
\thispagestyle{empty}

\noindent
\begin{center}
Министерство образования и науки Российской Федерации \\
Федеральное государственное автономное образовательное \\
учреждение высшего образования\\
«СЕВЕРО-ВОСТОЧНЫЙ ФЕДЕРАЛЬНЫЙ УНИВЕРСИТЕТ \\
имени М.\,К.~АММОСОВА» \\
Институт математики и информатики \\
Кафедра информационных технологий

\vspace{12mm}
\begin{flushright}
\parbox{80mm}{
УТВЕРЖДАЮ\\
Директор ИМИ\\[2mm]
\ulfield{}{}{/\,В.\,И.~Афанасьева\,/}{}\\
\datefield
\\[20mm]
}
\end{flushright}


РАБОЧАЯ ПРОГРАММА ДИСЦИПЛИНЫ
\\[2mm]
\textbf{Б1.В.ОД.2.1\ -- Объектно-ориентированное программирование} 
\\[5mm]

для программы магистратуры\\
по направлению подготовки \\
09.04.01 -- Информатика и вычислительная техника
\\[15mm]



\begin{tabular}{|p{0.3\textwidth}|p{0.3\textwidth}|p{0.3\textwidth}|}
  \hline
  ОДОБРЕНО &  ОДОБРЕНО  & РЕКОМЕНДОВАНО \\
  Заведующий кафедрой \newline разработчика &
  Заведующий выпускающей кафедрой ИТ&
  Нормоконтроль в составе ОП пройден \\
  \ulfield{}{}{\uline{/\hspace{30mm}/}} &
  \ulfield{}{}{\uline{/\hspace{30mm}/}} &
  \ulfield{}{}{\uline{/\hspace{30mm}/}} \\
  Протокол № \uline{\hspace{13pt}} от\newline \datefield[small] & 
  Протокол № \uline{\hspace{13pt}} от\newline \datefield[small] 
  
  \medskip\par
  Руководитель программы$^*$\newline
  \ulfield{}{}{\uline{/\hspace{30mm}/}}\newline \datefield[small]  
  & 
  Протокол № \uline{\hspace{13pt}} от\newline \datefield[small] \\
  \hline
  \multicolumn{2}{|p{0.625\textwidth}|}{Рекомендовано к утверждению в составе ОП\newline 
  09.04.01 «Информатика и вычислительная техника»\newline
  Председатель УМК ИМИ \uline{\hspace{21mm}} \mbox{/И.\,В.\, Николаева/}\newline 
  Протокол УМК № \uline{\hspace{12mm}} от \datefield[small]}& 
  Эксперт УМК ИМИ\newline
  \ulfield{}{}{\uline{/\hspace{30mm}/}}\newline\datefield[small]
  \\
  \hline
\end{tabular}
\par\vfill\vspace{6mm}
Якутск -- 2016

\end{center}


\newpage


\begin{center}
\section{АННОТАЦИЯ}
  {\bf к рабочей программе дисциплины\\
  Б1.В.ОД.2.1\ -- Объектно-ориентированное программирование} \\
  Трудоемкость \uline{~3~} з.~е.
\end{center}


\subsection{Цель освоения и краткое содержание дисциплины}
  
  Целью изучения диcциплины <<Объектно-ориентированное программирование>> является: знакомство с понятиями и овладение практическими навыками объектно-ориентированного программирования (ООП).
  
  
  \textit{Краткое содержание дисциплины.} Принципы ООП. Средства ООП в Си++. UML и шаблоны ОО проектирования. Обобщенное программирование и STL. Средства ООП в Java. Средства тестирования и совместной разработки ПО.
  
  



\subsection{Перечень планируемых результатов обучения по дисциплине, соотнесенных с~планируемыми результатами освоения образовательной программы}

\begin{longtable}{|p{54mm}|p{100mm}|}
  \caption{Перечень планируемых результатов обучения}\\
  \hline
  \centering
  Планируемые результаты освоения программы (содержание и коды компетенций) & 
  \centering\arraybackslash
  Планируемые результаты обучения по~дисциплине
  \\
  \hline
  
  ОК-8 : способностью к профессиональной эксплуатации современного оборудования и приборов (в соответствии с целями магистерской программы), \par 
  
  ОПК-1 : способностью воспринимать математические, естественнонаучные, социально-экономические и профессиональные знания, умением самостоятельно приобретать, развивать и применять их для решения нестандартных задач, в том числе в новой или незнакомой среде и в междисциплинарном контексте, \par 
  
  ПК-18 : способностью к разработке программного обеспечения для создания трехмерных изображений, \par 
  
  ПК-6 : пониманием существующих подходов к верификации моделей программного обеспечения (ПО)
  & 
  В результате изучения дисциплины обучающийся должен:\newline
  \emph{знать:}
  принципы объектно-ориентированной разработки программ;
  

  \emph{уметь:}
  \begin{itemize}[leftmargin=12pt]
    \item объяснять принципы объектно-ориентированного подхода к проектированию программного обеспечения; 
    \item читать UML-диаграммы классов, вариантов использования и последовательностей; 
    \item читать код классов чужих программ на языках Java и Си++; 
    \item распознавать простые случаи использования паттернов объектно-ориентированного проектирования; 
    \item использовать наследование для реализации отношений <<является>> и <<имеет>>; 
  \end{itemize}
  

  \emph{владеть навыками:}
  \begin{itemize}[leftmargin=12pt]
    \item составления несложных UML-диаграмм классов, вариантов использования и последовательностей; 
    \item проектирования, релизации и тестирования простых объектно-ориентированных программ на Java и Си++; 
    \item использования контейнерных классов стандартной библиотеки шаблонов языка Си++. 
  \end{itemize}
  
  \\
  \hline
  \end{longtable}


\subsection{Место дисциплины в~структуре образовательной программы}

  \begin{table}[H]
  \setlength\arraycolsep{3pt}
  \caption{Содержательно-логические связи дисциплины}
  \begin{tabular}{|l|p{18ex}|*{2}{p{23ex}|}}
  \hline
  \multicolumn{1}{|c|}{\multirow{2}{13ex}{\centering Индекс \linebreak дисциплины}} &
  \multicolumn{1}{c|}{\multirow{2}{18ex}{\centering Наименование \linebreak дисциплины}} & 
  \multicolumn{2}{p{46ex}|}{\centering Коды учебных дисциплин, практик} \\
  \cline{3-4}
   & & 
  \centering на которые опирается содержание дисциплины & 
  \centering\arraybackslash для которых содержание дисциплины выступает опорой
  \\ \hline
  Б1.В.ОД.2.1 & Объектно-ориентированное программирование 
  & 
  \raggedright
  
  -- 
  & 
  \raggedright\arraybackslash
  
  Б1.В.ДВ.1.1 -- Распределенные объектные технологии,
  Б1.В.ДВ.1.2 -- Объектные базы данных,
  Б1.В.ДВ.3.1 -- Программирование на языке Питон,
  Б1.В.ДВ.3.2 -- Разработка Java-приложений,
  Б1.В.ДВ.4.1 -- Разработка клиент-серверных приложений,
  Б1.В.ДВ.5.1 -- Алгоритмы цифровой обработки изображений,
  Б1.В.ДВ.5.2 -- CASE-системы разработки ПО 
  \\ \hline
  \end{tabular}
  \end{table}


\subsection{Язык преподавания} 
  Русский.
  



\newpage

\section{Объем дисциплины в зачетных единицах с указанием количества академических часов, выделенных на контактную работу обучающихся с преподавателем (по~видам учебных занятий) и~на~самостоятельную работу обучающихся}

\begin{table}[H]
\caption{Выписка из учебного плана} 
\begin{tabular}{|p{9cm}|c|c|}
\hline
Код и название дисциплины по учебному плану & \multicolumn{2}{p{6cm}|}{Б1.В.ОД.2.1\ -- Объектно-ориентированное программирование }\\
\hline
Курс изучения &\multicolumn{2}{c|}{ 1 }\\
\hline
Семестр(ы) изучения &\multicolumn{2}{c|}{ 1 }\\
\hline
Форма промежуточной аттестации (зачет/экзамен) &\multicolumn{2}{c|}{ зачет }\\
\hline
Курсовой проект / курсовая работа (указать вид работы при наличии в учебном плане), семестр выполнения &\multicolumn{2}{c|}{ }\\
\hline
Трудоемкость (в ЗЕТ) &\multicolumn{2}{c|}{ 3 (3) }\\
\hline
{\bf Трудоемкость (в часах)} (сумма строк №1, 2, 3), в~т.~ч.:& \multicolumn{2}{c|}{108}\\
\hline
\textbf{№\,1. Контактная работа обучающихся с преподавателем (КР),} в часах:
& \multicolumn{1}{p{3cm}|}{\centering Объем аудиторной работы, в часаx}
& \multicolumn{1}{p{3cm}|}{\centering\arraybackslash В~т.\,ч. с~применением ДОТ или ЭО, в~часах}\\
\hline  
Объем работы (в часах) (1.1.+1.2.+1.3.)& 59 & \\
\hline
1.1. Занятия лекционного типа (лекции) & 12 & \\
\hline
1.2. Занятия семинарского типа, всего, в т.ч.: & & \\
\hline
- семинары (практические занятия, коллоквиумы и~т.~п.)  & – & \\
\hline
- лабораторные работы& 42 & \\
\hline
- практикумы & & \\
\hline
1.3. КСР (контроль самостоятельной работы, консультации)& 5 & \\
\hline
{\bf №\,2. Самостоятельная работа обучающихся (СРС) (в часах)}& \multicolumn{2}{c|}{49}\\
\hline
{\bf №\,3. Количество часов на экзамен (при наличии экзамена в учебном плане)}& \multicolumn{2}{c|}{–}\\
\hline
\end{tabular}
\end{table}



\newpage
\section{Содержание дисциплины, структурированное по~темам с~указанием отведенного на~них количества академических часов и~видов учебных занятий}
\subsection{Распределение часов по~темам и~видам учебных занятий}
\begin{longtable}{|>{\raggedright\arraybackslash}p{59mm}|c|c|c|c|c|c|c|c|c|c|c|}
\caption{}
\\
\hline
 & & 
\multicolumn{9}{c|}{Контактная работа, в часах} & 
\\
\cline{3-11} 
\raisebox{18mm}{Тема}&
\rotleft{Всего часов} &
\rotleft{Лекции} &
\rotleft{из них с прим-м  ЭО и ДОТ} &
\rotleft{\parbox{5cm}{\raggedright\arraybackslash Семинары  (практические занятия, коллоквиумы)}} &
\rotleft{из них с прим-м  ЭО и ДОТ} &
\rotleft{Лабораторные работы} &
\rotleft{из них с прим-м  ЭО и ДОТ} &
\rotleft{Практикумы} &
\rotleft{из них с прим-м  ЭО и ДОТ} &
\rotleft{КСР (консультации)} & 
\rotleft{Часы СРС}
\\
\hline
Тема 1. Принципы ООП.                                     & 6 & 2 & 0 & 0 & 0 & 4 & 0 & 0 & 0 & 0 & 0 \\ 
\hline
Тема 2. Средства ООП в Си++.                              & 25 & 2 & 0 & 0 & 0 & 10 & 0 & 0 & 0 & 1 & 12 \\ 
\hline
Тема 3. Обобщенное программирование и STL                 & 17 & 2 & 0 & 0 & 0 & 6 & 0 & 0 & 0 & 1 & 8 \\ 
\hline
Тема 4. Средства ООП в Java.                              & 30 & 2 & 0 & 0 & 0 & 12 & 0 & 0 & 0 & 1 & 15 \\ 
\hline
Тема 5. UML и шаблоны ОО проектирования                   & 17 & 2 & 0 & 0 & 0 & 6 & 0 & 0 & 0 & 1 & 8 \\ 
\hline
Тема 6. Средства тестирования и совместной разработки ПО  & 13 & 2 & 0 & 0 & 0 & 4 & 0 & 0 & 0 & 1 & 6 \\ 
\hline
ВСЕГО ЧАСОВ & 108 & 12 & 0 & 0 & 0 & 42 & 0 & 0 & 0 & 5 & 49 \\ 

\hline
\end{longtable}

\subsection{Содержание тем программы дисциплины} 


\textbf{Тема 1. Принципы ООП.                                    }\\
Эволюция методологий программирования. Основные принципы объектного подхода. Наследование и отношения между объектами.

\textbf{Тема 2. Средства ООП в Си++.                             }\\
Пространства имен. Перегрузка. Классы. Доступ к членам класса. Наследование. Виртуальные методы. Особенности копирующих конструкторов. Особенности перегрузки операторов. Проблемы множественного наследования. Обработка исключений

\textbf{Тема 3. Обобщенное программирование и STL                }\\
Шаблоны функций. Шаблоны классов. Наследование и шаблоны. Контейнеры STL. Итераторы. Алгоритмы. Потоковые классы.

\textbf{Тема 4. Средства ООП в Java.                             }\\
Базовые типы Java, объекты, классы, пакеты. Доступ к членам класса. Интерфейсы. Наследование. Виртуальные методы. Многопоточные приложения.

\textbf{Тема 5. UML и шаблоны ОО проектирования                  }\\
Унифицированный язык проектирования UML. Основные виды диаграмм UML. Идея шаблона проектирования, различные подходы к выделению шаблонов. Шаблоны <<Наблюдатель>>, <<Итератор>>, <<Одиночка>>, <<Адаптер>>, <<Фасад>>, <<Фабрика>>.

\textbf{Тема 6. Средства тестирования и совместной разработки ПО }\\
Библиотека JUnit. Подходы к совместной разработке ПО с использованием средств управления версиями.
 

\subsection{Формы и методы проведения занятий, применяемые учебные технологии}
При проведении занятий и организации СРС используются традиционные технологии сообщающего обучения, предполагающие передачу информации в~готовом виде: проведение лекционных занятий, самостоятельная работа с~источниками. Предусмотрено использование активных и интерактивных форм обучения с целью формирования и развития профессиональных навыков студентов --- выполнение практических работ с применением компьютерных технологий. 



\section{Перечень учебно-методического обеспечения для самостоятельной работы обучающихся по дисциплине}
\begin{longtable}{|l|>{\raggedright\arraybackslash}p{40mm}|>{\raggedright\arraybackslash}p{54mm}|c|>{\raggedright\arraybackslash}p{30mm}|}
\hline
№ & \centering Наименование раздела (темы) дисциплины & 
\centering Вид СРС & \multicolumn{1}{p{14mm}|}{\centering Трудо\-емкость (в часах)} & \centering\arraybackslash Формы и методы контроля\\
\hline
1 & Средства ООП в Си++.                             & Прохождение онлайн-курса Intermediate C++. Сдача домашнего задания  & 12 & Предъявление веб-страницы с информацией о прохождении курса. Сдача работающего кода. \\ 
\hline
2 & Обобщенное программирование и STL                & Сдача домашнего задания   &  8 & Сдача работающего приложения \\ 
\hline
3 & Средства ООП в Java.                             & Прохождение начальных разделов онлайн-курса Software Construction in Java. Сдача домашнего задания   & 15 & Предъявление веб-страницы с информацией о прохождении разделов онлайн-курса. Сдача работающего кода. \\ 
\hline
4 & UML и шаблоны ОО проектирования                  & Сдача домашнего задания   & 8  & Сдача диаграмм. Сдача работающего приложения \\ 
\hline
5 & Средства тестирования и совместной разработки ПО & Сдача домашнего задания   &  6 & Сдача тестов к приложению <<Быки и коровы>>. Публикация исходного кода приложения <<Быки и Коровы>> в онлайн-репозиториии \\ 
\hline
 & ИТОГО                                            &  & 49 &  \\ 

\hline
\end{longtable}


\section{Методические указания для обучающихся по освоению дисциплины}
В связи с небольшим объемом аудиторных, особенно лекционных, часов, важное
значение в освоении дисциплины имеет самостоятельная работа. Она предполагает
в том числе и сдачу частей онлайн-курсов на английском языке. Это
требует самостоятельности и ответственности.
\par
В диагностическом разделе дисциплины приведены тесты по каждому модулю
дисциплины, которые необходимо выполнить для закрепления теоретических
знаний.
\par
Последовательное и добросовестное изучение курса является основой для
выработки практических навыков использования объектно-ориентированного
подхода к программированию, который является основным средством
управления сложностью во многих реальных программных проектах
в областях деятельности, предполагаемых стандартом подготовки по направлению
«Информатика и вычислительная техника».



\subsubsection*{Рейтинговый регламент по дисциплине}
\begin{longtable}{|>{\raggedright\arraybackslash}p{110mm}|r|r|}
\hline
\centering\arraybackslash Вид выполняемой учебной работы (контролирующие мероприятия) & 
\multicolumn{1}{p{20mm}|}{\centering\arraybackslash Количество баллов (min)} & 
\multicolumn{1}{p{20mm}|}{\centering\arraybackslash Количество баллов (max)} \\
\hline
Посещаемость & 5 & 10 \\ 
\hline
Домашние задания, онлайн курсы & 25 & 45 \\ 
\hline
Практические занятия & 10 & 15 \\ 
\hline
Тестирование & 20 & 30 \\ 
\hline
Количество баллов для получения зачета (min-max) & 60 & 100 \\ 

\hline
\end{longtable}

\section{Фонд оценочных средств для проведения промежуточной аттестации обучающихся по дисциплине}

\subsection{Показатели, критерии и шкала оценивания}

\begin{longtable}{|p{15mm}|p{53mm}|p{16mm}|p{43mm}|p{14mm}|}
\hline
  \centering\small Коды оцениваемых компетенций
& \centering Показатель оценивания (дескриптор) (по п.1.2) 
& \centering\small Уровни освоения 
& \centering Критерий оценивания 
& \centering\small\arraybackslash Оценка
\\
\hline

\multirow{2}{15mm}{ОК-8, ОПК-1, ПК-18, ПК-6}
&
\multirow{2}{53mm}{\parbox{53mm}{%
\vrule width 0pt height 10pt \emph{знать:}\newline
принципы объектно-ориентированной разработки программ; \newline
\emph{уметь:}\newline
объяснять принципы объектно-ориентированного подхода к проектированию программного обеспечения; читать UML-диаграммы классов, вариантов использования и последовательностей; читать код классов чужих программ на языках Java и Си++; распознавать простые случаи использования паттернов объектно-ориентированного проектирования; использовать наследование для реализации отношений <<является>> и <<имеет>>; \newline
\emph{владеть навыками:}\newline
составления несложных UML-диаграмм классов, вариантов использования и последовательностей; проектирования, релизации и тестирования простых объектно-ориентированных программ на Java и Си++; использования контейнерных классов стандартной библиотеки шаблонов языка Си++. 
}}
& 
освоено & способен выполнять большинство задач из следующего списка:
объяснять понятия абстракции, инкапсуляции, наследования и полиморфизма, а также
пояснять их пользу на примерах в языках Си++ и Java;
реализовывать класс по UML-диаграмме класса;
читать код классов чужих программ на языках Java и Си++;
применять паттерн <<фабричный метод>>
реализовывать на языках Java и Cи++ класс, наследующий от другого несложного класса для реализации отношения <<является>> ;
реализовывать на языке Си++ класс, наследующий от нескольких классов STL для реализации отношения <<имеет>>;
составлять UML-диаграмму вариантов использования по текстовому описанию;
проектировать, релизовать и тестировать простые объектно-ориентированные программы на Java и Си++;
писать программы, использующие потоковые классы, классы {\tt vector} и {\tt map}. & зачтено 
\\ 

\cline{3-5}
& & не освоено & не способен выполнить три и более пунктов из вышеперечисленного\linebreak~\linebreak & не зачтено 
\\

\hline

\end{longtable}



\subsection{Типовые контрольные задания (вопросы) для промежуточной аттестации}

\begin{longtable}{|p{15mm}|p{42mm}|p{17mm}|p{70mm}|}
\hline
\centering\small Коды оцениваемых компетенций  & \centering Оцениваемый показатель (ЗУВ) 
& \centering Тема  & \centering\arraybackslash Образец типового (тестового или практического) задания (вопроса)
\\
\hline

ОК-8, ОПК-1 & 
знать принципы объектно-ориентированной разработки программ & 
Принципы ООП & 
Зачем нужны абстрактные классы?
а)~Чтобы нельзя было создавать экземпляры данного класса\newline
б)~Чтобы от класса могли наследовать только другие абстрактные классы\newline
в)~Чтобы класс-наследник обязательно назывался по-другому\newline
г)~Чтобы все функции в классах-наследниках были виртуальными 
\\
\hline
ПК-18 & 
объяснять принципы объектно-ориентированного подхода к проектированию программного обеспечения & 
Принципы ООП; Средства ООП в Си++; Средства ООП в Java. & 
Объясните, как используется полиморфизм в выбранном языке,
реализовав метод вычисления объема для подклассов
<<тетраэдр>> и <<параллелепипед>> класса <<многогранник>>, 
\\
\hline
ОПК-1 & 
знать принципы объектно-ориентированной разработки программ; & 
Средства ООП в Си++. & 
Реализуйте на Си++ класс <<преподаватель>> и унаследуйте от него
классы <<постоянный сотрудник>> и <<преподаватель с почасовой оплатой труда>>,
класс <<преподаватель>> должен иметь виртуальную функцию  расчета
месячной оплаты труда {\tt compensation()}, определенную
как оклад, умноженный на коэффициент, для первого классах-наследника
или число проведенных часов, умноженных на ставку почасовой оплаты, для второго. 
\\
\hline
ПК-6 & 
проектирования, релизации и тестирования простых объектно-ориентированных программ на Java и Си++; & 
Средства тестирования и совместной разработки ПО & 
Напишите разумные модульные тесты JUnit для двух методов класса, реализованного вами в вопросе 2 (или его Java-эквивалента). 
\\
\hline
\end{longtable}



\subsection{Методические материалы, определяющие процедуры оценивания}

Форма промежуточной аттестации: зачет
\par
Данный вид комплексного испытания предполагает последовательное выполнение
всех форм текущего контроля, таких, как тесты, прохождение онлайн-курсов
и~выполнение практических заданий.
\par
Тестирование. Данная форма контроля направлена на оценку основных
теоретических знаний обучающегося по мере освоения основных разделов дисциплины.
\par
Контрольные работы. В этой форме промежуточного контроля проверяются способности
обобщенного анализа имеющихся теоретических знаний и умение пользоваться
специальной литературой. Во время выполнения контрольной работы разрешается
пользоваться справочной литературой



\newpage
\section{Перечень основной и дополнительной учебной литературы, необходимой для освоения дисциплины}

  \begin{longtable}{|l|p{7cm}|p{18mm}|c|p{32mm}|}
  \caption*{Перечень литературы}\\
  \hline
  № & 
  \centering\small\arraybackslash Автор, название, место издания, издательство, год издания учебной литературы, вид и характеристика иных информационных ресурсов &
  \multicolumn{1}{p{18mm}|}{\centering\small\arraybackslash Наличие грифа, вид грифа} &
  \multicolumn{1}{p{21mm}|}{\centering\small\arraybackslash НБ СВФУ, кафедральная библиотека и кол-во экземпляров} & 
  \centering\small\arraybackslash Электронные издания: точка доступа к ресурсу (наименование ЭБС, ЭБ СВФУ)\\
  \hline
  \multicolumn{5}{|c|}{Основная литература}\\
  \hline
  1 &\raggedright\arraybackslash Павловская Т. А. С/С++ . Программирование на языке высокого уровня. М.: Питер, 2013  &  МОН РФ  &  14  &  
  \\
  \hline
  
  \multicolumn{5}{|c|}{Дополнительная литература}\\
  \hline
  1 &\raggedright\arraybackslash Слободчикова А. А. Практикум по объектно-ориентированному анализу и проектированию с помощью языка UML. Якутск: СВФУ, 2007  &   &  14  &   
  \\
  \hline
  2 &\raggedright\arraybackslash Крылов Е. В. Технология, надежность и качество программного обеспечения. Техника разработки программ. М.: Высшая школа, 2007  &  УМО вузов России &  30  &  
  \\
  \hline
  3 &\raggedright\arraybackslash Плаугер П. Дж., Ли М., Степанов А. STL --- стандартная библиотека шаблонов С++ . СПб: БХВ-Петербург, 2004  &   &  6 &  
  \\
  \hline
  
  \end{longtable}
  
\section{Перечень ресурсов информационно-телекоммуникационной сети «Интернет» (далее сеть-Интернет), необходимых для освоения дисциплины}
\begin{enumerate}
  
  \item Справочник по языку C++ и стандартной библиотеке шаблонов STL. // The C++ Resources Network -- Режим доступа: http://cplusplus.com/ 
  
  \item Онлайн-курс Software Construction in Java. -- Режим доступа: \\https://www.edx.org/course/software-construction-java-mitx-6-005-1x 
  
  \item Менеджер исходного кода Git // Git source sontrol manager -- Режим доступа: http://git-scm.com/ 
  
  \item Онлайн-курс Microsoft DEV210-2x: Intermediate C++. -- Режим доступа: \\https://www.edx.org/course/intermediate-c-microsoft-dev210-2x 
  
\end{enumerate}


\newpage
\section{Описание материально-технической базы, необходимой для осуществления образовательного процесса по дисциплине}
  
  
       Для проведения лекционных занятий требуется аудитория, оборудованная доской,  мультимедийным проектором с экраном. 
       Для проведения лабораторных занятий требуется компьютерный класс с подключением к интернету.
  
  


\section{Перечень информационных технологий, используемых при осуществлении образовательного процесса по дисциплине, включая перечень программного обеспечения
}

\subsection{Перечень информационных технологий, используемых при осуществлении образовательного процесса по дисциплине}

При осуществлении образовательного процесса по дисциплине используются следующие информационные технологии:
\begin{itemize}[nolistsep]
  
\item использование на занятиях электронных изданий (чтение лекций с использованием слайд-презентаций);
  
\item ведение учета посещаемости и выполнения учебных заданий в системе Google Docs;
  
\item разработка обучающимися программ на языках Си++ и Java;
  
\item организация взаимодействия с обучающимися посредством электронной почты, специализированного образовательного форума Piazza;
  
\item компьютерное тестирование.
  
\end{itemize}

\subsection{Перечень программного обеспечения}
При осуществлении образовательного процесса по дисциплине используются следующее программное обеспечение:
\begin{itemize}[nolistsep]
  
\item среда разработки Microsoft Visual Studio или Code::Blocks с компилятором g++;
  
\item среда разработки IntelliJ IDEA или NetBeans;
  
\item Java Development Kit и библиотека JUnit;
  
\item менеджер версий Git;
  
\item интернет-браузер.
  
\end{itemize}



\newpage
\begin{center}
\section*{ЛИСТ АКТУАЛИЗАЦИИ РАБОЧЕЙ ПРОГРАММЫ ДИСЦИПЛИНЫ}
Б1.В.ОД.2.1\ --- Объектно-ориентированное программирование 
\end{center}

  \noindent
  \begin{tabular}{|p{15mm}|p{67mm}|p{25mm}|p{41mm}|}
    \hline
    \small\centering
    Учебный год 
    & \small\centering
    Внесенные изменения 
    & \small\centering
    Преподаватель (ФИО) 
    & \small\centering\arraybackslash
    Протокол заседания выпускающей кафедры (дата, номер), ФИО зав.кафедрой, подпись \\
    & & & \\\hline
    & & & \\\hline
    & & & \\\hline
    & & & \\\hline
    & & & \\\hline
    & & & \\\hline
    & & & \\\hline
    & & & \\\hline
    & & & \\\hline
    & & & \\\hline
    & & & \\\hline
    & & & \\\hline
    & & & \\\hline
    & & & \\\hline
    & & & \\\hline
    & & & \\\hline
    & & & \\\hline
    & & & \\\hline
    & & & \\\hline
    & & & \\\hline
    & & & \\\hline
    & & & \\\hline
    & & & \\\hline
    & & & \\\hline
    & & & \\\hline
    & & & \\\hline
    & & & \\\hline
    & & & \\\hline
    & & & \\\hline
    & & & \\\hline
    & & & \\\hline
    & & & \\\hline
    & & & \\\hline
    & & & \\\hline
    & & & \\\hline
    & & & \\\hline
  \end{tabular}

  \medskip\noindent\textit{В таблице указывается только характер изменений (например, изменение темы, списка источников по~теме или темам, средств промежуточного контроля) с~указанием пунктов рабочей программы. Само содержание изменений оформляется приложением по~сквозной нумерации.}

\newpage\tableofcontents

\end{document}